%%% Реализация библиографии пакетами biblatex и biblatex-gost с использованием движка biber %%%


\WarningFilter{biblatex-gost}{You set maxbibnames or maxcitenames}

\usepackage[autostyle=try]{csquotes} % biblatex рекомендует его подключать. Пакет для оформления сложных блоков цитирования.
%%% Загрузка пакета с основными настройками %%%
\makeatletter
\ifnumequal{\value{draft}}{0}{% Чистовик
\usepackage[%
backend=biber,% движок
bibencoding=utf8,% кодировка bib файла
sorting=none,% настройка сортировки списка литературы
%style=apa,% стиль цитирования и библиографии (по ГОСТ)
style=gost-numeric,% стиль цитирования и библиографии (по ГОСТ)
language=autobib,% получение языка из babel/polyglossia, default: autobib % если ставить autocite или auto, то цитаты в тексте с указанием страницы, получат указание страницы на языке оригинала
autolang=other,% многоязычная библиография
clearlang=true,% внутренний сброс поля language, если он совпадает с языком из babel/polyglossia
defernumbers=true,% нумерация проставляется после двух компиляций, зато позволяет выцеплять библиографию по ключевым словам и нумеровать не из большего списка
%sortcites=true,% сортировать номера затекстовых ссылок при цитировании (если в квадратных скобках несколько ссылок, то отображаться будут отсортированно, а не абы как)
%doi=false,% Показывать или нет ссылки на DOI
isbn=false,% Показывать или нет ISBN, ISSN, ISRN
%movenames=false,
maxnames=4,
minnames=3,
%mincitenames=1, maxcitenames=1, uniquelist=true,
%datamodel=ukrainian-gost,
]{biblatex}[2016/09/17]
\ltx@iffilelater{biblatex-gost.def}{2017/05/03}%
{\toggletrue{bbx:gostbibliography}%
\renewcommand*{\revsdnamepunct}{\addcomma}}{}
}{%Черновик
\usepackage[%
backend=biber,% движок
bibencoding=utf8,% кодировка bib файла
sorting=none,% настройка сортировки списка литературы
]{biblatex}[2016/09/17]%
}
\makeatother

\DeclareLanguageMapping{ukrainian}{../biblio/ukrainian}

\ifnumgreater{\value{usefootcite}}{0}{
    \ExecuteBibliographyOptions{autocite=footnote}
    \newbibmacro*{cite:full}{%
        \printtext[bibhypertarget]{%
            \usedriver{%
                \DeclareNameAlias{sortname}{default}%
            }{%
                \thefield{entrytype}%
            }%
        }%
        \usebibmacro{shorthandintro}%
    }
    \DeclareCiteCommand{\smartcite}[\mkbibfootnote]{%
        \usebibmacro{prenote}%
    }{%
        \usebibmacro{citeindex}%
        \usebibmacro{cite:full}%
    }{%
        \multicitedelim%
    }{%
        \usebibmacro{postnote}%
    }
}{}

%%% Подключение файлов bib %%%
\addbibresource[label=other]{../biblio/othercites.bib}
\addbibresource[label=vak]{../biblio/authorpapersVAK.bib}
\addbibresource[label=papers]{../biblio/authorpapers.bib}
\addbibresource[label=conf]{../biblio/authorconferences.bib}
\addbibresource[label=patents]{../biblio/authorpatents.bib}

%http://tex.stackexchange.com/a/141831/79756
%There is a way to automatically map the language field to the langid field. The following lines in the preamble should be enough to do that.
%This command will copy the language field into the langid field and will then delete the contents of the language field. The language field will only be deleted if it was successfully copied into the langid field.
\DeclareSourcemap{ %модификация bib файла перед тем, как им займётся biblatex 
    \maps{
        \map{% перекидываем значения полей language в поля langid, которыми пользуется biblatex
            \step[fieldsource=language, fieldset=langid, origfieldval, final]
            \step[fieldset=language, null]
        }
        \map[overwrite]{% перекидываем значения полей shortjournal, если они есть, в поля journal, которыми пользуется biblatex
            \step[fieldsource=shortjournal, final]
            \step[fieldset=journal, origfieldval]
        }
        \map[overwrite]{% перекидываем значения полей shortbooktitle, если они есть, в поля booktitle, которыми пользуется biblatex
            \step[fieldsource=shortbooktitle, final]
            \step[fieldset=booktitle, origfieldval]
        }
        \map[overwrite, refsection=0]{% стираем значения всех полей addendum
            \perdatasource{biblio/authorpapersVAK.bib}
            \perdatasource{biblio/authorpapers.bib}
            \perdatasource{biblio/authorconferences.bib}
            \step[fieldsource=addendum, final]
            \step[fieldset=addendum, null] %чтобы избавиться от информации об объёме авторских статей, в отличие от автореферата
        }
        \map{% перекидываем значения полей numpages в поля pagetotal, которыми пользуется biblatex
            \step[fieldsource=numpages, fieldset=pagetotal, origfieldval, final]
            \step[fieldset=pagestotal, null]
        }
        \map{% если в поле medium написано "Электронный ресурс", то устанавливаем поле media, которым пользуется biblatex, в значение eresource.
            \step[fieldsource=medium,
            match=\regexp{Электронный\s+ресурс},
            final]
            \step[fieldset=media, fieldvalue=eresource]
        }
        \map{% 
            \pertype{online}
            \step[fieldset=media, fieldvalue=eresource]
        }
        \ifthenelse{\equal{\thegoststyle}{2006}}
        	{\map{% 
        	    \step[fieldset=media, fieldvalue=text]
        	}}
        	{}
        \map[overwrite]{% стираем значения всех полей issn
            \step[fieldset=issn, null]
        }
        \map[overwrite]{% стираем значения всех полей abstract, поскольку ими не пользуемся, а там бывают "неприятные" латеху символы
            \step[fieldsource=abstract]
            \step[fieldset=abstract,null]
        }
        \map[overwrite]{ % переделка формата записи даты
            \step[fieldsource=urldate,
            match=\regexp{([0-9]{2})\.([0-9]{2})\.([0-9]{4})},
            replace={$3-$2-$1$4}, % $4 вставлен исключительно ради нормальной работы программ подсветки синтаксиса, которые некорректно обрабатывают $ в таких конструкциях
            final]
        }
        \map[overwrite]{ % добавляем ключевые слова, чтобы различать источники
            \perdatasource{../biblio/othercites.bib}
            \step[fieldset=keywords, fieldvalue={biblioother,bibliofull}]
        }
        \map[overwrite]{ % добавляем ключевые слова, чтобы различать источники
            \perdatasource{../biblio/authorpapersVAK.bib}
            \step[fieldset=keywords, fieldvalue={biblioauthorvak,biblioauthor,bibliofull}]
            \step[fieldset=url,null]
            \step[fieldset=doi,null]
        }
        \map[overwrite]{ % добавляем ключевые слова, чтобы различать источники
            \perdatasource{../biblio/authorpapers.bib}
            \step[fieldset=keywords, fieldvalue={biblioauthornotvak,biblioauthor,bibliofull}]
            \step[fieldset=url,null]
            \step[fieldset=doi,null]
        }
        \map[overwrite]{ % добавляем ключевые слова, чтобы различать источники
            \perdatasource{../biblio/authorconferences.bib}
            \step[fieldset=keywords, fieldvalue={biblioauthorconf,biblioauthor,bibliofull}]
            \step[fieldset=url,null]
            \step[fieldset=doi,null]
        }
        \map[overwrite]{ % добавляем ключевые слова, чтобы различать источники
            \perdatasource{../biblio/authorpatents.bib}
            \step[fieldset=keywords, fieldvalue={biblioauthorpatents,biblioauthor,bibliofull}]
        }
%        \map[overwrite]{% стираем значения всех полей series
%            \step[fieldset=series, null]
%        }
        \map[overwrite]{% перекидываем значения полей howpublished в поля organization для типа online
            \step[typesource=online, typetarget=online, final]
            \step[fieldsource=howpublished, fieldset=organization, origfieldval]
            \step[fieldset=howpublished, null]
        }
        % Так отключаем [Электронный ресурс]
%        \map[overwrite]{% стираем значения всех полей media=eresource
%            \step[fieldsource=media,
%            match={eresource},
%            final]
%            \step[fieldset=media, null]
%        }
		\map{
			\step[fieldsource=year, final]
			\step[fieldset=pubstate, null]
		}
		\map{
			\step[fieldsource=date, final]
			\step[fieldset=pubstate, null]
		}     
		\map{% стираем значения полей url если есть doi
			\step[fieldsource=doi, final]
			\step[fieldset=url, null]
		}                       
    }
}

%%% Убираем неразрывные пробелы перед двоеточием и точкой с запятой %%%
%\makeatletter
%\ifnumequal{\value{draft}}{0}{% Чистовик
%    \renewcommand*{\addcolondelim}{%
%      \begingroup%
%      \def\abx@colon{%
%        \ifdim\lastkern>\z@\unkern\fi%
%        \abx@puncthook{:}\space}%
%      \addcolon%
%      \endgroup}
%
%    \renewcommand*{\addsemicolondelim}{%
%      \begingroup%
%      \def\abx@semicolon{%
%        \ifdim\lastkern>\z@\unkern\fi%
%        \abx@puncthook{;}\space}%
%      \addsemicolon%
%      \endgroup}
%}{}
%\makeatother

%%% Правка записей типа thesis, чтобы дважды не писался автор
%\ifnumequal{\value{draft}}{0}{% Чистовик
%\DeclareBibliographyDriver{thesis}{%
%  \usebibmacro{bibindex}%
%  \usebibmacro{begentry}%
%  \usebibmacro{heading}%
%  \newunit
%  \usebibmacro{author}%
%  \setunit*{\labelnamepunct}%
%  \usebibmacro{thesistitle}%
%  \setunit{\respdelim}%
%  %\printnames[last-first:full]{author}%Вот эту строчку нужно убрать, чтобы автор диссертации не дублировался
%  \newunit\newblock
%  \printlist[semicolondelim]{specdata}%
%  \newunit
%  \usebibmacro{institution+location+date}%
%  \newunit\newblock
%  \usebibmacro{chapter+pages}%
%  \newunit
%  \printfield{pagetotal}%
%  \newunit\newblock
%  \usebibmacro{doi+eprint+url+note}%
%  \newunit\newblock
%  \usebibmacro{addendum+pubstate}%
%  \setunit{\bibpagerefpunct}\newblock
%  \usebibmacro{pageref}%
%  \newunit\newblock
%  \usebibmacro{related:init}%
%  \usebibmacro{related}%
%  \usebibmacro{finentry}}
%}{}

%\newbibmacro{string+doi}[1]{% новая макрокоманда на простановку ссылки на doi
%    \iffieldundef{doi}{#1}{\href{http://dx.doi.org/\thefield{doi}}{#1}}}

\ifnumequal{\value{draft}}{0}{% Чистовик
%\renewcommand*{\mkgostheading}[1]{\usebibmacro{string+doi}{#1}} % ссылка на doi с авторов. стоящих впереди записи
\renewcommand*{\mkgostheading}[1]{#1} % только лишь убираем курсив с авторов
}{}
%\DeclareFieldFormat{title}{\usebibmacro{string+doi}{#1}} % ссылка на doi с названия работы
%\DeclareFieldFormat{journaltitle}{\usebibmacro{string+doi}{#1}} % ссылка на doi с названия журнала
%%% Тире как разделитель в библиографии традиционной руской длины:
\renewcommand*{\newblockpunct}{\addperiod\addnbspace\cyrdash\space\bibsentence}
%%% Убрать тире из разделителей элементов в библиографии:
%\renewcommand*{\newblockpunct}{%
%    \addperiod\space\bibsentence}%block punct.,\bibsentence is for vol,etc.

%%% Возвращаем запись «Режим доступа» %%%
\DefineBibliographyStrings{english}{%
    urlfrom = {Mode of access}
}
\DeclareFieldFormat{url}{\bibstring{urlfrom}\addcolon\space\href{#1}{\nolinkurl{\thefield{urlraw}}}}

\DefineBibliographyStrings{english}{%
    mediatext = {text},
    mediaeresource = {electronic resource}
}
\NewBibliographyString{fromukrainian}
\DefineBibliographyStrings{russian}{%
	fromukrainian = {{с\addabbrvspace укр\adddot}},
}

%%% В списке литературы обозначение одной буквой диапазона страниц англоязычного источника %%%
\DefineBibliographyStrings{english}{%
    pages = {p\adddot} %заглавность буквы затем по месту определяется работой самого biblatex
}

%%% В ссылке на источник в основном тексте с указанием конкретной страницы обозначение одной большой буквой %%%
%\DefineBibliographyStrings{russian}{%
%    page = {C\adddot}
%}

%%% Исправление длины тире в диапазонах %%%
% \cyrdash --- тире «русской» длины, \textendash --- en-dash
\DefineBibliographyExtras{russian}{%
  \protected\def\bibrangedash{%
    \cyrdash\penalty\value{abbrvpenalty}}% almost unbreakable dash
  \protected\def\bibdaterangesep{\bibrangedash}%тире для дат
}
\DefineBibliographyExtras{english}{%
  \protected\def\bibrangedash{%
    \cyrdash\penalty\value{abbrvpenalty}}% almost unbreakable dash
  \protected\def\bibdaterangesep{\bibrangedash}%тире для дат
}

%% Убрать "и др." из заголовка, сократить до единственной фамилии
\DeclareNameFormat{heading}{%
  \nameparts{#1}%
  \ifnumequal{\value{listcount}}{1}
    {\ifgiveninits
      {\usebibmacro{headingname:family-given}
        {\namepartfamily}
        {\namepartgiveni}
        {\namepartprefix}
        {\namepartsuffix}}
      {\usebibmacro{headingname:family-given}
        {\namepartfamily}
        {\namepartgiven}
        {\namepartprefix}
        {\namepartsuffix}}}
    {}}


\DeclareNameFormat{citea}{%
	\usebibmacro{name:family-given}
	{\namepartfamily}
	{\namepartgiveni}
	{\namepartprefix}
	{\namepartsuffix}}

\DeclareCiteCommand{\citea}
{\usebibmacro{prenote}}%
{\printnames[citea][1-1]{labelname}}
{\addcomma\addspace}
{\usebibmacro{postnote}} 

%% Повторять авторов из заголовка в области ответственности
\renewbibmacro*{byauthor}{%
	\setrespdelim\printnames[byauthor]{author}}

%% Ставить запятую после фамилии в заголовке, если стандарт 2006
\ifthenelse{\equal{\thegoststyle}{2006}}
{\renewcommand*{\revsdnamepunct}{\addcomma\addspace}}
{\renewcommand*{\revsdnamepunct}{\addspace}}


\DeclareNumChars*{)(S}

%% Всегда писть № в поле number
%\DeclareFieldFormat{number}{%
%  \bibsstring{number}\addabbrvspace#1}
  
%% Всегда писть issue в поле issue
%\DeclareFieldFormat{issue}{%
%  \bibsstring{issue}\addabbrvspace#1}

%% Использовать патент в варианте под заголовком
\xpatchbibdriver{patent}%
    {\usebibmacro{heading}%
     \newunit
     \usebibmacro{title}%
     \setunit*{\subtitlepunct}%
     \printfield{type}%
     \setunit*{\addspace}%
     \printfield{number}%
     \iflistundef{location}
       {}
       {\setunit*{\addspace}%
        \printlist[][-\value{listtotal}]{location}}%
     \setunit{\addcolondelim}%
     \printfield{ipc}}
    {\mkgostheading{
       \printfield{type}%
       \setunit*{\addspace}%
       \printfield{number}%
       \iflistundef{location}{}
         {\setunit*{\addspace}%
          \printlist[][-\value{listtotal}]{location}}%
       \setunit{\addcolondelim}%
       \printfield{ipc}}
   	 \newunit
     \usebibmacro{title}}
    {\typeout{patching patent for subtitle to heading succeded}}%
    {\typeout{patching patent for subtitle to heading failed}}
    
\renewcommand*{\subtitlepunct}{\addcolondelim}

% Использовать тире как разделитель в fullcite
%\renewcommand*{\newunitpunct}{\addperiod\addnbspace\textemdash\space\bibsentence}
%\renewcommand*{\newblockpunct}{}

% Отображать части в журналах
\xpatchbibmacro{jour:volume+parts+issuetitle}
	{\printfield{number}}%
	{\printfield{number}%
	 \setunit*{\addcomma\space}%
	 \printfield{part}}%
	{\typeout{patching article for parts succeded}}%
	{\typeout{patching article for parts failed}}
	
% Двоеточие после подзаголовка в диссертациях
\xpatchbibmacro{thesistitle}
{\printfield[titlecase]{subtitle}}%
{\printfield[titlecase]{subtitle}%
 \setunit*{\addcolondelim}}%
{\typeout{patching thesis for colondelim after subtitle succeded}}%
{\typeout{patching thesis for colondelim after subtitle failed}}
	
% Отображать организацию в дисертациях
\xpatchbibdriver{thesis}%
	{\printnames[family-given:full]{author}}%
	{\printnames[family-given:full]{author}\setunit{\resppunct}\printlist{organization}}%
	{\typeout{patching thesis for organization succeded}}%
	{\typeout{patching thesis for organization failed}}
	

%% Set low penalties for breaks at uppercase letters and lowercase letters
\setcounter{biburllcpenalty}{500} %управляет разрывами ссылок после маленьких букв RTFM biburllcpenalty
\setcounter{biburlucpenalty}{3000} %управляет разрывами ссылок после больших букв, RTFM biburlucpenalty

%Set higher penalty for breaking in number, dates and pages ranges
\setcounter{abbrvpenalty}{10000} % default is \hyphenpenalty which is 12

%Set higher penalty for breaking in names
\setcounter{highnamepenalty}{10000} % If you prefer the traditional BibTeX behavior (no linebreaks at highnamepenalty breakpoints), set it to ‘infinite’ (10 000 or higher).
\setcounter{lownamepenalty}{10000}

%%% Set low penalties for breaks at uppercase letters and lowercase letters
\setcounter{biburllcpenalty}{500} %управляет разрывами ссылок после маленьких букв RTFM biburllcpenalty
\setcounter{biburlucpenalty}{3000} %управляет разрывами ссылок после больших букв, RTFM biburlucpenalty

%%% Список литературы с красной строки (без висячего отступа) %%%
\defbibenvironment{bibliography} % переопределяем окружение библиографии из gost-numeric.bbx пакета biblatex-gost
  {\list
     {\printtext[labelnumberwidth]{%
	\printfield{prefixnumber}%
	\printfield{labelnumber}}}
     {%
      \setlength{\labelwidth}{\labelnumberwidth}%
      \setlength{\leftmargin}{0pt}% default is \labelwidth
      \setlength{\labelsep}{\widthof{\ }}% Управляет длиной отступа после точки % default is \biblabelsep
      \setlength{\itemsep}{\bibitemsep}% Управление дополнительным вертикальным разрывом между записями. \bibitemsep по умолчанию соответствует \itemsep списков в документе.
      \setlength{\itemindent}{\parindent}% Пользуемся тем, что \bibhang по умолчанию принимает значение \parindent (абзацного отступа), который переназначен в styles.tex
      \addtolength{\itemindent}{\labelwidth}% Сдвигаем правее на величину номера с точкой
      \addtolength{\itemindent}{\labelsep}% Сдвигаем ещё правее на отступ после точки
      \setlength{\parsep}{\bibparsep}%
     }%
      \renewcommand*{\makelabel}[1]{\hss##1}%
  }
  {\endlist}
  {\item}

%% Счётчик использованных ссылок на литературу, обрабатывающий с учётом неоднократных ссылок
%http://tex.stackexchange.com/a/66851/79756
%\newcounter{citenum}
\newtotcounter{citenum}
\makeatletter
\defbibenvironment{counter} %Env of bibliography
  {\setcounter{citenum}{0}%
  \renewcommand{\blx@driver}[1]{}%
  } %what is doing at the beginining of bibliography. In your case it's : a. Reset counter b. Say to print nothing when a entry is tested.
  {} %Здесь то, что будет выводиться командой \printbibliography. \thecitenum сюда писать не надо
  {\stepcounter{citenum}} %What is printing / executed at each entry.
\makeatother
\defbibheading{counter}{}



\newtotcounter{citeauthorvak}
\makeatletter
\defbibenvironment{countauthorvak} %Env of bibliography
{\setcounter{citeauthorvak}{0}%
    \renewcommand{\blx@driver}[1]{}%
} %what is doing at the beginining of bibliography. In your case it's : a. Reset counter b. Say to print nothing when a entry is tested.
{} %Здесь то, что будет выводиться командой \printbibliography. Обойдёмся без \theciteauthorvak в нашей реализации
{\stepcounter{citeauthorvak}} %What is printing / executed at each entry.
\makeatother
\defbibheading{countauthorvak}{}

\newtotcounter{citeauthornotvak}
\makeatletter
\defbibenvironment{countauthornotvak} %Env of bibliography
{\setcounter{citeauthornotvak}{0}%
    \renewcommand{\blx@driver}[1]{}%
} %what is doing at the beginining of bibliography. In your case it's : a. Reset counter b. Say to print nothing when a entry is tested.
{} %Здесь то, что будет выводиться командой \printbibliography. Обойдёмся без \theciteauthornotvak в нашей реализации
{\stepcounter{citeauthornotvak}} %What is printing / executed at each entry.
\makeatother
\defbibheading{countauthornotvak}{}

\newtotcounter{citeauthorindexed}
\makeatletter
\defbibenvironment{countauthorindexed} %Env of bibliography
{\setcounter{citeauthorindexed}{0}%
	\renewcommand{\blx@driver}[1]{}%
} %what is doing at the beginining of bibliography. In your case it's : a. Reset counter b. Say to print nothing when a entry is tested.
{} %Здесь то, что будет выводиться командой \printbibliography. Обойдёмся без \theciteauthorindexed в нашей реализации
{\stepcounter{citeauthorindexed}} %What is printing / executed at each entry.
\makeatother
\defbibheading{countauthorindexed}{}

\newtotcounter{citeauthorpaper}
\makeatletter
\defbibenvironment{countauthorpaper} %Env of bibliography
{\setcounter{citeauthorpaper}{0}%
	\renewcommand{\blx@driver}[1]{}%
} %what is doing at the beginining of bibliography. In your case it's : a. Reset counter b. Say to print nothing when a entry is tested.
{} %Здесь то, что будет выводиться командой \printbibliography. Обойдёмся без \theciteauthorpaper в нашей реализации
{\stepcounter{citeauthorpaper}} %What is printing / executed at each entry.
\makeatother
\defbibheading{countauthorpaper}{}

\DeclareBibliographyCategory{biblioauthoreng}
\newtotcounter{citeauthoreng}
\makeatletter
\defbibenvironment{countauthoreng} %Env of bibliography
{\setcounter{citeauthoreng}{0}%
	\renewcommand{\blx@driver}[1]{}%
} %what is doing at the beginining of bibliography. In your case it's : a. Reset counter b. Say to print nothing when a entry is tested.
{} %Здесь то, что будет выводиться командой \printbibliography. Обойдёмся без \theciteauthoreng в нашей реализации
{\stepcounter{citeauthoreng}} %What is printing / executed at each entry.
\makeatother
\defbibheading{countauthoreng}{}

\newtotcounter{citeauthorconf}
\makeatletter
\defbibenvironment{countauthorconf} %Env of bibliography
{\setcounter{citeauthorconf}{0}%
    \renewcommand{\blx@driver}[1]{}%
} %what is doing at the beginining of bibliography. In your case it's : a. Reset counter b. Say to print nothing when a entry is tested.
{} %Здесь то, что будет выводиться командой \printbibliography. Обойдёмся без \theciteauthorconf в нашей реализации
{\stepcounter{citeauthorconf}} %What is printing / executed at each entry.
\makeatother
\defbibheading{countauthorconf}{}

\newtotcounter{citeauthor}
\makeatletter
\defbibenvironment{countauthor} %Env of bibliography
{\setcounter{citeauthor}{0}%
    \renewcommand{\blx@driver}[1]{}%
} %what is doing at the beginining of bibliography. In your case it's : a. Reset counter b. Say to print nothing when a entry is tested.
{} %Здесь то, что будет выводиться командой \printbibliography. Обойдёмся без \theciteauthor в нашей реализации
{\stepcounter{citeauthor}} %What is printing / executed at each entry.
\makeatother
\defbibheading{countauthor}{}

\defbibheading{authorpublications}[\authorbibtitle]{\section*{#1}}
\defbibheading{pubsubgroup}{\needspace{3em}\centering\textit{#1}}
\defbibheading{otherpublications}{\section*{#1}}


%%% Создание команд для вывода списка литературы %%%
\newcommand*{\insertbibliofull}[1][]{
\printbibliography[keyword=bibliofull,section=0,title=\ifstrempty{#1}{\fullbibtitle}{#1}]
\printbibliography[heading=counter,env=counter,keyword=bibliofull,section=0]
}

\newcommand*{\insertbiblioauthorcited}{
\printbibliography[heading=authorpublications,keyword=biblioauthor,section=0,title=\authorbibtitle]
}
\newcommand*{\insertbiblioauthor}{
\printbibliography[heading=authorpublications,section=1,title=\authorbibtitle]
}
\newcommand*{\insertbiblioauthorimportant}{
\printbibliography[heading=authorpublications,keyword=biblioauthor,section=2,title={Наиболее значимые \MakeLowercase{\protect\authorbibtitle{}}}]
}
\newcommand*{\insertbiblioauthorgrouped}{% Заготовка для вывода сгруппированных печатных работ автора. Порядок нумерации определяется в соответствующих счетчиках внутри окружения refsection в файле common/characteristic.tex
\section*{\authorbibtitle}
\printbibliography[heading=pubsubgroup, keyword=biblioauthorvak, notcategory=biblioauthoreng, section=1,title=\vakbibtitle]%
\printbibliography[heading=pubsubgroup, keyword=biblioauthorvak, category=biblioauthoreng, section=1,title=\vakindexedbibtitle]%
\printbibliography[heading=pubsubgroup, notkeyword=biblioauthorvak, category=biblioauthoreng, section=1,title=\engbibtitle]%
\printbibliography[heading=pubsubgroup, keyword=biblioauthornotvak, notcategory=biblioauthoreng, section=1,title=\notvakbibtitle]%
\printbibliography[heading=pubsubgroup, keyword=biblioauthorconf, section=1,title=\confbibtitle]%
}

\newcommand*{\insertbiblioother}{
\printbibliography[heading=otherpublications,keyword=biblioother]
}

\makeatletter
\newcommand{\EnableTranslatedBibCategory}{
	\WarningFilter{biblatex}{The following entry could not be found}
	\WarningFilter{biblatex}{citation 'eng_}
	\DeclareBibliographyCategory{orig}
	\DeclareBibliographyCategory{eng}
	\AtEveryCitekey{%
		\addtocategory{orig}{\thefield{entrykey}}%
		\nocite{eng_\thefield{entrykey}}%
		\blx@ifdata{eng_\thefield{entrykey}}%
		{\addtocategory{eng}{eng_\thefield{entrykey}}}%
		{\addtocategory{eng}{\thefield{entrykey}}}}}
\makeatother

% % %
\def\printdateextralabel{\printlabeldateextra}

\newtoggle{bbx:apastyle}
\newtoggle{bbx:related}
\global\settoggle{bbx:apastyle}{false}

\long\def\renewapa#1#2{%
	\csletcs{origapa@#1}{#1}%
	\long\csdef{#1}{\iftoggle{bbx:apastyle}{#2}{\csuse{origapa@#1}}}%
}
\long\def\renewapaa#1#2{%
	\csletcs{origapa@#1}{#1}%
	\csdef{#1}##1{\iftoggle{bbx:apastyle}{#2}{\csuse{origapa@#1}}}%
}
\long\def\renewapao#1#2#3{%
	\ifcsundef{#1}%
		{\csdef{#1}##1{\iftoggle{bbx:apastyle}{#3}{\csuse{#2}}}}%
		{\csletcs{origapa@#1}{#1}%
		 \csdef{#1}##1{\iftoggle{bbx:apastyle}{#3}{\csuse{origapa@#1}}}}%
}

\urlstyle{rm} % APA examples all have URLs in same font as text

%%%%%%%%%%%%%%%%%%%%%%%%%%%%%%%%%%%%%%%%%%%
%  Used to join citations/references to
%  extrayear

\newcommand{\apashortdash}{-}

%
%%%%%%%%%%%%%%%%%%%%%%%%%%%%%%%%%%%%%%%%%%%


%%%%%%%%%%%%%%%%%%%%%%%%%%%%%%%%%%%%%%%%%%%
%  Allow variable max authors/editors limit

\newcommand{\maxprtauth}{7}
\DeclareBibliographyOption{apamaxprtauth}{%
  \ifstrempty{#1}
    {}
    {\renewcommand{\maxprtauth}{#1}}}

%
%%%%%%%%%%%%%%%%%%%%%%%%%%%%%%%%%%%%%%%%%%%

%%%%%%%%%%%%%%%%%%%%%%%%%%%%%%%%%%%%%%%%%%%%%
% (APA 6.22) Force roman numerals into arabic
%            using etoolbox macros

\newcommand{\apanum}[1]{\ifrmnum{#1}{\rmntonum{#1}}{#1}}

%
%%%%%%%%%%%%%%%%%%%%%%%%%%%%%%%%%%%%%%%%%%%%%

%%%%%%%%%%%%%%%%%%%%%%%%%%%%%%%%%%%%%%%%%%%%%%%%%%%%%%%%%%%%%%%%%%%%%%%%%%
% Apa defines no particular hanging indent but this makes it look like the
% examples in the style manual.

\setlength{\bibhang}{2.5em}

%
%%%%%%%%%%%%%%%%%%%%%%%%%%%%%%%%%%%%%%%%%%%%%%%%%%%%%%%%%%%%%%%%%%%%%%%%%%

%%%%%%%%%%%%%%%%%%%%%%%%%%%%%%%%%%%%%%%
% (APA 4.16 Example 29) Some DSM macros

\gdef\DSMIII{\emph{DSM-III}}
\gdef\DSMIIIR{\emph{DSM-III-R}}
\gdef\DSMIV{\emph{DSM-IV}}
\gdef\DSMIVTR{\emph{DSM-IV-TR}}
\gdef\PsycSCAN{\emph{PsycSCAN}}
\gdef\PsycARTICLES{\emph{PsycARTICLES}}

%
%%%%%%%%%%%%%%%%%%%%%%%%%%%%%%%%%%%%%%%

%%%%%%%%%%%%%%%%%%%%%%%%%%%%%%%%
%

% Date formats. Suppress end range of less specific fields
\newcommand*{\mkdaterangeapalong}[1]{%
	%\blx@metadateinfo{#1}%
	\iffieldundef{#1year}{}
	{\datecircaprint
		\ifstrequal{#1}{url}% URL dates are unlikely to be BCE ...
		{\printtext{%
				\mkbibdateapalongmdy{#1year}{#1month}{#1day}%
				\iffieldundef{#1endyear}%
				{}%
				{\iffieldequalstr{#1endyear}{}% open-ended range?
					{\mbox{\bibdatedash}}
					{\bibdatedash%
						\iffieldsequal{#1year}{#1endyear}%
						{\iffieldsequal{#1month}{#1endmonth}%
							{\iffieldsequal{#1day}{#1endday}%
								{}%
								{\mkbibdateapalongmdy{}{}{#1endday}}}%
							{\mkbibdateapalongmdy{}{#1endmonth}{#1endday}}}%
						{\mkbibdateapalongmdy{#1endyear}{#1endmonth}{#1endday}}}}%
				\dateuncertainprint}}
		{\printtext{%
				\mkbibdateapalong{#1year}{#1month}{#1day}%
				\dateeraprint{#1year}%
				\iffieldundef{#1endyear}%
				{}%
				{\iffieldequalstr{#1endyear}{}% open-ended range?
					{\mbox{\bibdatedash}}
					{\bibdatedash%
						\iffieldsequal{#1year}{#1endyear}%
						{\iffieldsequal{#1month}{#1endmonth}%
							{\iffieldsequal{#1day}{#1endday}%
								{}%
								{\mkbibdateapalong{}{}{#1endday}}}%
							{\mkbibdateapalong{}{#1endmonth}{#1endday}}}%
						{\mkbibdateapalong{#1endyear}{#1endmonth}{#1endday}%
							\dateeraprint{#1endyear}}}%
					\enddateuncertainprint}}}}}

% Only for DATE as only \printdateextra is used
\newcommand*{\mkdaterangeapalongextra}[1]{%
	%\blx@metadateinfo{#1}%
	\iffieldundef{#1year}{}
	{\printtext{%
			\datecircaprint
			\mkbibdateapalongextra{#1year}{#1month}{#1day}%
			\dateeraprint{#1year}%
			\dateuncertainprint
			\iffieldundef{#1endyear}%
			{}%
			{\iffieldequalstr{#1endyear}{}% open-ended range?
				{\mbox{\bibdatedash}}
				{\bibdatedash%
					\iffieldsequal{#1year}{#1endyear}%
					{\iffieldsequal{#1month}{#1endmonth}%
						{\iffieldsequal{#1day}{#1endday}%
							{}%
							{\mkbibdateapalongextra{}{}{#1endday}}}
						{\mkbibdateapalongextra{}{#1endmonth}{#1endday}}}
					{\mkbibdateapalongextra{#1endyear}{#1endmonth}{#1endday}%
						\dateeraprint{#1endyear}}}%
				\enddateuncertainprint}}}}


	\renewcommand*{\datecircaprint}{%
		\ifdatecirca{\biblcstring{circa}\printdelim{datecircadelim}}{}}


%
%%%%%%%%%%%%%%%%%%%%%%%%%%%%%%%%

%%%%%%%%%%%%%%%%%%%%%%%%%%%%%%%%%
% (APA 7.09 Example 62) ERIC
% eprint references

\DeclareFieldFormat{eprint:eric}{%
  \printtext{\bibcpstring{retrieved}}%
  \setunit{\addspace}%
  \printtext{\bibstring{from}}\addspace%
  ERIC\addspace database\adddot\addspace%
  \mkbibparens{#1}}

%
%%%%%%%%%%%%%%%%%%%%%%%%%%%%%%%%%

%%%%%%%%%%%%%%%%%%%%%%%%%%%%%%%%%%%%%%%%%%%%%%%
% (APA 6.12) Five author max before "et al" and a one author truncation policy
%            However, only after the first cite, see the labelname format
%            in .cbx. It's also overridden per-entry by uniquelist
% (APA 6.x)  citation tracking is global
% (APA 6.14) uses initials to disambiguate shared surnames
% (APA 6.16) uses year postfix to disambiguate multiple items in same year
% (APA 6.10) Never reference anything not cited
% (APA 6.25) author initials only
% (APA 6.25) sorting is nyt but we need to account for PUBSTATE which comes
%            after all normal cites for the same author

%\DeclareSortingScheme{apa}{
%  \sort{
%    \field{presort}
%  }
%  \sort[final]{
%    \field{sortkey}
%  }
%  \sort{
%    \field{sortname}
%    \field{author}
%    \field{editor}
%    \field{translator}
%    \field{writer}
%    \field{director}
%    \field{producer}
%    \field{execproducer}
%    \field{origauthor}
%    \field{sorttitle}
%    \field{title}
%  }
%  \sort{
%    \field{sortyear}
%    \field{year}
%    \field{pubstate}
%  }
%  \sort{
%    \field{sorttitle}
%    \field{title}
%  }
%  \sort{
%    \field[padside=left,padwidth=4,padchar=0]{volume}
%    \literal{0000}
%  }
%}

% Don't fall back on other year fields if there is no year, use
% "nodate" string
\DeclareLabeldate{%
  \field{pubstate}
  \field{date}
  \field{year}%legacy - no EDTF support     
  \literal{nodate}
}

% Due to APA strange requirements like truncation after first cite and
% ellipsis from 7th to n-1 in bib, there might be some really strange
% edge cases which can't be handled as this needs treating in the style
% after biber has finished. Very unlikely though.
\ExecuteBibliographyOptions{
	date=apalong,%
	dateabbrev=false,%
	datecirca=true,%
	dateera=christian,%
	labeldate=apalong,%
	labeldateparts=true,%
	dateuncertain=true,%       
}

%
%%%%%%%%%%%%%%%%%%%%%%%%%%%%%%%%%%%%%%%%%%%%%%%


%% Enforce ignoring of PUBSTATE if there is a YEAR or DATE field
%\DeclareStyleSourcemap{
%  \maps[datatype=bibtex]{
%    \map{
%      \step[fieldsource=year, final]
%      \step[fieldset=pubstate, null]
%    }
%    \map{
%      \step[fieldsource=date, final]
%      \step[fieldset=pubstate, null]
%   }
%   \map{% перекидываем значения полей language в поля langid, которыми пользуется biblatex
%       \step[fieldsource=year, fieldset=origyear, origfieldval, final]
%   }
%  }
%}

%%%%%%%%%%%%%%%%%%%%%%%%%%%%%%%%%%%%%%%%%%%%%%%%%%%%
% It is not APA standard to have backrefs in the bib
% Some users might like it though.

\newbool{apa:backref}
\DeclareBibliographyOption{apabackref}{%
  \ifstrequal{#1}{true}
    {\global\booltrue{apa:backref}}
    {\global\boolfalse{apa:backref}}}

%
%%%%%%%%%%%%%%%%%%%%%%%%%%%%%%%%%%%%%%%%%%%%%%%%%%%%%

%%%%%%%%%%%%%%%%%%%%%%%%%%%%%%%%%%%%%%%%%%%%%%%%%%%%%%%%%
% (APA 6.29) Additional material sometimes goes in parens
%            after title. This bool tracks the parens.

\newbool{bbx:parens}
\AtEveryBibitem{\global\boolfalse{bbx:parens}}

%
%%%%%%%%%%%%%%%%%%%%%%%%%%%%%%%%%%%%%%%%%%%%%%%%%%%%%%%%%

%%%%%%%%%%%%%%%%%%%%%%%%%%%%%%%%%%%%%%%%%%%%%%%%%%%%%%%%%
% (APA 7.06:47) Reviews are awkward - if no author, date
%               position changes so we need a flag to
%               track this

\newbool{bbx:noreviewauthor}
\AtEveryBibitem{\global\boolfalse{bbx:noreviewauthor}}

%
%%%%%%%%%%%%%%%%%%%%%%%%%%%%%%%%%%%%%%%%%%%%%%%%%%%%%%%%%

%%%%%%%%%%%%%%%%%%%%%%%%%%%%%%%%%%%%%%%%%%%%%%%%%%%%%%
% (APA 6.30) Sometimes "Vol" is inside the additional
%            material parens, sometimes not. This bool
%            tracks if it has been inserted yet.     
%            Can't use \clearfield{volume} as some
%            later number format tests need to know
%            whether volume was defined.

\newbool{bbx:volseen}
\AtEveryBibitem{\global\boolfalse{bbx:volseen}}

%
%%%%%%%%%%%%%%%%%%%%%%%%%%%%%%%%%%%%%%%%%%%%%%%%%%%%%%

%%%%%%%%%%%%%%%%%%%%%%%%%%%%%%%%%%%%%%%%%%%%%%%%%%
% (APA 6.27) This bool tracks if the title was put
%            in the author position.
%            Can't use \clearfield{title} as some
%            later punctuation tests need to know
%            whether title was defined.

\newbool{bbx:titleinauthpos}
\AtEveryBibitem{\global\boolfalse{bbx:titleinauthpos}}

%
%%%%%%%%%%%%%%%%%%%%%%%%%%%%%%%%%%%%%%%%%%%%%%%%%%

%%%%%%%%%%%%%%%%%%%%%%%%%%%%%%%%%%%%%%%%%%%%%%%
% (APA 6.27) Need a flag to say when the editor
%            is in author position because this
%            can effect where the year goes.

\newbool{bbx:editorinauthpos}
\AtEveryBibitem{\global\boolfalse{bbx:editorinauthpos}}

%
%%%%%%%%%%%%%%%%%%%%%%%%%%%%%%%%%%%%%%%%%%%%%%%

%%%%%%%%%%%%%%%%%%%%%%%%%%%%%%%%%%%%%%%%%%%%%
% (APA 6.27) Flag to say whether the "in" has
%            been placed already in IN*
%            Reset every bibitem.

\newbool{bbx:in}
\AtEveryBibitem{\global\boolfalse{bbx:in}}

%
%%%%%%%%%%%%%%%%%%%%%%%%%%%%%%%%%%%%%%%%%%%%%

%%%%%%%%%%%%%%%%%%%%%%%%%%%%%%%%%%%%%%
% Set up some standard APA formats

%\DeclareFieldFormat{volume}{\apanum{#1}}
%\DeclareFieldFormat{series}{#1}
%\DeclareFieldFormat{chapter}{\bibcpstring{chapter}~\apanum{#1}}
%\DeclareFieldFormat{volumes}{\bibcpstring{volumes}~#1}
%\DeclareFieldFormat{addendum}{\mkbibparens{#1}}
%\DeclareFieldFormat{part}{#1}

%
%%%%%%%%%%%%%%%%%%%%%%%%%%%%%%%%%%%%%%

%%%%%%%%%%%%%%%%%%%%%%%%%%%%%%%%%%%%%%%%%%%%%%%%%%%%%%%%%%
% (APA 6.25) Works by the same author have the full author
%            name, not an eliding dash. Author is always
%            last name first.
% (APA 6.27) Ellipsis in 7th position and then nothing until last
% (APA 7.10 Example 67) Unknown names and dates
%

% \printnames does different things depending on whether the format you are
% calling is the default for the name field or is a custom format:
%
%   * Default format (e.g. "labelname" format for printing labelname): no
%     start/stop max/minnames truncation is done first - the format is
%     supposed to do it.
%   * Custom format (e.g. "labelname" format for printing author):
%     max/minnames truncation is done before calling the format so we have
%     to do \printnames[format][-\value{listtotal}]{field} to ensure we get
%     all of the names in the list to work on

%\DeclareNameAlias{default}{apaauthor}

\DeclareNameFormat{apaauthor}{%
  \ifthenelse{\value{listcount}=\maxprtauth\AND\value{listcount}<\value{listtotal}}
    {\addcomma\addspace\ldots\addspace}
    {\ifthenelse{\value{listcount}>\maxprtauth\AND\value{listcount}<\value{listtotal}}
      {}
      {\iffieldequalstr{doubtfulauthor}{true}
        {\mkbibbrackets{\usebibmacro{name:apa:family-given}%
                         {\namepartfamily}%
                         {\namepartgiven}%
                         {\namepartgiveni}%
                         {\namepartprefix}%
                         {\namepartsuffix}?}}
        {\usebibmacro{name:apa:family-given}%
          {\namepartfamily}%
          {\namepartgiven}%
          {\namepartgiveni}%
          {\namepartprefix}%
          {\namepartsuffix}}}}%
    \ifthenelse{\value{listcount}=\value{listtotal}}%
      {\ifmorenames{\andothersdelim\bibstring{andothers}}{}}{}}

\newbibmacro*{apa__author}{%
  \ifnameundef{author}
    {\usebibmacro{labeltitle}}
    {\printnames[apaauthor][-\value{listtotal}]{author}%
     \setunit*{\addspace}%
     \printfield{nameaddon}%
     \ifnameundef{with}
       {}
       {\setunit{}\addspace\mkbibparens{\printtext{\bibstring{with}\addspace}%
        \printnames[apaauthor][-\value{listtotal}]{with}}
        \setunit*{\addspace}}}%
  \newunit\newblock%
  \usebibmacro{labelyear+extrayear}}

%
%%%%%%%%%%%%%%%%%%%%%%%%%%%%%%%%%%%%%%%%%%%%%%%%%%%%%%%%%%

%%%%%%%%%%%%%%%%%%%%%%%%%%%%%%%%%%%%%%%%%%%%%%%%%%%%%%%%%%%%%%%%%%%%%%
% (APA 6.27) References section delimiters are ampersands, not " and "
%            Needs to be in this hook otherwise it sets this for all
%            citations too.
% (APA 6.27) Use blank for long lists
% (APA 4.03) Serial comma for lists of three or more

\renewapa{finalnamedelim}{%
    \ifthenelse{\value{listcount}>\maxprtauth}
      {}
      {\ifthenelse{\value{liststop}>2}
         {\finalandcomma\addspace\&\space}
         {\addspace\&\space}}}

%
%%%%%%%%%%%%%%%%%%%%%%%%%%%%%%%%%%%%%%%%%%%%%%%%%%%%%%%%%%%%%%%%%%%%%%

%%%%%%%%%%%%%%%%%%%%%%%%%%%%%%%%%%%%%%%%%%%%%%%%%%%%%%%%%%%%%%%%%%%%%%%%%%%%%
% (APA 6.28) Use "nodate" bibstring for references with no year
%            Months and days go into parenthesis with year, if set
% (APA 6.29) Issue goes where month normally goes if it
%            is set.

% Make sure endmonth gets an expansion too
\DeclareFieldFormat{endmonth}{\mkbibmonth{#1}}

\newbibmacro*{monthorissue}{%
  \iffieldundef{issue}
    {\iffieldundef{month}
      {}
      {\addcomma\space\printfield{month}%
      \iffieldundef{endmonth}{}{/\printfield{endmonth}}}}
    {\addcomma\space\printfield{issue}}}

\newbibmacro*{labelyear+extrayear}{%
  \iffieldundef{labelyear}
    {\iffieldundef{origyear}
      {}
      {\printtext[apadate[{\printorigdate}}}%
  {\printtext[apadate]{\printlabeldateextra}}}

\DeclareFieldFormat{apadate}{%
\ifboolexpr{ test {\ifdatecirca} or test {\ifdateuncertain} }
{\printtext{\mkbibbrackets{#1}}}
{\printtext{\mkbibparens{#1}}}}  
%
%%%%%%%%%%%%%%%%%%%%%%%%%%%%%%%%%%%%%%%%%%%%%%%%%%%%%%%%%%%%%%%%%%%%%%%%%%%%%

%%%%%%%%%%%%%%%%%%%%%%%%%%%%%%%%%%%%%%%%%%%%%%%%%%%%%%%%%%%%%%%%%%
% (APA 6.29) General format of titles.
%            Ugly Chicago-style lowercasing of English titles

\DeclareFieldFormat{apacase}{\MakeSentenceCase*{#1}}



%\renewapaa{abx@ffd@*@issuetitle}{#1\isdot}
%\renewapaa{abx@ffd@*@title}{\iffieldequalstr{titleisdescription}{true}{\mkbibbrackets{#1}}{\mkbibemph{#1}\isdot}}
%\renewapaa{abx@ffd@*@origtitle}{\mkbibemph{\MakeSentenceCase*{#1}}\isdot}
%\renewapaa{abx@ffd@article@title}{#1\isdot}
%\renewapaa{abx@ffd@article@origtitle}{\MakeSentenceCase*{#1}\isdot}
%\renewapaa{abx@ffd@inbook@title}{#1\isdot}
%\renewapaa{abx@ffd@inbook@origtitle}{\MakeSentenceCase*{#1}\isdot}
%\renewapaa{abx@ffd@incollection@title}{#1\isdot}
%\renewapaa{abx@ffd@incollection@origtitle}{\MakeSentenceCase*{#1}\isdot}
%\renewapaa{abx@ffd@inproceedings@title}{#1\isdot}
%\renewapaa{abx@ffd@inproceedingsicle@origtitle}{\MakeSentenceCase*{#1}\isdot}
%\renewapaa{abx@ffd@proceedings@title}{#1\isdot}
%\renewapaa{abx@ffd@proceedings@origtitle}{\MakeSentenceCase*{#1}\isdot}

%\DeclareFieldFormat{issuetitle}{#1\isdot}
%\DeclareFieldFormat{title}{\iffieldequalstr{titleisdescription}{true}{\mkbibbrackets{#1}}{\mkbibemph{#1}\isdot}}
%\DeclareFieldFormat{origtitle}{\mkbibemph{\MakeSentenceCase*{#1}}\isdot}
%\DeclareFieldFormat[article]{title}{#1\isdot}
%\DeclareFieldFormat[article]{origtitle}{\MakeSentenceCase*{#1}\isdot}
%\DeclareFieldFormat[inbook]{title}{#1\isdot}
%\DeclareFieldFormat[inbook]{origtitle}{\MakeSentenceCase*{#1}\isdot}
%\DeclareFieldFormat[incollection]{title}{#1\isdot}
%\DeclareFieldFormat[incollection]{origtitle}{\MakeSentenceCase*{#1}\isdot}
%\DeclareFieldFormat[inproceedings]{title}{#1\isdot}
%\DeclareFieldFormat[inproceedings]{origtitle}{\MakeSentenceCase*{#1}\isdot}
%\DeclareFieldFormat[proceedings]{title}{#1\isdot}
%\DeclareFieldFormat[proceedings]{origtitle}{\MakeSentenceCase*{#1}\isdot}

%
%%%%%%%%%%%%%%%%%%%%%%%%%%%%%%%%%%%%%%%%%%%%%%%%%%%%%%%%%%%%%%%%%%

%%%%%%%%%%%%%%%%%%%%%%%%%%%%%%%%%%%%%%%%%%%%%%%%%%%%%%%
% (APA 7.01 Example 10) Newspapers have prefix for pages.

\DeclareFieldFormat[newsarticle]{pages}{%
    \DeclareNumChars{.abcdefghijklmnopqrstuvwxyzABCDEFGHIJKLMNOPQRSTUVWXYZ}%
    \mkpageprefix{#1}%
    \DeclareNumChars{.}}

\DeclareFieldFormat[newsarticle]{newspaper}{\mkbibemph{#1}}
\DeclareFieldFormat[newsarticle]{entrysubtype}{\mkbibbrackets{#1}}
\DeclareFieldFormat[newsarticle]{title}{#1}
\DeclareFieldFormat[article]{pages}{\iftoggle{bbx:apastyle}{#1}{\mkpageprefix[bookpagination]{#1}}}
%\DeclareFieldFormat[article]{pages}{#1}
%\renewapao(abx@ffd@article@pages){abx@ffd@*@pages}{#1}
%\ifcsundef{abx@ffd@article@pages}%
%	{\csdef{abx@ffd@article@pages}#1{\iftoggle{bbx:apastyle}{#1}{\mkpageprefix[bookpagination]{#1}}}}%
%	{\csletcs{origapa@abx@ffd@article@pages}{abx@ffd@article@pages}%
% 	 \csdef{abx@ffd@article@pages}#1{\iftoggle{bbx:apastyle}{#1}{\csuse{origapa@abx@ffd@article@pages}}}}%

\newbibmacro*{newspaper}{%
  \printfield{newspaper}
  \setunit{\addcomma\space}}

%
%%%%%%%%%%%%%%%%%%%%%%%%%%%%%%%%%%%%%%%%%%%%%%%%%%%%%%%

%%%%%%%%%%%%%%%%%%%%%%%%%%%%%%%%%%%%%%%%%%%%%%%%%%%%%%%%%%%%%%%%%%%%%%%%
% (APA 6.29)      Additional {main}title information in brackets (using)
%                 {MAIN}TITLEADDON field. Colon after maintitle.

\DeclareFieldFormat{titleaddon}{\mkbibbrackets{\MakeSentenceCase*{#1}}}
\DeclareFieldFormat{maintitleaddon}{\mkbibbrackets{\MakeSentenceCase*{#1}}}
\DeclareFieldFormat{booktitleaddon}{\mkbibbrackets{\MakeSentenceCase*{#1}}}

\newbibmacro*{bookaddinfo}{%
  \ifthenelse{\iffieldundef{edition}\AND\iffieldundef{volumes}}
    {}
    {\printtext{\bibopenparen}%
     \printfield{edition}%
     \setunit*{\addcomma\addspace}%
     \printfield{volumes}%
     \setunit{}%
     \printtext{\bibcloseparen}}%
}

\newbibmacro*{apa__title}{%
  \ifthenelse{\iffieldundef{title}\AND\iffieldundef{subtitle}}
    {}
    {\iffieldundef{origtitle}
      {\printtext[title]{%
        \printfield[apacase]{title}%
        \setunit{\subtitlepunct}%
        \printfield[apacase]{subtitle}}}
      {\printfield{origtitle}%
       \setunit{\addspace}%
       \printtext[brackets]{%
        \printfield[apacase]{title}%
        \setunit{\subtitlepunct}%
        \printfield[apacase]{subtitle}}}%
     \setunit{\addspace}%
     \printfield{titleaddon}%
     \iffieldequalstr{entrytype}{book}%
       {\setunit{\addspace}\usebibmacro{bookaddinfo}}%
       {}%
     \ifthenelse{%
       \ifnameundef{author}\AND%
       \(\ifnameundef{editor}\AND\NOT\boolean{bbx:editorinauthpos}\)\AND%
       \ifnameundef{producer}\AND%
       \ifnameundef{director}\AND%
       \ifnameundef{writer}}
        {\newunit\newblock
         \usebibmacro{labelyear+extrayear}}
        {}}}

\newbibmacro*{apa__maintitle}{%
  \ifthenelse{\iffieldundef{maintitle}\AND\iffieldundef{mainsubtitle}}
    {}
    {\printtext[maintitle]{%
       \printfield[apacase]{maintitle}%
       \setunit{\subtitlepunct}%
       \printfield[apacase]{mainsubtitle}}%
    \setunit{\addspace}}
  \printfield{maintitleaddon}%
  \addcolon}

%
%%%%%%%%%%%%%%%%%%%%%%%%%%%%%%%%%%%%%%%%%%%%%%%%%%%%%%%%%%%%%%%%%%%%%%%%

%%%%%%%%%%%%%%%%%%%%%%%%%%%%%%%%%%%%%%%%%%%%%%%%%%%%%
% (APA 6.30) Format of volume and number for articles

%\renewapao{abx@ffd@article@volume}{abx@ffd@*@volume}{\mkbibemph{\apanum{#1}}}
%\renewapao{abx@ffd@article@number}{abx@ffd@*@number}{\mkbibparens{\apanum{#1}}}

\DeclareFieldFormat[article]{journaltitle}{\iftoggle{bbx:apastyle}{\mkbibemph{#1}}{#1}}
\DeclareFieldFormat[article]{volume}{\iftoggle{bbx:apastyle}{\mkbibemph{\apanum{#1}}}{%
  \ifbibstring{volume}
{\bibstring{jourvol}\addabbrvspace#1}
{#1}}}
\DeclareFieldFormat[article]{number}{\iftoggle{bbx:apastyle}{\mkbibparens{\apanum{#1}}}{%
  \iffieldnums{number}
{\ifbibstring{number}
	{\bibsstring{number}\addabbrvspace#1}
	{\unspace\adddot#1}}%
{\ifcapital{\MakeCapital{#1}}{#1}\isdot}}}
\DeclareFieldFormat[article]{issue}{\iftoggle{bbx:apastyle}{\mkbibparens{\apanum{#1}}}{%
  \iffieldnums{issue}
{\ifbibstring{issue}
	{\bibstring{issue}\addabbrvspace#1}
	{#1}}
{\ifcapital{\MakeCapital{#1}}{#1}\isdot}}}


%
%%%%%%%%%%%%%%%%%%%%%%%%%%%%%%%%%%%%%%%%%%%%%%%%%%%%%

%%%%%%%%%%%%%%%%%%%%%%%%%%%%%%%%%%%%%%%%%%%%%%%%%%%%%%%%%
% (APA 6.30) Commas between title and volume for articles

\newbibmacro*{journal+issuetitle}{%
  \usebibmacro{journal}%
  \setunit*{\addcomma\addspace}%
  \iffieldundef{series}
    {}
    {\newunit
     \printfield{series}
     \setunit{\addcomma\space}}%
  \printfield{volume}%
  \printfield{number}%
  \printfield{issue}%
  \setunit{\addspace}\newblock
  \usebibmacro{issuename}%
  \newunit}

\newbibmacro*{issuename}{%
  \iffieldundef{issuetitle}
    {}
    {\printtext[issuetitle]{%
       \printfield[noformat]{issuetitle}%
       \setunit{\subtitlepunct}%
       \printfield[noformat]{issuesubtitle}}}}

%
%%%%%%%%%%%%%%%%%%%%%%%%%%%%%%%%%%%%%%%%%%%%%%%%%%%%%%%%%

%%%%%%%%%%%%%%%%%%%%%%%%%%%%%%%%%%%%%%%%%%%%%%%%%%%%%%%%%%
% (APA 6.29) Additional information for non-periodicals in
%            parenthesis after title. This is ugly but it's 
%            hard to put in "optional parens" like this
%            around an unknown amount of characters.

\newbibmacro*{addinfo}{%
  \ifthenelse{\iffieldundef{edition}\AND%
              \iffieldundef{chapter}\AND%
              \iffieldundef{volumes}\AND%
              \iffieldundef{pages}\AND%
              \iffieldundef{number}\AND%
              \(\iffieldundef{volume}\OR\boolean{bbx:volseen}\)}
  {}
  {\printtext{\bibopenparen}%
   \printfield{edition}%
   \setunit*{\addcomma\addspace}%
   \printfield{chapter}%
   \setunit*{\addcomma\addspace}%
   \notbool{bbx:volseen}%
     {\iffieldundef{volume}{}{\global\booltrue{bbx:volseen}}%
      \printfield{volume}%
      \iffieldundef{part}{}{\printfield{part}}}{}%
   \setunit*{\addcomma\addspace}%
   \printfield{number}%
   \setunit*{\addcomma\addspace}%
   \printfield{volumes}%
   \setunit*{\addcomma\addspace}%
   \printfield{pages}%
   \setunit{}%
   \printtext{\bibcloseparen}%
   \newunit}}

%
%%%%%%%%%%%%%%%%%%%%%%%%%%%%%%%%%%%%%%%%%%%%%%%%%%%%%%%%%%

%%%%%%%%%%%%%%%%%%%%%%%%%%%%%%%%%%%%%%%%%%%%%%%%%%%%%%%
% (APA 6.27) "In " precedes editor/trans list, no colon

\newbibmacro*{in}{%
  \ifbool{bbx:in}%
    {}%
    {\global\booltrue{bbx:in}%
     \bibcpstring{in}\setunit{\space}}}

%
%%%%%%%%%%%%%%%%%%%%%%%%%%%%%%%%%%%%%%%%%%%%%%%%%%%%%%%

%%%%%%%%%%%%%%%%%%%%%%%%%%%%%%%%%%%%%%%%%%%%%%%%%%%%%%%%%%%
% (APA 6.27) Editors have first names first ...
% (APA 7.02 Example 27) ... unless there is no author

\DeclareNameFormat{apanames}{%
  \ifthenelse{\value{listcount}=\maxprtauth\AND\value{listcount}<\value{listtotal}}
    {\addcomma\addspace\ldots}
    {\ifthenelse{\value{listcount}>\maxprtauth\AND\value{listcount}<\value{listtotal}}
      {}
      {\usebibmacro{name:apa:given-family}%
        {\namepartfamily}%
        {\namepartgiven}%
        {\namepartgiveni}%
        {\namepartprefix}%
        {\namepartsuffix}}}%
    \ifthenelse{\value{listcount}=\value{listtotal}}%
      {\ifmorenames{\andothersdelim\bibstring{andothers}}{}}{}}

\newbibmacro*{apa__author/editor}{%
  \ifnameundef{author}
    {\ifnameundef{editor}
      {}
      {\usebibmacro{editorinauthpos}}}
    {\global\boolfalse{bbx:editorinauthpos}%
     \usebibmacro{apa__author}}}

\newbibmacro*{editorinauthpos}{%
    \global\booltrue{bbx:editorinauthpos}%
    \printnames[apaauthor][-\value{listtotal}]{editor}%
    \setunit{\addspace}%
    \ifnameundef{editor}
      {}
      {\printtext[parens]{\usebibmacro{apaeditorstrg}{editor}}%
       % need to clear editor so we don't get an "In" clause later
       % But we also need to set a flag to say we did this so we
       % don't lose sight of the fact we once had an editor for
       % various year placement tests
       \clearname{editor}%
       \setunit{\adddot\addspace}%
       \usebibmacro{labelyear+extrayear}%
       \setunit{\adddot\addspace}}}

%
%%%%%%%%%%%%%%%%%%%%%%%%%%%%%%%%%%%%%%%%%%%%%%%%%%%%%%%%%%%

%%%%%%%%%%%%%%%%%%%%%%%%%%%%%%%%%%%%%%%%%%%%%%%%%%%
% (APA 6.27) Name format. Don't capitalise prefixes
% (APA 6.27) Full name disambiguation using brackets

% #1 = family name
% #2 = given name
% #3 = given name (initials)
% #4 = name prefix
% #5 = name suffix

\newbibmacro*{name:apa:family-given}[5]{%
  \ifuseprefix
    {\usebibmacro{name:delim}{#4#1}%
     \usebibmacro{name:hook}{#4#1}%
     \ifdefvoid{#4}{}{%
       \mkbibnameprefix{#4}%
       \ifpunctmark{'}{}{\addhighpenspace}}%
     \mkbibnamefamily{#1\isdot}%
     \ifdefvoid{#2}{}{\addcomma\addlowpenspace\mkbibnamegiven{#3}\isdot%
                    \ifthenelse{\value{uniquename}>1}
                      {\addspace\mkbibbrackets{#2}}
                      {}}%
     \ifdefvoid{#5}{}{\addcomma\addlowpenspace\mkbibnamesuffix{#5}\isdot}}
    {\usebibmacro{name:delim}{#1}%
     \usebibmacro{name:hook}{#1}%
     \mkbibnamefamily{#1}\isdot
     \ifboolexpe{%
       test {\ifdefvoid{#2}}
       and
       test {\ifdefvoid{#4}}}
       {}
       {\addcomma}%
     \ifdefvoid{#2}{}{\addlowpenspace\mkbibnamegiven{#3}%
                    \ifthenelse{\value{uniquename}>1}
                      {\addspace\mkbibbrackets{#2}}
                      {}}%
     \ifdefvoid{#4}{}{%
       \addhighpenspace\mkbibnameprefix{#4}%
       \ifpunctmark{'}{}{\addhighpenspace}}%
     \ifdefvoid{#5}{}{\addcomma\addlowpenspace\mkbibnamesuffix{#5}\isdot}}}

\newbibmacro*{name:apa:given-family}[5]{%
  \ifuseprefix
    {\usebibmacro{name:delim}{#2}%
     \usebibmacro{name:hook}{#2}%
     \ifdefvoid{#2}{}{\mkbibnamegiven{#3}\isdot%
                    \ifthenelse{\value{uniquename}>1}
                      {\addspace\mkbibbrackets{#2}}
                      {}\addspace}%
     \ifdefvoid{#4}{}{%
       \mkbibnameprefix{#4\isdot}%
       \ifpunctmark{'}{}{\addhighpenspace}}%
     \mkbibnamefamily{#1\isdot}%
     \ifdefvoid{#5}{}{\addlowpenspace\mkbibnamesuffix{#5}\isdot}}
    {\usebibmacro{name:delim}{#1}%
     \usebibmacro{name:hook}{#1}%
     \ifdefvoid{#2}{}{\mkbibnamegiven{#3}\isdot%
                    \ifthenelse{\value{uniquename}>1}
                      {\addspace\mkbibbrackets{#2}}
                      {}\addspace}%
     \mkbibnamefamily{#1}\isdot
     \ifdefvoid{#5}{}{\addcomma\addlowpenspace\mkbibnamesuffix{#5}\isdot}}}

%
%%%%%%%%%%%%%%%%%%%%%%%%%%%%%%%%%%%%%%%%%%%%%%%%%%%

%%%%%%%%%%%%%%%%%%%%%%%%%%%%%%%%%%%%%%%%%%%%%%%%%%%%%%
% (APA 6.27) Editor string in parens after author list
% (APA 7.02 Example 21) Collapse editor and trans if same
% (APA 7.02 Example 26) Translator


% Separated out into book and in* macros because otherwise
% this makes one ugly, huge, unreadable beast.

\newbibmacro*{book:editor+trans}{%
  \ifthenelse{\ifnameundef{editor}\AND%
              \ifnameundef{editora}\AND%
              \ifnameundef{editorb}\AND%
              \ifnameundef{editorc}\AND%
              \ifnameundef{translator}}%
    {}%
    {\ifnamesequal{editor}{translator}%
       {\printtext{\bibopenparen}\global\booltrue{bbx:parens}%
        \printnames[apanames][-\value{listtotal}]{editor}%
        \setunit*{\addcomma\addspace}%
        \usebibmacro{apaeditorstrg}{editor}%
        \setunit*{\addspace\&\space}%
        \printtext{\bibcpstring{translator}}%
        \ifbool{bbx:parens}{\printtext{\bibcloseparen}\global\boolfalse{bbx:parens}}{}}
       {\printtext{\bibopenparen}\global\booltrue{bbx:parens}%
        \ifnameundef{editor}%
          {}%
          {\printnames[apanames][-\value{listtotal}]{editor}%
           \setunit{\addcomma\addspace}%
           \usebibmacro{apaeditorstrg}{editor}%
           \clearname{editor}%
           \setunit{\adddot}%
           \setunit*{\addspace\&\space}}%
        \ifnameundef{translator}%
          {\setunit{}}%
          {\printnames[apanames][-\value{listtotal}]{translator}%
           \setunit{\addcomma\addspace}%
           \printtext{\bibcpstring{translator}}%
           \clearname{translator}%
           \setunit{\adddot}}%
        \ifbool{bbx:parens}{\printtext{\bibcloseparen}\global\boolfalse{bbx:parens}}{}}}}
 
\newbibmacro*{editor+trans}{%
  \ifthenelse{\ifnameundef{editor}\AND%
              \ifnameundef{editora}\AND%
              \ifnameundef{editorb}\AND%
              \ifnameundef{editorc}\AND%
              \ifnameundef{translator}}%
    {\setunit{\adddot\addspace}}%
    {\ifnamesequal{editor}{translator}%
      {\usebibmacro{in}%
       \printnames[apanames][-\value{listtotal}]{editor}%
       \setunit{\addspace\bibopenparen\global\booltrue{bbx:parens}}%
       \usebibmacro{apaeditorstrg}{editor}%
       \setunit*{\addspace\&\space}%
       \printtext{\bibcpstring{translator}}%
       \ifbool{bbx:parens}{\printtext{\bibcloseparen}\global\boolfalse{bbx:parens}}{}}
      {\ifnameundef{translator}%
        {}%
        {\setunit{\addspace}%
         \printtext[parens]{\printnames[apanames][-\value{listtotal}]{translator}%
         \setunit{\addcomma\addspace}%
         \printtext{\bibcpstring{translator}}}%
         \clearname{translator}%
         \setunit{\adddot\addspace}}%
       \usebibmacro{in}%
       \usebibmacro{editorx}}%
     \setunit{\addcomma\addspace}}}

%
%%%%%%%%%%%%%%%%%%%%%%%%%%%%%%%%%%%%%%%%%%%%%%%%%%

%%%%%%%%%%%%%%%%%%%%%%%%%%%%%%%%%%%
% (APA 6.29) Special journal issues

%\renewapaa{abx@ffd@periodical@title}{#1\isdot}
%\renewapaa{abx@ffd@periodical@title}{\mkbibemph{#1}\isdot}


\newbibmacro*{apa__periodical}{%
  \iffieldundef{title}
    {}
    {\printtext[title]{%
       \printfield[apacase]{title}%
       \setunit{\subtitlepunct}%
       \printfield[apacase]{subtitle}}}}

\newbibmacro*{title+issuetitle}{%
  \usebibmacro{apa__periodical}%
  \newunit
  \usebibmacro{issue}%
  \setunit*{\addcomma\addspace}%
  \setunit{\addcomma\space}}

%
%%%%%%%%%%%%%%%%%%%%%%%%%%%%%%%%%%%

%%%%%%%%%%%%%%%%%%%%%%%%%%%%%%%%%%%%%%%%%
% (APA 7.02 Example 21) Original year

\DeclareFieldFormat{origyear}{\mkbibparens{\bibcpstring{origyear}~\thefield{origyear}}}

% Only give ORIGYEAR in references if both:
%   YEAR is also given
%   YEAR and ORIGYEAR are different

\newbibmacro*{origyear}{%
  \ifthenelse{\NOT\iffieldundef{labelyear}\AND\NOT\iffieldsequal{labelyear}{origyear}}
    {\printfield{origyear}}
    {}}

%
%%%%%%%%%%%%%%%%%%%%%%%%%%%%%%%%%%%%%%%%%

%%%%%%%%%%%%%%%%%%%%%%%%%%%%%%%%%%%%%%%%%%%%%%%%%%%%%%%%%%%%%
% (APA 7.02) "In " before booktitle, edited or not
% (APA 6.30) Non-periodical volume number followed by "."

\newbibmacro*{maintitle+title}{%
  \iffieldsequal{maintitle}{title}
    {\clearfield{maintitle}%
     \clearfield{mainsubtitle}%
     \clearfield{maintitleaddon}}
    {\iffieldundef{maintitle}
      {}
      {\usebibmacro{apa__maintitle}%
       \newunit\newblock
       \iffieldundef{volume}
         {}
         {\setunit{\global\booltrue{bbx:volseen}}%
          \printfield{volume}%
          \printfield{part}%
          \setunit{\adddot}%
          \printfield{number}%
          \setunit{\adddot\space}}}}%
  \usebibmacro{apa__title}%
  \newunit}

\newbibmacro*{maintitle+booktitle}{%
  \ifthenelse{\iffieldundef{maintitle}\AND\iffieldundef{booktitle}}
    {}
    {\usebibmacro{in}%
     \iffieldundef{maintitle}
      {}
      {\usebibmacro{apa__maintitle}%
       \newunit\newblock
       \iffieldundef{volume}
         {}
         {\setunit{\addspace\global\booltrue{bbx:volseen}}%
          \printfield{volume}%
          \printfield{part}%
          \setunit{\adddot}%
          \printfield{number}%
          \setunit{\adddot\addspace}}}%
    \usebibmacro{apa__booktitle}}}

\newbibmacro*{apa__booktitle}{%
  \ifthenelse{\iffieldundef{booktitle}\AND\iffieldundef{booksubtitle}}
    {}
    {\printtext[booktitle]{%
       \printfield[apacase]{booktitle}%
       \setunit{\subtitlepunct}%
       \printfield[apacase]{booksubtitle}}%
     \setunit{\addspace}}%
  \printfield{booktitleaddon}}

%
%%%%%%%%%%%%%%%%%%%%%%%%%%%%%%%%%%%%%%%%%%%%%%%%%%%%%%%%%%%%%

%%%%%%%%%%%%%%%%%%%%%%%%%%%%%%%%%%%%%%%%%%%
% (APA 6.29) Colon separates title/subtitle
%            Skip if following terminating punctuation

%\renewcommand*{\subtitlepunct}{\ifterm{}{\addcolon}\space}
\renewapa{subtitlepunct}{\ifterm{}{\addcolon}\space}

%
%%%%%%%%%%%%%%%%%%%%%%%%%%%%%%%%%%%%%%%%%%%

%%%%%%%%%%%%%%%%%%%%%%%%%%%%%%%%%%%%%%%%%%%%%%%%%%%%%%%%%%%%%%
% (APA 6.30) Format of volume depends on whether
%            there is a maintitle (what APA calls "series") or not.

%\DeclareFieldFormat{volume}{\iffieldundef{maintitle}
%                              {\bibcpstring{volume}~\apanum{#1}}
%                              {\mkbibemph{\bibcpstring{volume}~\apanum{#1}}}}

%
%%%%%%%%%%%%%%%%%%%%%%%%%%%%%%%%%%%%%%%%%%%%%%%%%%%%%%%%%%%%%%

%%%%%%%%%%%%%%%%%%%%%%%%%%%%%%%%%%%%%%%%%%%%%%%%%%%%%%%%%%%%%%%%
% (APA 6.30) Number for reports with no prefix if not
%            a numeral. Report number is optional and so must be
%            the parens.

%\DeclareFieldFormat{number}{\iffieldundef{volume}
%                             {\ifinteger{\thefield{number}}
%                               {\addspace\mkbibemph{\bibcpstring{number}~\apanum{#1}}\adddot}
%                               {\apanum{#1}}}
%                             {\mkbibemph{\apanum{#1}}}}
%
%\DeclareFieldFormat[report]{number}{\mkbibparens{\iffieldundef{type}
%                                                  {}
%                                                  {\printfield{type}\setunit{\addspace}}%
%                                                  \bibcpstring{number}~\apanum{#1}}}

\newbibmacro*{apa:reportnum}{%
  \iffieldundef{number}%
    {}%
    {\printfield{number}}}

%
%%%%%%%%%%%%%%%%%%%%%%%%%%%%%%%%%%%%%%%%%%%%%%%%%%%%%%%%%%%%%%%%

%%%%%%%%%%%%%%%%%%%%%%%%%%%%%%%%%%%%%%%%%%%
% (APA 6.30) Location only shows first item

%\DeclareListFormat{default}{%
%  \usebibmacro{list:delim}{#1}%
%  #1\isdot}

%
%%%%%%%%%%%%%%%%%%%%%%%%%%%%%%%%%%%%%%%%%%%

%%%%%%%%%%%%%%%%%%%%%%%%%%%%%%%%%%%%%%%%%%%%%%
% (APA 6.30) Periodicals emph number, location
%            and pages (with no prefix)

%\DeclareFieldFormat[periodical]{pages}{\mkbibemph{#1}}
%\DeclareFieldFormat[periodical]{number}{\mkbibemph{\apanum{#1}}}
%\DeclareListFormat[periodical]{location}{\mkbibemph{#1}}

%
%%%%%%%%%%%%%%%%%%%%%%%%%%%%%%%%%%%%%%%%%%%%%%

%%%%%%%%%%%%%%%%%%%%%%%%%%%%%%%%%%%%%%%%%%%%%%%%%%%%%%%%%%
% (APA 7.02 Example 38) Proceedings emph number, location
%                       and location. Pages have no prefix

%\DeclareListFormat[proceedings]{location}{\mkbibemph{#1}}
%\DeclareFieldFormat[proceedings]{pages}{#1}
%\DeclareFieldFormat[proceedings]{number}{\mkbibemph{\apanum{#1}}}
%\DeclareFieldFormat[proceedings]{volume}{\mkbibemph{\apanum{#1}}}

%
%%%%%%%%%%%%%%%%%%%%%%%%%%%%%%%%%%%%%%%%%%%%%%%%%%%%%%%%%%

%%%%%%%%%%%%%%%%%%%%%%%%%%%%%%%%%%%%%%%%%%%%%%%%%%%
% (APA 7.02 Example 40--44) Theses titles and volume
%                           Titles are not emph if only an abstract

%\DeclareFieldFormat[thesis]{title}{\mkbibemph{#1}}

%
%%%%%%%%%%%%%%%%%%%%%%%%%%%%%%%%%%%%%%%%%%%%%%%%%%%

%%%%%%%%%%%%%%%%%%%%%%%%%%%%%%%
% (APA 7.05) Unpublished theses

%\DeclareFieldFormat[unpublished]{title}{\mkbibemph{#1}}

%
%%%%%%%%%%%%%%%%%%%%%%%%%%%%%%%

%%%%%%%%%%%%%%%%%%%%%%%%%%%%%%%%%%
% (APA 7.06) Reviews

% The title of the review depends on the entrytype of the thing reviewed
%\DeclareFieldFormat[review]{title}{%
%  \entrydata*{\thefield{related}}{%
%    \iffieldequalstr{entrytype}{article}
%      {\mkbibemph{\printtext[apacase]{\thefield{savedtitle}}}}
%      {\printtext[apacase]{\thefield{savedtitle}}}}}

%\DeclareFieldFormat[review]{pages}{#1}
%\DeclareFieldFormat[review]{volume}{\mkbibemph{\apanum{#1}}}
%\DeclareFieldFormat[review]{number}{\mkbibparens{\apanum{#1}}}

\newbibmacro*{reviewauthor}{%
  \ifnameundef{author}
    {\booltrue{bbx:noreviewauthor}}
    {\usebibmacro{apa__author}}}

%
%%%%%%%%%%%%%%%%%%%%%%%%%%%%%%%%%%

%%%%%%%%%%%%%%%%%%%%%%%%%%%%%
% Media authors

\newbibmacro*{producer}{%
  \ifnameundef{producer}
    {}
    {\printnames[apaauthor][-\value{listtotal}]{producer}%
     \newunit
     \printtext[parens]{\bibcpstring{producer}}%
     \setunit*{\addcomma\addspace\&\addspace}}}

\newbibmacro*{director}{%
  \ifnameundef{director}
    {}
    {\printnames[apaauthor][-\value{listtotal}]{director}%
     \newunit
     \printtext[parens]{\bibcpstring{director}}%
     \setunit*{\addcomma\addspace\&\addspace}}}

\newbibmacro*{writer}{%
  \ifnameundef{writer}
    {}
    {\printnames[apaauthor][-\value{listtotal}]{writer}%
     \newunit
     \printtext[parens]{\bibcpstring{writer}}%
     \setunit*{\addcomma\addspace\&\addspace}}}

%
%%%%%%%%%%%%%%%%%%%%%%%%%%%%%

%%%%%%%%%%%%%%%%%
% (APA 7.07) Misc and data

%\DeclareFieldFormat[data]{title}{\mkbibemph{#1}}
%\DeclareFieldFormat[data]{entrysubtype}{\mkbibbrackets{#1}}
%\DeclareFieldFormat[misc]{entrysubtype}{\mkbibbrackets{#1}}
%\DeclareFieldFormat[misc]{nameaddon}{\mkbibparens{#1}}
%\DeclareFieldFormat[misc]{title}{\iffieldequalstr{titleisdescription}{true}{\mkbibbrackets{#1}}{{#1}\isdot}}

\newbibmacro*{datatitle}{%
  \iffieldundef{title}
    {\iffieldundef{entrysubtype}
      {}
      {\printfield{entrysubtype}}}
    {\iffieldundef{entrysubtype}
      {}
      {\usebibmacro{apa__title}\addspace
       \printfield{entrysubtype}}}}

%
%%%%%%%%%%%%%%%%%

%%%%%%%%%%%%%%%%%%%%%%%%
% (APA 7.07) Audiovisual

%\DeclareFieldFormat[video]{title}{\iffieldundef{maintitle}{\mkbibemph{#1}}{#1}}
%\DeclareFieldFormat[video]{maintitle}{\mkbibemph{#1}}
%\DeclareFieldFormat[video]{entrysubtype}{\mkbibbrackets{#1}}
%\DeclareFieldFormat[audio]{entrysubtype}{\mkbibbrackets{#1}}
%\DeclareFieldFormat[music]{title}{#1}
%\DeclareFieldFormat[music]{maintitle}{\mkbibemph{#1}}
%\DeclareFieldFormat[music]{mainsubtitle}{\mkbibemph{#1}}
%\DeclareFieldFormat[audio]{howpublished}{\mkbibparens{#1}}

\newbibmacro*{tvseries}{%
  \iffieldundef{maintitle}
    {}
    {\usebibmacro{in}%
     \ifnameundef{execproducer}
       {}
       {\printnames[apanames][-\value{listtotal}]{execproducer}%
         \addspace
         \printtext[parens]{\bibcpstring{execproducer}}}
     \setunit{\addcomma\addspace}%
     \usebibmacro{avmaintitle}%
     \newunit}}

\newbibmacro*{avmaintitle}{%
  \ifthenelse{\iffieldundef{maintitle}\AND\iffieldundef{mainsubtitle}}
    {}
    {\printtext[maintitle]{%
       \printfield[apacase]{maintitle}%
       \setunit{\subtitlepunct}%
       \printfield[apacase]{mainsubtitle}}%
    \setunit{\addspace}}
  \printfield{maintitleaddon}}

\newbibmacro*{album}{%
  \iffieldundef{maintitle}
    {}
    {\bibcpstring{on}%
     \setunit{\addspace}%
     \usebibmacro{avmaintitle}%
     \newunit}}


%
%%%%%%%%%%%%%%%%%%%%%%%%

%%%%%%%%%%%%%%%%%%%%
% (APA 7.10) Letter/Letters

%\DeclareFieldFormat[letter]{title}{\mkbibbrackets{#1}}
%\DeclareFieldFormat[letters]{title}{#1}
%\DeclareFieldFormat[letter]{number}{\mkbibparens{#1}}
%\DeclareFieldFormat[letters]{number}{\mkbibparens{#1}}

%
%%%%%%%%%%%%%%%%%%%%

%%%%%%%%%%%%%%%%%
% (APA 6.31) URLs

%\DeclareFieldFormat{url}{\url{#1}}
%\DeclareFieldFormat{abstracturl}{\url{#1}}
%\DeclareFieldFormat{urldate}{#1}

\newbibmacro*{apa__url+urldate}{%
     \ifthenelse{\(\iffieldundef{url}\AND\iffieldundef{abstracturl}\AND\iffieldundef{abstractloc}\)\OR\NOT\iffieldundef{doi}}
       {}
       {\ifthenelse{\iffieldundef{abstracturl}\AND\iffieldundef{abstractloc}}
         {}
         {\printtext{\bibcpstring{abstract}}\addspace}%
          \printtext{\bibstring{retrieved}}%
          \setunit{\addspace}%
          \iffieldundef{urlyear}
            {}
            {\printurldate%
             \setunit*{\addcomma\addspace}}%
          \printtext{\bibstring{from}}%
          \setunit*{\addspace}%
          \printfield{urldescription}%
          \setunit*{\addcolon\addspace}%
          \iffieldundef{url}{}{\printfield{url}\renewcommand*{\finentrypunct}{\relax}}%
          \iffieldundef{abstractloc}{}{\printfield{abstractloc}\renewcommand*{\finentrypunct}{\relax}}
          \iffieldundef{abstracturl}{}{\printfield{abstracturl}\renewcommand*{\finentrypunct}{\relax}}}}

%
%%%%%%%%%%%%%%%%%

%%%%%%%%%%%%%%%%%%%%%%%%%%%%%%%%%%%%%%%%%%%%%
% (APA 7.11) Non-emph titles for online items
%            Subtype in brackets


%\DeclareFieldFormat[online]{title}{#1}
%\DeclareFieldFormat[online]{entrysubtype}{\mkbibbrackets{#1}}

%
%%%%%%%%%%%%%%%%%%%%%%%%%%%%%%%%%%%%%%%%%%%%%

%%%%%%%%%%%%%%%%%%%%%%%%%%%%%%%%%%%%%
% (APA A7.07) Emph titles for patents

%\DeclareFieldFormat[patent]{title}{\mkbibemph{#1}}

%
%%%%%%%%%%%%%%%%%%%%%%%%%%%%%%%%%%%%%

%%%%%%%%%%%%%%%%%%%%%%%%%%%%%%%%%%%%%%%%%%%%%%%%%%%%%
% (APA 7.08 Example 56) Software has version in title

\DeclareFieldFormat[software]{title}{#1}
\DeclareFieldFormat[software]{version}{\mkbibparens{\bibcpstring{version}~#1}}

\newbibmacro*{apa:softwaretitle}{%
  \printtext[title]{%
  \printfield[apacase]{title}%
  \setunit{\subtitlepunct}%
  \printfield[apacase]{subtitle}}%
  \setunit{\addspace}%
  \iffieldundef{version}%
     {}
     {\printfield{version}}
  \printfield{titleaddon}}

%
%%%%%%%%%%%%%%%%%%%%%%%%%%%%%%%%%%%%%%%%%%%%%%%%%%%%%

%%%%%%%%%%%%%%%%%%%%%%%%%%%%%%%%%%%%%%%%%%%%%%%%%%%%%
% Hacky bits to suppress final period in some cases.
% * If there is an ORIGYEAR field (APA 7.02:21)
% * Unless there is a RELATED and ADDENDUM
\newbibmacro*{apa:finpunct}{%
  \iffieldundef{origyear}
    {\ifboolexpr{
       test {\iffieldundef{addendum}}
       and
       test {\iffieldundef{related}}}
     {}
     {\renewcommand*{\finentrypunct}{\relax}}}
    {\renewcommand*{\finentrypunct}{\relax}}}

%
%%%%%%%%%%%%%%%%%%%%%%%%%%%%%%%%%%%%%%%%%%%%%%%%%%%%%
       
%%%%%%%%%%%%%%%%%%%%%%%%%%%%%%%%%%%%%%%%%%%%%%%%%%%%%
% Related entries

%\DeclareFieldFormat{related:reprintfrom}{\mkbibparens{#1}}
%\DeclareFieldFormat{related:reviewof}{\mkbibbrackets{#1}}

\newbibmacro*{apa__related:reprintfrom}[1]{%
  \entrydata*{#1}{%
    \printtext{\mkbibemph{\printfield[apacase]{title}}}%
    \setunit{\bibpagespunct}%
    \printfield{pages}%
    \setunit{\addcomma\addspace}%
    \bibstring{byauthor}\addspace
    \printnames[apanames][-\value{listtotal}]{editor}%
    \printnames[apanames][-\value{listtotal}]{author}%
    \setunit{\addcomma\addspace}%
    \usebibmacro{location+publisher}%
    \newunit\newblock
    \usebibmacro{related}}}

\newbibmacro*{related:reviewof}[1]{%
  \setunit{}% Sanitize this in case no author
  \entrydata*{#1}{%
    \ifentrytype{article}
      {\printtext{\printfield[apacase]{title}}}
      {\printtext{\mkbibemph{\printfield[apacase]{title}}}}%
    \setunit{\addspace}%
    \bibstring{byauthor}\addspace
    \printnames[apanames][-\value{listtotal}]{author}%
    \printnames[apanames][-\value{listtotal}]{producer}%
    \printnames[apanames][-\value{listtotal}]{director}%
    \setunit{\addcomma\addspace}%
    \printdateextra
    \newunit\newblock
    \usebibmacro{related}}}

%
%%%%%%%%%%%%%%%%%%%%%%%%%%%%%%%%%%%%%%%%%%%%%%%%%%%%%

%%%%%%%%%%%%%%%%%%%%%%%%%%%%%%%%
% (APA 7.x) General type layouts

\renewapa{blx@bbx@article}{%
  \usebibmacro{bibindex}%
  \usebibmacro{begentry}%
  \usebibmacro{apa__author/editor}%
  \setunit{\labelnamepunct}\newblock
  \usebibmacro{apa__title}%
  \newunit\newblock
  \usebibmacro{journal+issuetitle}%
  \setunit{\bibpagespunct}%
  \printfield{pages}%
  \newunit\newblock
  \printfield{note}%
  \newunit\newblock
  \usebibmacro{doi+eprint+url}%
  \newunit\newblock
  \printfield{addendum}%
  \newunit\newblock
  \iftoggle{bbx:related}
    {\usebibmacro{related:init}%
     \usebibmacro{related}}
    {}%
  \usebibmacro{apa:finpunct}%
  \usebibmacro{apa:pageref}%
  \usebibmacro{finentry}}

\renewapa{blx@bbx@newsarticle}{%
  \usebibmacro{bibindex}%
  \usebibmacro{begentry}%
  \usebibmacro{apa__author/editor}%
  \setunit{\labelnamepunct}\newblock
  \usebibmacro{apa__title}%
  \newunit\newblock
  \usebibmacro{newspaper}%
  \setunit{\bibpagespunct}%
  \printfield{pages}%
  \newunit\newblock
  \printfield{entrysubtype}%
  \newunit\newblock
  \printfield{note}%
  \newunit\newblock
  \usebibmacro{doi+eprint+url}%
  \newunit\newblock
  \printfield{addendum}%
  \newunit\newblock
  \iftoggle{bbx:related}
    {\usebibmacro{related:init}%
     \usebibmacro{related}}
    {}%
  \usebibmacro{apa:finpunct}%
  \usebibmacro{apa:pageref}%
  \usebibmacro{finentry}}

\renewapa{blx@bbx@book}{%
  \usebibmacro{bibindex}%
  \usebibmacro{begentry}%
  \usebibmacro{apa__author/editor}%
  \setunit{\labelnamepunct}\newblock
  \usebibmacro{maintitle+title}%
  \setunit{\addspace}\newblock
  \usebibmacro{book:editor+trans}%
  \newunit\newblock
  \printfield{series}%
  \newunit\newblock
  \printfield{note}%
  \newunit\newblock
  \usebibmacro{location+publisher}%
  \newunit\newblock
  \usebibmacro{doi+eprint+url}%
  \newunit\newblock
  \usebibmacro{origyear}%
  \newunit\newblock
  \printfield{addendum}%
  \newunit\newblock
  \iftoggle{bbx:related}
    {\usebibmacro{related:init}%
     \usebibmacro{related}}
    {}%
  \usebibmacro{apa:pageref}%
  \usebibmacro{apa:finpunct}%
  \usebibmacro{finentry}}

\renewapa{blx@bbx@booklet}{%
  \usebibmacro{bibindex}%
  \usebibmacro{begentry}%
  \usebibmacro{apa__author/editor}%
  \setunit{\labelnamepunct}\newblock
  \usebibmacro{apa__title}%
  \newunit\newblock
  \printfield{howpublished}%
  \newunit\newblock
  \printfield{type}%
  \newunit\newblock
  \printfield{note}%
  \newunit\newblock
  \usebibmacro{location+publisher}%
  \newunit\newblock
  \usebibmacro{doi+eprint+url}%
  \newunit\newblock
  \printfield{addendum}%
  \newunit\newblock
  \iftoggle{bbx:related}
    {\usebibmacro{related:init}%
     \usebibmacro{related}}
    {}%
  \usebibmacro{apa:pageref}%
  \usebibmacro{apa:finpunct}%
  \usebibmacro{finentry}}

\renewapa{blx@bbx@proceedings}{%
  \usebibmacro{bibindex}%
  \usebibmacro{begentry}%
  \usebibmacro{apa__author/editor}%
  \setunit{\labelnamepunct}\newblock
  \usebibmacro{apa__title}%
  \newunit\newblock
  \usebibmacro{apa__booktitle}%
  \setunit{\addcomma\addspace}\newblock
  \usebibmacro{location+publisher}%
  \setunit{\addcomma\addspace}%
  \printfield{volume}%
  \setunit*{\adddot}%
  \printfield{number}%
  \setunit{\addcomma\addspace}
  \printfield{pages}%
  \newunit\newblock
  \printfield{note}%
  \newunit\newblock
  \printlist{organization}%
  \newunit\newblock
  \usebibmacro{doi+eprint+url}%
  \newunit\newblock
  \printfield{addendum}%
  \newunit\newblock
  \iftoggle{bbx:related}
    {\usebibmacro{related:init}%
     \usebibmacro{related}}
    {}%
  \usebibmacro{apa:pageref}%
  \usebibmacro{apa:finpunct}%
  \usebibmacro{finentry}}

\renewapa{blx@bbx@collection}{%
  \usebibmacro{bibindex}%
  \usebibmacro{begentry}%
  \usebibmacro{apa__author/editor}%
  \setunit{\labelnamepunct}\newblock
  \usebibmacro{apa__title}%
  \newunit\newblock
  \usebibmacro{location+publisher}%
  \setunit{\addcomma\addspace}%
  \printfield{number}%
  \newunit\newblock
  \printfield{series}%
  \newunit\newblock
  \printfield{note}%
  \newunit\newblock
  \usebibmacro{doi+eprint+url}%
  \newunit\newblock
  \printfield{addendum}%
  \newunit\newblock
  \iftoggle{bbx:related}
    {\usebibmacro{related:init}%
     \usebibmacro{related}}
    {}%
  \usebibmacro{apa:pageref}%
  \usebibmacro{apa:finpunct}%
  \usebibmacro{finentry}}

\renewapa{blx@bbx@inbook}{%
  \usebibmacro{bibindex}%
  \usebibmacro{begentry}%
  \usebibmacro{apa__author}%
  \setunit{\labelnamepunct}\newblock
  \usebibmacro{apa__title}%
  \ifthenelse{\NOT\iffieldundef{title}\OR\boolean{bbx:titleinauthpos}}{\newunit}{\setunit{\addspace}}\newblock
  \usebibmacro{editor+trans}%
  \newblock
  \usebibmacro{maintitle+booktitle}%
  \newblock
  \usebibmacro{addinfo}%
  \newunit\newblock
  \printfield{series}%
  \newunit\newblock
  \printfield{note}%
  \newunit\newblock
  \usebibmacro{location+publisher}%
  \newunit\newblock
  \usebibmacro{doi+eprint+url}%
  \setunit*{\addspace}\newblock
  \usebibmacro{origyear}%
  \newunit\newblock
  \printfield{addendum}%
  \newunit\newblock
  \iftoggle{bbx:related}
    {\usebibmacro{related:init}%
     \usebibmacro{related}}
    {}%
  \usebibmacro{apa:pageref}%
  \usebibmacro{apa:finpunct}%
  \usebibmacro{finentry}}

\renewapa{blx@bbx@incollection}{%
  \usebibmacro{bibindex}%
  \usebibmacro{begentry}%
  \usebibmacro{apa__author}%
  \setunit{\labelnamepunct}\newblock
  \usebibmacro{apa__title}%
  \ifthenelse{\NOT\iffieldundef{title}\OR\boolean{bbx:titleinauthpos}}{\newunit}{\setunit{\addspace}}\newblock
  \usebibmacro{editor+trans}%
  \setunit*{\addcomma\addspace}\newblock
  \usebibmacro{maintitle+booktitle}%
  \newblock
  \usebibmacro{addinfo}%
  \newunit\newblock
  \printfield{series}%
  \newunit\newblock
  \printfield{note}%
  \newunit\newblock
  \usebibmacro{location+publisher}%
  \newunit\newblock
  \usebibmacro{origyear}%
  \newunit\newblock
  \usebibmacro{doi+eprint+url}%
  \newunit\newblock
  \printfield{addendum}%
  \newunit\newblock
  \iftoggle{bbx:related}
    {\usebibmacro{related:init}%
     \usebibmacro{related}}
    {}%
  \usebibmacro{apa:pageref}%
  \usebibmacro{apa:finpunct}%
  \usebibmacro{finentry}}

\renewapa{blx@bbx@inproceedings}{%
  \usebibmacro{bibindex}%
  \usebibmacro{begentry}%
  \usebibmacro{apa__author}%
  \setunit{\labelnamepunct}\newblock
  \usebibmacro{apa__title}%
  \ifthenelse{\NOT\iffieldundef{title}\OR\boolean{bbx:titleinauthpos}}{\newunit}{\setunit{\addspace}}\newblock
  \usebibmacro{editor+trans}%
  \setunit*{\addcomma\addspace}\newblock
  \usebibmacro{maintitle+booktitle}%
  \iffieldundef{eventyear}{}{\setunit{\addcomma\addspace}}%
  \printeventdate
  \setunit*{\addspace}\newblock
  \usebibmacro{addinfo}%
  \newunit\newblock
  \printfield{series}%
  \newunit\newblock
  \printfield{note}%
  \newunit\newblock
  \printlist{organization}%
  \newunit
  \printfield[apacase]{eventtitle}%
  \newunit
  \printfield{venue}%
  \newunit\newblock
  \usebibmacro{location+publisher}%
  \newunit\newblock
  \usebibmacro{origyear}%
  \newunit\newblock
  \usebibmacro{doi+eprint+url}%
  \newunit\newblock
  \printfield{addendum}%
  \newunit\newblock
  \iftoggle{bbx:related}
    {\usebibmacro{related:init}%
     \usebibmacro{related}}
    {}%
  \usebibmacro{apa:pageref}%
  \usebibmacro{apa:finpunct}%
  \usebibmacro{finentry}}

\renewapa{blx@bbx@manual}{%
  \usebibmacro{bibindex}%
  \usebibmacro{begentry}%
  \usebibmacro{apa__author/editor}%
  \setunit{\labelnamepunct}\newblock
  \usebibmacro{apa__title}%
  \newunit\newblock
  \printfield{edition}%
  \newunit\newblock
  \printfield{series}%
  \newunit\newblock
  \printfield{type}%
  \newunit
  \printfield{version}%
  \newunit
  \printfield{note}%
  \newunit\newblock
  \printlist{organization}%
  \newunit
  \usebibmacro{location+publisher}%
  \newunit\newblock
  \usebibmacro{doi+eprint+url}%
  \newunit\newblock
  \printfield{addendum}%
  \newunit\newblock
  \iftoggle{bbx:related}
    {\usebibmacro{related:init}%
     \usebibmacro{related}}
    {}%
  \usebibmacro{apa:pageref}%
  \usebibmacro{apa:finpunct}%
  \usebibmacro{finentry}}

\renewapa{blx@bbx@online}{%
  \usebibmacro{bibindex}%
  \usebibmacro{begentry}%
  \usebibmacro{apa__author}%
  \setunit{\labelnamepunct}\newblock
  \usebibmacro{apa__title}%
  \ifthenelse{\iffieldundef{title}\AND\boolean{bbx:titleinauthpos}}{\newunit}{\setunit{\addspace}}\newblock
  \printfield{entrysubtype}%
  \addperiod\addspace
  \iftoggle{bbx:doi}
    {\printfield{doi}%
     \iffieldundef{doi}{}{\renewcommand*{\finentrypunct}{\relax}}}
    {}%
  \newunit\newblock
  \iftoggle{bbx:eprint}
    {\usebibmacro{eprint}%
     \iffieldundef{eprint}{}{\renewcommand*{\finentrypunct}{\relax}}}
    {}%
  \newunit\newblock
  \usebibmacro{apa__url+urldate}%
  \iffieldundef{url}{}{\renewcommand*{\finentrypunct}{\relax}}
  \newunit\newblock
  \printfield{addendum}%
  \newunit\newblock
  \iftoggle{bbx:related}
    {\usebibmacro{related:init}%
     \usebibmacro{related}}
    {}%
  \usebibmacro{apa:pageref}%
  \usebibmacro{apa:finpunct}%
  \usebibmacro{finentry}}

\renewapa{blx@bbx@patent}{%
  \usebibmacro{bibindex}%
  \usebibmacro{begentry}%
  \usebibmacro{apa__author}%
  \setunit{\labelnamepunct}\newblock
  \usebibmacro{apa__title}%
  \ifthenelse{\NOT\iffieldundef{title}\OR\boolean{bbx:titleinauthpos}}{\newunit}{\setunit{\addspace}}\newblock
  \setunit*{\addspace}%
  \printfield{number}%
  \newunit\newblock
  \usebibmacro{location+publisher}%
  \newunit\newblock
  \usebibmacro{byholder}%
  \newunit\newblock
  \printfield{note}%
  \newunit\newblock
  \usebibmacro{doi+eprint+url}%
  \newunit\newblock
  \printfield{addendum}%
  \newunit\newblock
  \iftoggle{bbx:related}
    {\usebibmacro{related:init}%
     \usebibmacro{related}}
    {}%
  \usebibmacro{apa:pageref}%
  \usebibmacro{apa:finpunct}%
  \usebibmacro{finentry}}

\renewapa{blx@bbx@periodical}{%
  \usebibmacro{bibindex}%
  \usebibmacro{begentry}%
  \usebibmacro{apa__author/editor}%
  \setunit{\labelnamepunct}\newblock
  \usebibmacro{title+issuetitle}%
  \setunit{\addcomma\addspace}
  \printlist{location}%
  \setunit{\addcomma\addspace}
  \printfield{volume}%
  \setunit*{\adddot}%
  \printfield{number}%
  \setunit{\addcomma\addspace}
  \printfield{pages}%
  \newunit\newblock
  \printfield{note}%
  \newunit\newblock
  \usebibmacro{doi+eprint+url}%
  \newunit\newblock
  \printfield{addendum}%
  \newunit\newblock
  \iftoggle{bbx:related}
    {\usebibmacro{related:init}%
     \usebibmacro{related}}
    {}%
  \usebibmacro{apa:pageref}%
  \usebibmacro{apa:finpunct}%
  \usebibmacro{finentry}}

\renewapa{blx@bbx@report}{%
  \usebibmacro{bibindex}%
  \usebibmacro{begentry}%
  \usebibmacro{apa__author/editor}%
  \setunit{\labelnamepunct}\newblock
  \usebibmacro{apa__title}%
  \usebibmacro{apa:reportnum}%
  \newunit\newblock
  \printlist{institution}%
  \newunit\newblock
  \printfield{note}%
  \newunit\newblock
  \usebibmacro{location+publisher}%
  \newunit\newblock
  \usebibmacro{doi+eprint+url}%
  \newunit\newblock
  \printfield{addendum}%
  \newunit\newblock
  \iftoggle{bbx:related}
    {\usebibmacro{related:init}%
     \usebibmacro{related}}
    {}%
  \usebibmacro{apa:pageref}%
  \usebibmacro{apa:finpunct}%
  \usebibmacro{finentry}}

\renewapa{blx@bbx@thesis}{%
  \usebibmacro{bibindex}%
  \usebibmacro{begentry}%
  \usebibmacro{apa__author}%
  \setunit{\labelnamepunct}\newblock
  \usebibmacro{apa__title}%
  \ifthenelse{\NOT\iffieldundef{title}\OR\boolean{bbx:titleinauthpos}}{\newunit}{\setunit{\addspace}}\newblock
  \usebibmacro{type+institution}%
  \newunit\newblock
  \printfield{note}%
  \newunit\newblock
  \usebibmacro{doi+eprint+url}%
  \newunit\newblock
  \printfield{addendum}%
  \newunit\newblock
  \iftoggle{bbx:related}
    {\usebibmacro{related:init}%
     \usebibmacro{related}}
    {}%
  \usebibmacro{apa:pageref}%
  \usebibmacro{apa:finpunct}%
  \usebibmacro{finentry}}

\renewapa{blx@bbx@review}{%
  \usebibmacro{bibindex}%
  \usebibmacro{begentry}%
  \usebibmacro{reviewauthor}%
  \setunit{\labelnamepunct}\newblock
  \usebibmacro{apa__title}%
  \newblock
  \iftoggle{bbx:related}
    {\usebibmacro{related:init}%
     \usebibmacro{related}}
    {}%
  \newunit\newblock
  \ifbool{bbx:noreviewauthor}{\usebibmacro{labelyear+extrayear}}{}%
  \newunit\newblock
  \usebibmacro{journal+issuetitle}%
  \newunit\newblock
  \printfield{note}%
  \setunit{\bibpagespunct}%
  \printfield{pages}%
  \newunit\newblock
  \usebibmacro{doi+eprint+url}%
  \newunit\newblock
  \printfield{addendum}%
  \usebibmacro{apa:pageref}%
  \usebibmacro{apa:finpunct}%
  \usebibmacro{finentry}}

\renewapa{blx@bbx@misc}{%
  \usebibmacro{bibindex}%
  \usebibmacro{begentry}%
  \usebibmacro{apa__author}%
  \newunit\newblock
  \usebibmacro{apa__title}%
  \ifthenelse{\iffieldundef{title}\AND\boolean{bbx:titleinauthpos}}{\newunit}{\setunit{\addspace}}\newblock
  \printfield{entrysubtype}%
  \newunit\newblock
  \printfield{howpublished}%
  \newunit\newblock
  \printfield{type}%
  \newunit
  \printfield{version}%
  \newunit
  \printfield{note}%
  \newunit\newblock
  \printlist{organization}%
  \newunit\newblock
  \usebibmacro{location+publisher}%
  \newunit\newblock
  \usebibmacro{doi+eprint+url}%
  \newunit\newblock
  \printfield{addendum}%
  \newunit\newblock
  \iftoggle{bbx:related}
    {\usebibmacro{related:init}%
     \usebibmacro{related}}
    {}%
  \usebibmacro{apa:pageref}%
  \usebibmacro{apa:finpunct}%
  \usebibmacro{finentry}}

\renewapa{blx@bbx@data}{%
  \usebibmacro{bibindex}%
  \usebibmacro{begentry}%
  \usebibmacro{apa__author}%
  \newunit\newblock
  \usebibmacro{datatitle}%
  \newunit\newblock
  \printfield{type}%
  \newunit
  \printfield{version}%
  \newunit
  \printfield{note}%
  \newunit\newblock
  \printlist{organization}%
  \newunit\newblock
  \usebibmacro{location+publisher}%
  \newunit\newblock
  \usebibmacro{doi+eprint+url}%
  \newunit\newblock
  \printfield{addendum}%
  \newunit\newblock
  \iftoggle{bbx:related}
    {\usebibmacro{related:init}%
     \usebibmacro{related}}
    {}%
  \usebibmacro{apa:pageref}%
  \usebibmacro{apa:finpunct}%
  \usebibmacro{finentry}}

\renewapa{blx@bbx@letter}{%
  \usebibmacro{bibindex}%
  \usebibmacro{begentry}%
  \usebibmacro{apa__author/editor}%
  \setunit{\labelnamepunct}\newblock
  \usebibmacro{apa__title}%
  \newunit\newblock
  \printlist{institution}%
  \setunit{\addspace}\newblock
  \printfield{number}%
  \newunit\newblock
  \printfield{note}%
  \setunit{\addcomma\addspace}\newblock
  \usebibmacro{location+publisher}%
  \newunit\newblock
  \usebibmacro{doi+eprint+url}%
  \newunit\newblock
  \printfield{addendum}%
  \newunit\newblock
  \iftoggle{bbx:related}
    {\usebibmacro{related:init}%
     \usebibmacro{related}}
    {}%
  \usebibmacro{apa:pageref}%
  \usebibmacro{apa:finpunct}%
  \usebibmacro{finentry}}

\renewapa{blx@bbx@letters}{%
  \usebibmacro{bibindex}%
  \usebibmacro{begentry}%
  \usebibmacro{apa__author/editor}%
  \setunit{\labelnamepunct}\newblock
  \usebibmacro{apa__title}%
  \newunit\newblock
  \printlist{institution}%
  \setunit{\addspace}\newblock
  \printfield{number}%
  \newunit\newblock
  \printfield{note}%
  \newunit\newblock
  \usebibmacro{location+publisher}%
  \newunit\newblock
  \usebibmacro{doi+eprint+url}%
  \newunit\newblock
  \printfield{addendum}%
  \newunit\newblock
  \iftoggle{bbx:related}
    {\usebibmacro{related:init}%
     \usebibmacro{related}}
    {}%
  \usebibmacro{apa:pageref}%
  \usebibmacro{apa:finpunct}%
  \usebibmacro{finentry}}

\renewapa{blx@bbx@video}{%
  \usebibmacro{bibindex}%
  \usebibmacro{begentry}%
  \usebibmacro{writer}%
  \usebibmacro{director}%
  \usebibmacro{producer}%
  \newunit\newblock
  \usebibmacro{labelyear+extrayear}%
  \setunit{\labelnamepunct}\newblock
  \usebibmacro{apa__title}%
  \setunit{\addspace}\newblock
  \printfield{entrysubtype}%
  \newunit\newblock
  \usebibmacro{tvseries}%
  \newunit\newblock
  \usebibmacro{location+publisher}%
  \newunit\newblock
  \usebibmacro{doi+eprint+url}%
  \newunit\newblock
  \printfield{addendum}%
  \newunit\newblock
  \iftoggle{bbx:related}
    {\usebibmacro{related:init}%
     \usebibmacro{related}}
    {}%
  \usebibmacro{apa:pageref}%
  \usebibmacro{apa:finpunct}%
  \usebibmacro{finentry}}

\renewapa{blx@bbx@movie}{%
  \usebibmacro{bibindex}%
  \usebibmacro{begentry}%
  \usebibmacro{writer}%
  \usebibmacro{director}%
  \usebibmacro{producer}%
  \newunit\newblock
  \usebibmacro{labelyear+extrayear}%
  \setunit{\labelnamepunct}\newblock
  \usebibmacro{apa__title}%
  \setunit{\addspace}\newblock
  \printfield{entrysubtype}%
  \newunit\newblock
  \usebibmacro{tvseries}%
  \newunit\newblock
  \usebibmacro{location+publisher}%
  \newunit\newblock
  \usebibmacro{doi+eprint+url}%
  \newunit\newblock
  \printfield{addendum}%
  \newunit\newblock
  \iftoggle{bbx:related}
    {\usebibmacro{related:init}%
     \usebibmacro{related}}
    {}%
  \usebibmacro{apa:pageref}%
  \usebibmacro{apa:finpunct}%
  \usebibmacro{finentry}}

\renewapa{blx@bbx@audio}{%
  \usebibmacro{bibindex}%
  \usebibmacro{begentry}%
  \usebibmacro{writer}%
  \usebibmacro{director}%
  \usebibmacro{producer}%
  \newunit\newblock
  \usebibmacro{labelyear+extrayear}%
  \setunit{\labelnamepunct}\newblock
  \usebibmacro{apa__title}%
  \setunit{\addspace}\newblock
  \printfield{entrysubtype}%
  \newunit\newblock
  \usebibmacro{location+publisher}%
  \newunit\newblock
  \usebibmacro{doi+eprint+url}%
  \newunit\newblock
  \printfield{addendum}%
  \newunit\newblock
  \iftoggle{bbx:related}
    {\usebibmacro{related:init}%
     \usebibmacro{related}}
    {}%
  \usebibmacro{apa:pageref}%
  \usebibmacro{apa:finpunct}%
  \usebibmacro{finentry}}

\renewapa{blx@bbx@music}{%
  \usebibmacro{bibindex}%
  \usebibmacro{begentry}%
  \usebibmacro{apa__author}%
  \setunit{\labelnamepunct}\newblock
  \usebibmacro{apa__title}%
  \ifthenelse{\NOT\iffieldundef{title}\OR\boolean{bbx:titleinauthpos}}{\newunit}{\setunit{\addspace}}\newblock
  \usebibmacro{album}%
  \newunit\newblock
  \usebibmacro{location+publisher}%
  \printfield{addendum}%
  \newunit\newblock
  \iftoggle{bbx:related}
    {\usebibmacro{related:init}%
     \usebibmacro{related}}
    {}%
  \usebibmacro{apa:pageref}%
  \usebibmacro{apa:finpunct}%
  \usebibmacro{finentry}}

\renewapa{blx@bbx@software}{%
  \usebibmacro{bibindex}%
  \usebibmacro{begentry}%
  \usebibmacro{apa__author/editor}%
  \setunit{\labelnamepunct}\newblock
  \usebibmacro{apa:softwaretitle}%
  \newunit\newblock
  \usebibmacro{location+publisher}%
  \newunit\newblock
  \usebibmacro{doi+eprint+url}%
  \printfield{addendum}%
  \newunit\newblock
  \iftoggle{bbx:related}
    {\usebibmacro{related:init}%
     \usebibmacro{related}}
    {}%
  \usebibmacro{apa:pageref}%
  \usebibmacro{apa:finpunct}%
  \usebibmacro{finentry}}

\renewapa{blx@bbx@unpublished}{%
  \usebibmacro{bibindex}%
  \usebibmacro{begentry}%
  \usebibmacro{apa__author/editor}%
  \setunit{\labelnamepunct}\newblock
  \usebibmacro{apa__title}%
  \newunit\newblock
  \printfield{howpublished}%
  \setunit{\addcomma\addspace}\newblock
  \printlist{institution}%
  \newunit\newblock
  \printfield{note}%
  \setunit*{\addcomma\addspace}\newblock
  \printlist{location}%
  \newunit\newblock
  \usebibmacro{doi+eprint+url}%
  \newunit\newblock
  \printfield{addendum}%
  \newunit\newblock
  \iftoggle{bbx:related}
    {\usebibmacro{related:init}%
     \usebibmacro{related}}
    {}%
  \usebibmacro{apa:pageref}%
  \usebibmacro{apa:finpunct}
  \usebibmacro{finentry}}

%
%%%%%%%%%%%%%%%%%%%%%%%%%%%%%%%%

%%%%%%%%%%%%%%%%%%%%%%%%%%%%%%%%
% Modified, common macros

\newbibmacro*{editorx}{%
  \ifnameundef{editor}
    {}
    {\printnames[apanames][-\value{listtotal}]{editor}%
     \setunit{\addspace}%
     \printtext[parens]{\usebibmacro{apaeditorstrg}{editor}}
     \clearname{editor}%
     \newunit}%
  \ifnameundef{editora}
    {}
    {\printnames[apanames][-\value{listtotal}]{editora}%
     \setunit{\addspace}%
     \printtext[parens]{\usebibmacro{apaeditorstrg}{editora}}
     \clearname{editora}%
     \newunit}%
  \ifnameundef{editorb}
    {}
    {\printnames[apanames][-\value{listtotal}]{editorb}%
     \setunit{\addspace}%
     \printtext[parens]{\usebibmacro{apaeditorstrg}{editorb}}
     \clearname{editorb}%
     \newunit}%
  \ifnameundef{editorc}
    {}
    {\printnames[apanames][-\value{listtotal}]{editorc}%
     \setunit{\addspace}%
     \printtext[parens]{\usebibmacro{apaeditorstrg}{editorc}}
     \clearname{editorc}%
     \newunit}}

\newbibmacro*{apaeditorstrg}[1]{%
  \iffieldundef{#1type}
    {\ifthenelse{\value{#1}>1\OR\ifandothers{#1}}
       {\bibcpstring{editors}}
       {\bibcpstring{editor}}}
    {\ifthenelse{\value{#1}>1\OR\ifandothers{#1}}
         {\bibcpstring{type\thefield{#1type}s}}
         {\bibcpstring{type\thefield{#1type}}}}}

\newbibmacro*{doi+eprint+url}{%
  \iftoggle{bbx:doi}
    {\printfield{doi}%
     \iffieldundef{doi}{}{\renewcommand*{\finentrypunct}{\relax}}}
    {}%
  \newunit\newblock
  \iftoggle{bbx:eprint}
    {\usebibmacro{eprint}%
     \iffieldundef{eprint}{}{\renewcommand*{\finentrypunct}{\relax}}}
    {}%
  \newunit\newblock
  \iftoggle{bbx:url}
    {\usebibmacro{apa__url+urldate}%
     \iffieldundef{url}{}{\renewcommand*{\finentrypunct}{\relax}}}
    {}}


\DeclareFieldFormat{doi}{%
  \iftoggle{bbx:apastyle}{doi\addcolon}{DOI\addcolon\space}%
  \ifhyperref
    {\href{https://dx.doi.org/#1}{\nolinkurl{#1}}}
    {\nolinkurl{#1}}}

% APA 6th 
\newbibmacro*{location+publisher}{%
  \printlist[default][1-1]{location}%
  \setunit*{\addcolon\space}%
  \printlist{publisher}%
  \newunit}

\newbibmacro*{type+institution}{%
  \setunit{\addspace}%
  \ifthenelse{\iffieldundef{type}\AND\iffieldundef{institution}}
    {}
    {\printtext[parens]{%
       \iflistundef{institution}
         {\setunit*{\addspace}}
         {\setunit*{\addcolon\space}}%
       \printfield{type}%
       \setunit*{\addcomma\space}%
       \printlist{institution}%
       \setunit*{\addcomma\space}%
       \printlist{location}}%
     \newunit}}

\newbibmacro*{labeltitle}{%
  \iffieldundef{label}
    {\printfield{title}%
     \clearfield{title}%
     \setunit{\addspace}%
     \printfield{entrysubtype}%
     \clearfield{entrysubtype}%
     \global\booltrue{bbx:titleinauthpos}}
    {\printfield{label}}}

%
%%%%%%%%%%%%%%%%%%%%%%%%%%%%%%%%

%%%%%%%%%%%%%%%%%%%%%%%%%%%%%%%%
% Wrapper for backrefs

\newbibmacro{apa:pageref}{%
  \ifbool{apa:backref}{\usebibmacro{pageref}}{}}

%
%%%%%%%%%%%%%%%%%%%%%%%%%%%%%%%%

%%%%%%%%%%%%%%%%%%%%%%%%%%%%%%%%
%

% Unchanged from authoryear-comp

%\DeclareBibliographyOption{dashed}[true]{%
%  \ifstrequal{#1}{true}
%    {\ExecuteBibliographyOptions{pagetracker}%
%     \newbibmacro*{apa__bbx:savehash}{\savefield{fullhash}{\bbx@lasthash}}}
%    {\ExecuteBibliographyOptions{pagetracker=false}%
%     \newbibmacro*{apa__bbx:savehash}{}}}

%\DeclareFieldFormat{shorthandwidth}{#1}
%\setlength{\bibitemsep}{0pt}

%\DeclareNameAlias{author}{sortname}
%\DeclareNameAlias{editor}{sortname}
%\DeclareNameAlias{translator}{sortname}

%\defbibenvironment{bibliography}
%  {\list
%     {}
%     {\setlength{\leftmargin}{\bibhang}%
%      \setlength{\itemindent}{-\leftmargin}%
%      \setlength{\itemsep}{\bibitemsep}%
%      \setlength{\parsep}{\bibparsep}}}
%  {\endlist}
%  {\item}
%
%\defbibenvironment{shorthands}
%  {\list
%     {\printfield[shorthandwidth]{shorthand}}
%     {\setlength{\labelwidth}{\shorthandwidth}%
%      \setlength{\leftmargin}{\labelwidth}%
%      \setlength{\labelsep}{\biblabelsep}%
%      \addtolength{\leftmargin}{\labelsep}%
%      \setlength{\itemsep}{\bibitemsep}%
%      \setlength{\parsep}{\bibparsep}%
%      \renewcommand*{\makelabel}[1]{##1\hss}}}
%  {\endlist}
%  {\item}

%\newbibmacro*{bbx:savehash}{}


%\newbool{bbx@inset}
%\DeclareBibliographyDriver{set}{%
%  \booltrue{bbx@inset}%
%  \entryset{}{}%
%  \newunit\newblock
%  \usebibmacro{setpageref}%
%  \finentry}

%\endinput


%\DefineBibliographyExtras[]{%
  \def\finalandcomma{\addcomma}%
  \protected\def\mkbibdateapalong#1#2#3{%
    \iffieldundef{#1}%
      {}%
      {\iffieldbibstring{#1}{\biblcstring{\thefield{#1}}}{\thefield{#1}}}%
    \iffieldundef{#2}%
      {}%
      {\iffieldundef{#1}%
        {}%
        {\addcomma\addspace}%
       \mkbibmonth{\thefield{#2}}}%
    \iffieldundef{#3}%
      {}%
      {\ifthenelse{\iffieldundef{#2}\OR\iffieldundef{#1}}%
        {}%
        {\addspace}%
       \stripzeros{\thefield{#3}}}}%
  \protected\def\mkbibdateapalongextra#1#2#3{%
    \iffieldundef{#1}%
      {}%
      {\iffieldbibstring{#1}{\biblcstring{\thefield{#1}}}{\thefield{#1}}\printfield{extrayear}}%
    \iffieldundef{#2}%
      {}%
      {\iffieldundef{#1}%
        {}%
        {\addcomma\addspace}%
       \mkbibmonth{\thefield{#2}}}%
    \iffieldundef{#3}%
      {}%
      {\ifthenelse{\iffieldundef{#2}\OR\iffieldundef{#1}}%
        {}%
        {\addspace}%
       \stripzeros{\thefield{#3}}}}%
  \protected\def\mkbibdateapalongmdy#1#2#3{%
    \iffieldundef{#2}%
      {}%
      {\mkbibmonth{\thefield{#2}}}%
    \iffieldundef{#3}%
      {}%
      {\addspace}%
       \stripzeros{\thefield{#3}}%
    \iffieldundef{#1}%
      {}%
      {\ifthenelse{\iffieldundef{#3}}%
        {\ifthenelse{\iffieldundef{#2}}%
          {}%
          {\addspace}}%
        {\addcomma\addspace}%
       \iffieldbibstring{#1}{\biblcstring{\thefield{#1}}}{\thefield{#1}}}}%
%}

\renewapa{newunitpunct}{\addperiod\space\bibsentence}
\renewapa{newblockpunct}{}


\NewBibliographyString{retrieved}
\NewBibliographyString{from}

\DefineBibliographyStrings{english}{%
  retrieved        = {retrieved},
  from             = {from},
  urlfrom = {url}
}

\newfontfamily\cyrillicukrtranslit[Mapping=../tec/ukr-to-latin]{Times New Roman}
\newfontfamily\cyrillicrustranslit[Mapping=../tec/rus-to-latin]{Times New Roman}
%\renewapa{bibfont}{...\cyrillicukrtranslit}

\renewbibmacro{begentry}{%
  \iftoggle{bbx:apastyle}{%
    \iffieldequalstr{langid}{ukrainian}%
      {\cyrillicukrtranslit}%
      {\iffieldequalstr{langid}{russian}{\cyrillicrustranslit}{\normalfont}}}%
    {}}


\newcommand{\insertbiblioapa}[1][]{
\normalfont
\global\settoggle{bbx:apastyle}{true}
\csdef{blx@beglangcite}{}
\csdef{blx@beglangbib}{}
\csdef{blx@endlangcite}{}
\csdef{blx@endlangbib}{}
\begin{otherlanguage}{english}
\printbibliography[keyword=bibliofull,section=0,title=\ifstrempty{#1}{References}{#1}]
\end{otherlanguage}
\csletcs{blx@beglangcite}{blx@beglang}
\csletcs{blx@beglangbib}{blx@beglang}
\csletcs{blx@endlangcite}{blx@endlang}
\csletcs{blx@endlangbib}{blx@endlang}
\global\settoggle{bbx:apastyle}{false}
}

\newcommand{\fullciteapa}[1]{
\normalfont
\global\settoggle{bbx:apastyle}{true}
\csdef{blx@beglangcite}{}
\csdef{blx@beglangbib}{}
\csdef{blx@endlangcite}{}
\csdef{blx@endlangbib}{}
\begin{otherlanguage}{english}
\fullcite{#1}
\end{otherlanguage}
\csletcs{blx@beglangcite}{blx@beglang}
\csletcs{blx@beglangbib}{blx@beglang}
\csletcs{blx@endlangcite}{blx@endlang}
\csletcs{blx@endlangbib}{blx@endlang}
\global\settoggle{bbx:apastyle}{false}
}



%\renewcommand{\printbibliography}[1][]{}
%\renewcommand{\fullcite}[1]{}