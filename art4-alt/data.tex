
\newcommand{\articleUDK}{656.073, 004.934}

\newcommand{\articleTitleUkr}{
Формалізація голосової інформації в системах диспетчерського контролю за рухом автотранспорту}

\newcommand{\articleTitleRus}{
Формализация голосовой информации в системах диспетчерского контроля за движением автотранспорта}

\newcommand{\articleTitleEng}{
Formalization of voice information in the systems of dispatching control over the movement of vehicles}

\newcommand{\annotationUkr}{
У роботі запропоновано метод формалізації голосової інформації в системах диспетчерського контролю за рухом автотранспорту. При цьому запропоновано використання системи, що складається з двох основних модулів: автоматичного фонетичного стенографа і ядра рефлекторної системи голосового управління, поточна реалізація яких визначає умови їх використання в моделі голосової взаємодії. Також для формалізації голосової інформації в моделі голосової взаємодії водія при диспетчерському контролі за рухом автотранспорту запропоновано використовувати згорткові нейронні мережі. При використанні такої системи непотрібно створювати ніяких словників, виконувати морфологічний, синтаксичний, семантичний аналіз тексту, а також виділяти слова і команди; виникнення реакції відбувається на звуковий потік, з якого система, як і людина сама «вміє виділяти» інформативну частину за максимальною визначеністю.
}

\newcommand{\annotationRus}{
В работе предложен метод формализации голосовой информации в системах диспетчерского контроля за движением автотранспорта. При этом предложено использование системы, состоящей из двух основных модулей: автоматического фонетического стенографа и ядра рефлекторной системы голосового управления, текущая реализация которых определяет условия их использования в модели голосового взаимодействия. Также для формализации голосовой информации в модели голосового взаимодействия водителя при диспетчерском контроле за движением автотранспорта предложено использовать сверточные нейронные сети. При использовании такой системы не нужно создавать никаких словарей, выполнять морфологический, синтаксический, семантический анализ текста, а также выделять слова и команды; возникновение реакции происходит на звуковой поток, из которого система, как и человек сама «умеет выделять» информативную часть по максимальной определенности.

}

\newcommand{\annotationEng}{
In the work proposes a method of formalizing voice information in the systems of dispatching control over the movement of vehicles. It was proposed to use a system consisting of two main modules: an automatic phonetic stenographer and a core of a voice control reflex system, the current implementation of which determines the conditions for their use in the voice interaction model. It is also proposed to use convolution neural networks to formalize voice information in the driver’s voice interaction model during traffic control. When using such a system, there is no need to create any dictionaries, perform morphological, syntactic, semantic analysis of the text, as well as highlight words and commands; the occurrence of the reaction occurs on the audio stream, from which the system, like the person, “can isolate” the informative part by the maximum certainty.

}

\newcommand{\keywordsUkr}{
голосова інформація, диспетчерський контроль, дистрибуція, метод, формалізація, згорткові нейронні мережі}

\newcommand{\keywordsRus}{
голосовая информация, диспетчерский контроль, дистрибуция, метод, формализация, сверточные нейронные сети}

\newcommand{\keywordsEng}{
voice information, dispatch control, distribution, method, formalization, convolution neural networks}
