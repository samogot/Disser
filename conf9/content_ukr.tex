\section{Вступ}
Одним з актуальних напрямів розвитку геоінформаційних систем є транспортна логістика, яка має враховувати топологію місцевості, дорожньої мережі та інші параметри доставки \cite{Markelov_2015}. Крім того ефективна геоінформаційна маршрутизація здатна зменшити обсяг викидів від транспортних засобів у атмосферу, а відповідно і ризики впливу на навколишнє середовище, що розглядається як одна із проблем в геологічних та суміжних науках \cite{Zhukov_2013}.

Транспортна логістика --- це система організації доставки, а саме переміщення будь-яких матеріальних предметів або речовин з однієї точки в іншу за оптимальним маршрутом. Транспортна логістика є частиною процесів дистрибуції, кур’єрської доставки, геологічних робіт та інших транспортних систем, що включають в себе перевезення вантажів. Важливим етапом в процесах транспортної логістики є так званий етап «останньої милі» --- останній етап доставки вантажу з розподільчого центру до клієнта. Цей етап найменш ефективний з усього ланцюгу поставок, і може коштувати до 28\% від усієї вартості доставки. \cite{Scott_2009}

\section{Результати дослідження}

Для транспортної логістики на етапі останньої милі дуже важливим є планування маршрутів, а також моніторинг та диспетчеризація процесу доставки. Адже якісний маршрут дозволяє зменшити транспортні витрати, а моніторинг — підвищує рівень сервісу в реакціях на позапланові ситуації.

На жаль сьогодні в більшості компаній планування відбувається недостатньо ефективно — логісти визначають, який транспортний засіб повезе який вантаж, але не створюють конкретний маршрут, залишаючи це рішення на водіїв. Це пов’язано в першу чергу з тим, що логісти розробляють планові маршрути вручну без залучення автоматизованих систем. Друга причина, що випливає з першої, — логісти не можуть гарантувати принципову виконуваність маршрутів, оскільки орієнтуються в більшій мірі на масо-габаритні параметри, а часові вимоги враховують лише частково. Адже масо-габаритні параметри можна порахувати сумарно незалежно від порядку об’їзду точок доставки, а часові параметри можливо перевірити лише для конкретного маршруту, побудова і прорахунок якого виходить за межі людських можливостей без використання технічних засобів. Оскільки набір точок доставки для кожного окремого транспортного засобу не є гарантовано виконуваним, то розраховувати маршрути окремих транспортних засобів не має сенсу — потрібно вирішувати задачу в цілому для всіх точок доставки та машин, що відноситься до суттєво складнішого класу задач — Vehicle Routing Problem (VRP).

Запровадження ГІС для автоматичного розрахунку маршрутів несе в собі декілька суттєвих переваг. По перше, це гарантованість принципової виконуваності маршрутів, без урахування позапланових ситуацій. По друге, це підвищення рівня сервісу, адже маючи конкретний маршрут з плановим часом прибуття, ми можемо повідомити його клієнту, скоротивши його час очікування. Наприклад якщо клієнт замовив доставку з 15:00 до 18:00, він не буде 3 години ”сидіти на стільці” в очікуванні доставки — він в цей час лише переважно буде знаходитися в місці доставки, але буде займатися своїми справами, можливо на щось відволічеться чи відійде на короткий час. Якщо ми повідомимо клієнту орієнтовний час доставки (наприклад з 17:00 по 17:30), то ми забезпечимо клієнту більшу свободу дій, та знизимо ймовірність того, що саме в час прибуття доставки, клієнт не зможе її прийняти вчасно, що призведе до затримки.

По трете, наявність планового маршруту підвищить можливості моніторингу та реакції на позапланові ситуації. Без планового маршруту, за допомогою GPS моніторингу можливо лише побачити де перебував транспортний засіб та де він зупинявся. Але оскільки невідомий план за яким рухається водій, не зрозуміло, чи зупинка означає обслуговування точки доставки, чи водій відклав цю точку пізніше за планом і просто проїжджав повз, а зупинка — наприклад, очікування світлофора. Крім того, не маючи плану руху транспортного засобу, не можна спрогнозувати, чи транспортний засіб встигає обслуговувати усі точки доставки вчасно — можливо побачити лише факт того, що якась із точок не відвідана, а замовлений клієнтом час уже вийшов. Маючи плановий маршрут, можливо в кожний момент часу спрогнозувати приблизний час прибуття на кожну з наступних точок із урахуванням можливого відставання, і побачити, чи не призводить це відставання до порушення замовлених клієнтом часових вікон в майбутньому. Маючи цю інформацію набагато легше прийняти своєчасні дії — повідомити водію про необхідність пришвидшити обслуговування точок або обговорити з клієнтом можливість перенесення часу доставки. Голосова...

Варто не відкидати також і можливу економію за рахунок покращення ефективності планових маршрутів після запровадження автоматизованого планування. Тим не менше, практика показує, що водій, який добре знає ввірену йому територію, планує свій маршрут на достатньо високому рівні. Іноді рівень, який може забезпечити водій з досвідом роботи, навіть кращий за автоматизовані рішення за рахунок наявності більш детальної інформації про карту району. Тим не менше, автоматизоване планування може відв’язати якість маршрутів від людського фактору — рівня досвіду кожного конкретного водія.

Точне вирішення задачі маршрутизації транспортних засобів неможливе для розмірів задач, з якими стикаються сучасні кур’єрські служби в містахмегаполісах. Навіть не всі евристичні алгоритми можуть задовольнити сучасні потреби у швидкості обрахунку. В залежності від бізнес-процесів конкретної компанії та організації системи обліку та контролю за помилками, бувають ситуації коли остаточна інформація про наявні замовлення отримується лише після прибуття вантажу на розподільчий пункт і часу на планування, завантаження та відправлення кур’єрів залишається дуже мало, тому час розрахунку стандартної задачі в 2–3 тисячі точок доставки не може займати більше 30–40 хвилин.

Основна проблема впровадження систем побудови планових маршрутів на практиці полягає у спротиві інноваціям на рівні кінцевих виконавців. Водії, особливо, якщо це наймані перевізники, а не співробітники компанії, відмовляються їхати по запропонованим програмою маршрутам. Перевізників не дуже хвилює глобальна оптимальність всіх маршрутів чи рівень сервісу кінцевого клієнта, якщо до цих параметрів не прив’язана їх платня. Вони звикли отримувати маршрутний лист сформований лише за територіальними та масогабаритними критеріями, а до часових обмежень кінцевих клієнтів ставитися доволі формально. Тому будь-які зміни звичних планів сприймаються негативно, аж до саботування всього процесу. Навіть в ідеально правильному рішенні може бути необхідність заїхати на територію іншого водія для доставки в деякі точки, які інший водій не встигне виконати або необхідність доставляти точки в районі, який водій погано знає, оскільки в його знайомому районі в певний день мале навантаження і він може весь бути обслугований сусідами без участі цього водія, а в іншому місці навпаки навантаження велике і потрібно залучення додаткових машин. Але більшість наявних евристик орієнтуються лише на сумарну вартість, і може робити подібного роду помилки в формуванні окремого маршруту, навіть коли в цьому немає нагальної необхідності.

Таким чином, розроблено евристичний алгоритм, який максимально враховує вимоги логістів та водіїв щодо оптимальності вибору точок для кожного конкретного маршруту, при цьому не відкидаючи глобальну оптимальність при високій швидкості обрахунку з урахуванням топології дорожньої мережі \cite{eng_art1}. 

Для подальшої оптимізації логістичих задач геоінформаційні системи мають бути доповнені компонентом оптимізації голосової взаємодії між диспетчером та водієм. Аналіз маси параметрів голосової взаємодії дав змогу виділити ключові: кількість часу, характеристика зупинки, факт закінчення виконання доставки.

Практика показує, що одним з найбільш визначальних вхідних параметрів, що може сильно вплинути на результат планування та можливість його втілення у реальність, є кількість часу, що запланована на обслуговування в точці доставки. Адже інші параметри визначені достатньо чітко - масо-габаритні параметри відомі заздалегідь, дозволені часові вікна визначає кінцевий клієнт. Для визначення часу та відстані руху по дорозі між двома точками, є багато вже розроблених інструментів, які прогнозують результат з достатнім рівнем похибки на основі статистичних даних. Але для визначення часу обслуговування точки немає достовірного джерела інформації: ані клієнт, ані водій, ані логіст не можуть його назвати. Найбільш розповсюджена помилка — писати всім точкам однаковий час, наприклад 10 чи 5 хвилин, незалежно від ваги та кількості вантажу який необхідно доставити або складності пошуку та під’їзду до точки доставки. Кращий варіант, який можна зустріти доволі часто - категоризація точок або конкретних замовлень, використовувати фіксований час обслуговування для всіх замовлень в категорії. Найбільш правильним варіантом, що може принести суттєву економію транспортного ресурсу при побудові планових маршрутів, є статистичний аналіз історії часу обслуговування точок. Моделювати кількість часу необхідного для виконання точки, потрібно виходячи з наступних параметрів: скільки часу використовувалося на обслуговування цієї та схожих точок в минулому, якими водіями та на яких машинах вони при цьому обслуговувалися, яку вагу, об’єм та кількість вантажу було доставлено, та інші. Питання вибору оптимального способу моделювання заслуговує окремого дослідження.

Тим не менше, для подібного статистичного аналізу, постає проблема збору цих історичних даних. Достатньо точно час зупинки можна визначити за допомогою аналізу GPS треку, але цей метод має ряд недоліків. По перше, GPS данні мають похибку, яка може збільшуватись як в залежності від якості апаратного забезпечення, так і в залежності від обслуговуваної території. Наприклад відомо що зонах висотної забудови GPS сигнал істотно погіршується, а іноді навіть втрачається повністю. Таке погіршення сигналу може згубно впливати на визначення часу зупинки або навіть факту зупинки взагалі. По друге, навіть якщо відкинути похибку GPS як несуттєву або прийнятну, постає питання співвідношення зупинок та точок доставки. У загальному випадку таке співвідношення однозначно можливе тільки якщо в заданому радіусі є лише одна зупинка та одна точка доставки. У випадках коли зупинка відбулася поза заданим радіусом, коли біля точки було декілька зупинок (в тому числі з причин хибної інтерпретації зупинки через похибки GPS), або, як це трапляється найчастіше, одна зупинка знаходиться близько до декількох точок, — однозначно співвіднести з якої з зупинок необхідно записати час обслуговування точки неможливо. В сучасному світі, в містах мегаполісах, ситуація коли необхідно зробити декілька доставок в один багатоквартирний будинок, або сусідні будинки зі спільним двором, відбувається достатньо часто, і це автоматично унеможливлює збір та аналіз великої частини статистичної інформації про час обслуговування точки на основі GPS даних.

Отже необхідно доповнити дані GPS додатковою інформацією про те, коли водій-експедитор закінчив виконання однієї з точок в межах єдиної зупинки і почав виконання наступної точки. Практика показує що спроби зобов’язати водія в цей момент діставати телефон/планшет і вибирати відповідну команду в мобільному додатку експедитора, у кращому випадку призводить лише до того, що водій відмітить всі точки як виконані ще до або вже після виконання всіх доставок на зупинці, адже маніпуляції з планшетом потребують часу, а руки в цей момент зазвичай зайняті. Для вирішення цієї проблеми потрібен зручний для водія/експедитора інтерфейс, який би не відволікав його від основного завдання. Таким може виступати голосовий інтерфейс, який буде сприймати команди про початок та завершення виконання доставки, моделі якого розроблено автором \cite{eng_art2}.

Подібний голосовий інтерфейс може надати додаткові переваги. Крім уже обговореного способу отримання точного часу обслуговування кожної доставки, ми можемо супроводжувати водія автоматичними голосовими командами уникаючи розповсюджених помилок. Наприклад визначивши із GPS даних що ми під'їжджаємо до декількох точок розташованих поруч, ми можемо нагадати водію не забути повідомити про завершення кожної із них вчасно, або помітивши що водій відхиляється від плану повідомити його (а також диспетчера) про це — особливо це може бути корисно для перевізників вогненебезпечних вантажів, таких як паливо, яким дозволено рухатись виключно по певним дозволеним маршрутам. Крім того за допомогою голосового інтерфейсу можна полегшити роботу водія формалізувавши його комунікацію з диспетчером у випадку непередбачуваних обставин. На основі переліку типових обставин з якими стикалися українські компанії була розроблена модель голосової взаємодії водій-диспетчер яка приймає від водія інформацію про наявні обставини: відміни, переноси обслуговування точки, проблеми з транспортним засобом, проблему у клієнта, тощо і відправляє їх диспетчеру в формалізованому вигляді. Такий підхід дозволяє звільнити руки водія вид телефону (що до речі заборонено Українським законодавством), та скоротити час на передачу важливої інформації та реагування на неї.

\section{Висновки}

Розглянуто роль ГІС в розвязанні логістичних задач з урахуванням топології місцевості. Обґрунтовано важливість компоненту автоматизації голосової взаємодії диспетчера та водія. Виділено ключові параметри голосової взаємодії, які підвищують ефективність маршрутизації: кількість часу, що запланована на обслуговування в точці доставки, характеристика зупинки, факт закінчення виконання доставки. Окреслено перспективи подальшого дослідження.