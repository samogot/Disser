У дисертаційній роботі вирішено актуальне наукове завдання розробки моделей і методів формалізації голосової інформації в системах диспетчерського контролю за рухом автотранспорту. Загалом можна зробити наступні висновки.

1. Дослідження теоретико-методологічних засад формалізації голосової інформації в системах дистрибуції показало, що значну роль в їх управлінні відіграють процеси голосової взаємодії особливо стосовно своєчасного коригування планових маршрутів руху автотранспорту. Розроблення моделі голосової взаємодії без блоку переведення звуку голосу в текст може принципово покращити автоматизацію голосової взаємодії в системах контролю дистрибуції.

2. Розроблена система автоматичного розрахунку планових маршрутів та практика її використання забезпечили накопичення параметрів непередбачуваних ситуацій в процесі доставки, що впливають на створення сценаріїв голосової взаємодії, які представляються у вигляді орієнтованого графу та контекстів взаємодії. 
Принципи побудови рефлекторних систем на основі теорії несилової взаємодії адаптовано для формалізації голосової інформації в системах диспетчерського контролю за рухом автотранспорту.

3. Розроблено математичну модель голосової взаємодії водія та диспетчера в системах диспетчерського контролю за рухом автотранспорту, яка представлена у вигляді повного графу сценаріїв усіх етапів дистрибуції «склад – дорога – точка доставки». Виділено перелік унікальних контекстів голосової взаємодії, формалізація голосової інформації в яких може відбуватися незалежно, що дозволяє знизити кількість реакцій для автоматизованого розпізнання.

4. Розроблено метод формалізації голосової інформації в системах підтримки диспетчеризації автотранспорту з використанням інтелектуальних рефлекторних систем, що дозволяє автоматизувати процес передачі голосової інформації з уникненням переводу звукової інформації в лексичний текст за рахунок використання двох основних модулів (автоматичного фонетичного стенографа і ядра рефлекторної системи голосового управління). Для реалізації ядерного компонента запропоновано дуальну систему класифікації голосових команд, яка може використовувати метод інтелектуальних рефлекторних систем або метод згорткових нейронних мереж.

5. Метод структурної ідентифікації згорткових нейронних мереж для класифікації голосових команд адаптовано до розпізнавання фонемного тексту, що дозволяє класифікувати голосові команди без переведення голосу в лексичний текст. 

6. Метод інтелектуальних рефлекторних систем поєднано з теоретичним апаратом теорії нейронних мереж, шо дає можливість оптимізувати значення інформованості та визначеності шляхом навчання методом зворотного розповсюдження помилки.

7. Результати математичного моделювання формалізації голосової інформації показав підвищення ефективності розпізнавання повідомлень у голосовій взаємодії водія з диспетчером, а саме підвищення точності розпізнавання у середньому на 6.6 \% для згорткових нейронних мереж і на 19.1 \% для інтелектуальних рефлекторних систем за рахунок використання моделі голосової взаємодії водія та диспетчера. Крім того використання моделей на основі згорткових нейронних мереж показало підвищення швидкості розпізнавання на 15 \% порівняно з інтелектуальними рефлекторними системами.

8. Результати досліджень впроваджені в ТОВ «УІТ», м. Київ (довідка від 4 січня 2019) та використовувалися у трьох логістичних компаніях-клієнтах протягом року.

9. Мета досліджень щодо підвищення ефективності розпізнавання повідомлень у голосовій взаємодії водія з диспетчером досягнута іта всі часткові завдання вирішені повністю. Наукові результати досліджень є внеском у розвиток наукових і методологічних основ створення та застосування інформаційних технологій та інформаційних систем для автоматизованої переробки інформації й управління.

10. Перспективним шляхом подальших досліджень у зазначеному напрямку може бути широке коло питань щодо розробки та дослідження інших реалізацій фонемного стенографа, використання розроблених методів та моделей класифікації фонемного тексту для роботи з лексичним текстом, а також створення моделей голосової взаємодії у вигляді графу сценаріїв для інших предметних областей.
