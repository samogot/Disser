%% Согласно ГОСТ Р 7.0.11-2011:
%% 5.3.3 В заключении диссертации излагают итоги выполненного исследования, рекомендации, перспективы дальнейшей разработки темы.
%% 9.2.3 В заключении автореферата диссертации излагают итоги данного исследования, рекомендации и перспективы дальнейшей разработки темы.
У дисертаційній роботі вирішено актуальне наукове завдання розробки моделей і методів формалізації голосової інформації в системах диспетчерського контролю за рухом автотранспорту. Загалом можна зробити наступні висновки.

1. Дослідження теоретико-методологічних засад формалізації голосової інформації в системах дистрибуції показало, що значну роль в їх управлінні відіграють процеси голосової взаємодії особливо стосовно своєчасного коригування планових маршрутів руху автотранспорту. Розроблення моделі голосової взаємодії без блоку переведення звуку голосу в текст може принципово покращити автоматизацію голосової взаємодії в системах контролю дистрибуції.

%Сучасний етап автоматизації голосового управління в організаційно-технічних системах характеризується проблемою вчасного корегування маршруту, що призводить до витрат часу на комунікацію. Встановлено, що процес автоматизації управління в системах дистрибуції повинен включати етап моніторингу руху автомобілів у режимі реального часу, а існуючі системи є занадто простими, що не зможуть справитись з задачами дистрибуції для використання в управлінні транспортними доставками.

2. Розроблена система автоматичного розрахунку планових маршрутів та практика її використання забезпечили накопичення параметрів непередбачуваних ситуацій в процесі доставки, що впливають на створення сценаріїв голосової взаємодії, які представляються у вигляді орієнтованого графу та контекстів взаємодії. 
Принципи побудови рефлекторних систем на основі теорії несилової взаємодії адаптовано для формалізації голосової інформації в системах диспетчерського контролю за рухом автотранспорту.

%Проаналізовані сучасні інформаційні системи обробки та формалізації голосової інформації показали існування достатньої кількості таких методів. Встановлено, що недоліком традиційних систем розпізнання мови є те, що вони не забезпечують автоматизацію голосової взаємодії в задачах управління дистрибуцією, оскільки це потребує додаткових технічних та телекомунікаційних засобів. Орієнтуючись на новітні підходи до автоматизації голосової взаємодії встановлено необхідність застосування теорії несилової взаємодії з рефлекторною системою голосового управління, що включає аналіз інформаційної складової та виконання відомої реакції відповідним обʼєктом.

3. Розроблено модель голосової взаємодії субʼєктів дистрибуції в системах диспетчерського контролю за рухом автотранспорту, яка представлена у вигляді повного графу сценаріїв усіх етапів дистрибуції «склад – дорога – точка доставки». Виділено перелік унікальних контекстів голосової взаємодії, формалізація голосової інформації в яких може відбуватися незалежно, що дозволяє знизити кількість реакцій для автоматизованого розпізнання.

%Під час дослідження процесу автоматизації руху автотранспорту в дистрибуції та розгляд принципів побудови рефлекторної системи голосової взаємодії встановлено, що найбільш перспективним напрямом, який дає змогу запропонувати нове принципове рішення і побудувати рефлекторну модель голосової взаємодії в задачах управління дистрибуцією є застосування моделей логічних сценаріїв взаємодії у процесах дистрибуції, які мають враховувати параметри основних причин невідповідності реальної ситуації запланованому маршруту. Тому набула подальшого розвитку модель на трьох етапах дистрибуції, проблемні моменти в якій вирішуються із залученням диспетчера для вибору найкращої стратегії і мінімізації втрат, яка є основою для побудови дерева сценаріїв голосової взаємодії для кожного з субʼєктів (диспетчера та водія). Дерево сценаріїв запропоновано будувати у вигляді орієнтовного графу і використовувати такі сутності як: Контекст або Стан, Стимул або Подія, Реакція системи відповідно до стимулу. Крім того, для побудови рефлекторної системи голосової взаємодії встановлено закономірності застосування теорії несилової взаємодії як основи інтелектуальних рефлекторних систем та теоретично запропоновано використовувати рефлекторний метод в системі голосової взаємодії диспетчерського контролю за рухом автотранспорту.

4. Створено метод формалізації голосової інформації в системах підтримки диспетчеризації автотранспорту з використанням інтелектуальних рефлекторних систем, що дозволяє автоматизувати голосову взаємодію субʼєктів дистрибуції з уникненням переводу звукової інформації в лексичний текст за рахунок використання двох основних модулів (автоматичного фонетичного стенографа і ядра рефлекторної системи голосового управління). Для реалізації ядерного компонента запропоновано дуальну систему класифікації голосових команд, яка може використовувати метод інтелектуальних рефлекторних систем або метод згорткових нейронних мереж.

%Під час розгляду методів формалізації голосової інформації в системах диспетчерського контролю за рухом автотранспорту вперше отримано модель голосової взаємодії водія в системах диспетчерського контролю за рухом автотранспорту, яка представлена у вигляді дерева сценаріїв усіх етапів дистрибуції «склад – дорога – точка доставки» та визначено перелік контекстів та можливих реакцій на них.

%В системі формалізації голосової інформації в моделі голосової взаємодії водія при диспетчерському контролі за рухом автотранспорту удосконалено процес формалізації голосової інформації, що складається з двох основних модулів: автоматичного фонетичного стенографа і ядра рефлекторної системи голосового управління, поточна реалізація яких визначає умови їх використання в моделі голосової взаємодії;

%Для формалізації голосової інформації вперше запропоновано використовувати згорткові нейронні мережі. У даному випадку система формалізації голосової інформації в моделі голосової взаємодії водія при диспетчерському контролі за рухом автотранспорту містить фонемний стенограф та згорткову нейронну мережу, яка працює з фонемами.

%Крім того, отримали подальший розвиток інтелектуальні рефлекторні системи, що призначені для формалізації голосової інформації, що представлені у термінах згорткових нейронних мереж та, які дають можливість отримати оптимальні значення параметрів шляхом навчання методом зворотного розповсюдження помилки.

5. Оцінка ефективності дуальної системи формалізації голосової інформації проведена експериментальним шляхом у три етапи: на першому етапі первинного моделювання виявлено необхідність збільшення кількості вхідних даних; на другому перевірено гіпотезу недостатності кількості вхідних даних; на третьому --- гіпотезу недостатньої якості звукового сигналу. Прийнятний для практичного використання рівень точності в моделі, побудованій методом згорткових нейронних мереж досягнуто на другому етапі моделювання, а в моделі, побудованій методом інтелектуальних рефлекторних систем --- на третьому.

Аналіз результатів моделювання показав що застосування дерева сценаріїв та розбиття повного набору команд на контексти є доцільним, оскільки значення точності без використання контекстів є найменшими для обох методів класифікації. Порівняння результатів моделювання для різних методів класифікації показало що обидва методи можуть бути використані на практиці, причому навчання моделі при використанні методу інтелектуальних рефлекторних систем набагато швидше ніж для згорткових нейронних мереж, але фактичне розпізнавання на навченій системі відбувається швидше з використанням згорткових нейронних мереж. Точність моделювання також вища при використанні згорткових нейронних мереж.

6. Впровадження протягом року у трьох дистрибуційних компаніях підтвердило ефективність розробленої інформаційної технології формалізації голосової інформації: система підтримки диспетчеризації автотранспорту підвищує загальну ефективність процесу доставки за рахунок скорочення кількості необхідних транспортних засобів та підвищення кількості точок які можуть бути обслуговуванні одним транспортним засобом; запровадження голосового інтерфейсу може підвищити відсоток уникнення чи виправлення водіями інцидентів і відхилень від планового маршруту.

%У результаті проведених експериментальних досліджень формалізації голосової інформації, отриманої від водіїв, набув подальшого розвитку процес моделювання рефлекторними системами, в якому отримано нові дані під час апробації засобів формалізації голосової інформації в системах диспетчерського контролю за рухом автотранспорту, які призначені для оптимізації диспетчерського контролю за рухом автотранспорту при виконанні процесів дистрибуції. Крім того, результати досліджень при апробації засобів формалізації голосової інформації в системах диспетчерського контролю за рухом автотранспорту показали те, що з підвищенням загальної кількості стимулів відбувається зріст розпізнаних стимулів, що призводить до збільшення точності розпізнання.
