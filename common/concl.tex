У дисертаційній роботі вирішено актуальне наукове завдання розробки моделей і методів формалізації голосової інформації в системах диспетчерського контролю за рухом автотранспорту. Загалом можна зробити такі висновки.

1. Дослідження теоретико-методологічних засад формалізації голосової інформації в системах дистрибуції показало, що значну роль в їх управлінні відіграють процеси голосової взаємодії особливо щодо своєчасного коригування планових маршрутів руху автотранспорту. Розроблення моделі голосової взаємодії без блоку переведення звуку голосу в текст може принципово покращити автоматизацію голосової взаємодії в системах контролю дистрибуції.

2. Запропонована система автоматичного розрахунку планових маршрутів та практика її використання забезпечили накопичення параметрів непередбачуваних ситуацій в процесі доставки, що впливають на створення сценаріїв голосової взаємодії, які представлено у вигляді орієнтованого графу та контекстів взаємодії.
Принципи побудови рефлекторних систем на основі теорії несилової взаємодії адаптовано для формалізації голосової інформації в системах диспетчерського контролю за рухом автотранспорту.

3. Розроблено математичну модель голосової взаємодії водія та диспетчера в системах диспетчерського контролю за рухом автотранспорту, яка має вигляд повного графу сценаріїв усіх етапів дистрибуції «склад – дорога – точка доставки». Виділено перелік унікальних контекстів голосової взаємодії, формалізація голосової інформації в яких може відбуватися незалежно, що дає можливість зменшити кількість реакцій для автоматизованого розпізнання.

4. Розроблено метод формалізації голосової інформації в системах підтримки диспетчеризації автотранспорту з використанням інтелектуальних рефлекторних систем, що дає змогу автоматизувати процес передачі голосової інформації, уникнувши переведення звукової інформації в лексичний текст завдяки використанню двох основних модулів – автоматичного фонетичного стенографа і ядра рефлекторної системи голосового управління. Для реалізації ядерного компонента запропоновано дуальну систему класифікації голосових команд, яка може використовувати метод інтелектуальних рефлекторних систем або метод згорткових нейронних мереж.

5. Метод структурної ідентифікації згорткових нейронних мереж для класифікації голосових команд адаптовано до розпізнавання фонемного тексту, що дає змогу класифікувати голосові команди без переведення голосу в лексичний текст.

6. Метод інтелектуальних рефлекторних систем поєднано з понятійним апаратом теорії нейронних мереж, шо дає змогу оптимізувати значення інформованості та визначеності шляхом навчання моделі методом зворотного розповсюдження помилки.

7. Результати математичного моделювання формалізації голосової інформації засвідчили підвищення ефективності розпізнавання повідомлень у голосовій взаємодії водія з диспетчером, а саме підвищення точності розпізнавання в середньому на 6.6 \% для згорткових нейронних мереж і на 19.1 \% --- для інтелектуальних рефлекторних систем завдяки використанню моделі голосової взаємодії водія та диспетчера. Крім того, використання моделей на основі згорткових нейронних мереж дало підвищення швидкості розпізнавання на 15 \% порівняно з інтелектуальними рефлекторними системами.

8. Результати досліджень упроваджені в ТОВ «УІТ», м. Київ (довідка від 4 січня 2019 р.) та протягом року використовувалися в трьох логістичних компаніях-клієнтах.

9. Мету досліджень щодо підвищення ефективності розпізнавання повідомлень у голосовій взаємодії водія з диспетчером досягнуто, всі часткові завдання вирішено повністю. Наукові результати досліджень є внеском у розвиток наукових і методологічних основ створення та застосування інформаційних технологій та інформаційних систем для автоматизованої переробки інформації й управління.

10. Перспективним напрямком подальших досліджень у зазначеній сфері може бути вивчення широкого кола питань, зокрема щодо розробки та дослідження інших реалізацій фонемного стенографа, використання розроблених методів та моделей класифікації фонемного тексту для роботи з лексичним текстом, а також створення моделей голосової взаємодії у вигляді графу сценаріїв для інших предметних областей.