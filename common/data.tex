%%% Основные сведения %%%
\newcommand{\thesisAuthorLastName}{Найдьонов}
\newcommand{\thesisAuthorOtherNames}{Іван Михайлович}
\newcommand{\thesisAuthorInitials}{І.\,М.}
\newcommand{\thesisAuthor}             % Диссертация, ФИО автора
{%
    \texorpdfstring{% \texorpdfstring takes two arguments and uses the first for (La)TeX and the second for pdf
        \thesisAuthorLastName~\thesisAuthorOtherNames% так будет отображаться на титульном листе или в тексте, где будет использоваться переменная
    }{%
        \thesisAuthorLastName, \thesisAuthorOtherNames% эта запись для свойств pdf-файла. В таком виде, если pdf будет обработан программами для сбора библиографических сведений, будет правильно представлена фамилия.
    }
}
\newcommand{\thesisAuthorShort}        % Диссертация, ФИО автора инициалами
{\thesisAuthorInitials~\thesisAuthorLastName}
\newcommand{\thesisUdk}                % Диссертация, УДК
{\todo{xxx.xxx}}
\newcommand{\thesisTitle}              % Диссертация, название
{Інформаційна технологія формалізації голосової інформації в системах диспетчерського контролю за рухом автотранспорту}
\newcommand{\thesisSpecialtyNumber}    % Диссертация, специальность, номер
{05.13.06}
\newcommand{\thesisSpecialtyTitle}     % Диссертация, специальность, название
{інформаційні технології}
\newcommand{\thesisDegree}             % Диссертация, ученая степень
{кандидата технічних наук}
\newcommand{\thesisDegreeShort}        % Диссертация, ученая степень, краткая запись
{канд. техн. наук}
\newcommand{\thesisCity}               % Диссертация, город написания диссертации
{Київ}
\newcommand{\thesisYear}               % Диссертация, год написания диссертации
{2018}
\newcommand{\thesisOrganizationDone}            % Диссертация, ВНЗ або наукова установа, де виконана робота
{Київський національний університет імені Тараса Шевченка}
\newcommand{\thesisOrganizationMinistryDone}    % Диссертация, Орган, до сфери управління якого належить установа
{Міністерство освіти і науки України}
\newcommand{\thesisOrganizationDefence}         % Диссертация, Назва установи, де проводився захист дисертації
{Київський національний університет імені Тараса Шевченка}
\newcommand{\thesisOrganizationMinistryDefence} % Диссертация, Орган, до сфери управління якого належить установа
{Міністерство освіти і науки України}
\newcommand{\thesisOrganizationShort}  % Диссертация, краткое название организации для доклада
{КНУ ім. Т. Шевченка}

\newcommand{\thesisInOrganization}     % Диссертация, организация в предложном падеже: Работа выполнена в ...
{кафедрі технологій управління Київського
	національного університету імені Тараса Шевченка Міністерства освіти і науки
	України, м. Київ}

\newcommand{\supervisorFio}            % Научный руководитель, ФИО
{Тесля Юрій Миколайович}
\newcommand{\supervisorRegalia}        % Научный руководитель, регалии
{доктор технічних наук, професор}
\newcommand{\supervisorFioShort}       % Научный руководитель, ФИО
{Ю.\,М.~Тесля}
\newcommand{\supervisorRegaliaShort}   % Научный руководитель, регалии
{д.т.н., проф.}


\newcommand{\opponentOneFio}           % Оппонент 1, ФИО
{\todo{Фамилия Имя Отчество}}
\newcommand{\opponentOneRegalia}       % Оппонент 1, регалии
{\todo{доктор физико-математических наук, профессор}}
\newcommand{\opponentOneJobPlace}      % Оппонент 1, место работы
{\todo{Не очень длинное название для места работы}}
\newcommand{\opponentOneJobPost}       % Оппонент 1, должность
{\todo{старший научный сотрудник}}

\newcommand{\opponentTwoFio}           % Оппонент 2, ФИО
{\todo{Фамилия Имя Отчество}}
\newcommand{\opponentTwoRegalia}       % Оппонент 2, регалии
{\todo{кандидат физико-математических наук}}
\newcommand{\opponentTwoJobPlace}      % Оппонент 2, место работы
{\todo{Основное место работы c длинным длинным длинным длинным названием}}
\newcommand{\opponentTwoJobPost}       % Оппонент 2, должность
{\todo{старший научный сотрудник}}

\newcommand{\leadingOrganizationTitle} % Ведущая организация, дополнительные строки
{\todo{Федеральное государственное бюджетное образовательное учреждение высшего профессионального образования с~длинным длинным длинным длинным названием}}

\newcommand{\defenseDate}              % Защита, дата
{\todo{DD mmmmmmmm YYYY~г.~в~XX часов}}
\newcommand{\defenseCouncilNumber}     % Защита, номер диссертационного совета
{\todo{Д\,123.456.78}}
\newcommand{\defenseCouncilTitle}      % Защита, учреждение диссертационного совета
{\todo{Название учреждения}}
\newcommand{\defenseCouncilAddress}    % Защита, адрес учреждение диссертационного совета
{\todo{Адрес}}
\newcommand{\defenseCouncilPhone}      % Телефон для справок
{\todo{+7~(0000)~00-00-00}}

\newcommand{\defenseSecretaryFio}      % Секретарь диссертационного совета, ФИО
{\todo{Фамилия Имя Отчество}}
\newcommand{\defenseSecretaryRegalia}  % Секретарь диссертационного совета, регалии
{\todo{д-р~физ.-мат. наук}}            % Для сокращений есть ГОСТы, например: ГОСТ Р 7.0.12-2011 + http://base.garant.ru/179724/#block_30000

\newcommand{\synopsisLibrary}          % Автореферат, название библиотеки
{\todo{Название библиотеки}}
\newcommand{\synopsisDate}             % Автореферат, дата рассылки
{\todo{DD mmmmmmmm YYYY года}}

% To avoid conflict with beamer class use \providecommand
\providecommand{\keywords}%            % Ключевые слова для метаданных PDF диссертации и автореферата
{}
