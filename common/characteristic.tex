\textbf{Актуальність теми дослідження.} 
Системи диспетчерського контролю за рухом автотранспорту, призначені ефективно коригувати відхилення від запланованих маршрутів при зіткненні з непередбачуваними обставинами, потребують ефективного обміну повідомленнями між водієм і диспетчером. Різні форми автоматизації диспетчерського контролю (GPS, додатки з сенсорним інтерфейсом, мобільний інтернет) на сьогодні не здатні замінити голосову взаємодію, в якій диспетчер отримує необхідну для прийняття рішень інформацію зокрема про характер і причини відхилень від плану. 

Таким чином підвищення ефективності передачі повідомлень за рахунок формалізації голосової взаємодій між водієм та диспетчером є одним із перспективних напрямів вдосконалення системи диспетчерського контролю, що робить тему дисертаційного дослідження інформаційних технологій формалізації голосової інформації в системах диспетчерського контролю за рухом автотранспорту \textbf{актуальною}.

%
%.
%
%.
%
%.
%
%Системи диспетчерського контролю за рухом автотранспорту призвані (у тому числі) ефективно коригувати відхилення від запланованих маршрутів при зіткненні з непередбачуваними обставинами. Інформація про обставини має повідомлятися диспетчеру в найкоротші терміни для забезпечення можливості прийняття ефективних рішень.
%
%В існуючих системах диспетчерського контролю, параметри, такі як GPS трек, повідомляюся диспетчеру у формалізованому вигляді із застосуванням мобільного інтернету або супутникового звʼязку. З цих даних диспетчер має можливість бачити, що характеристика руху не відповідає запланованій, але не має інформації щодо причин такої невідповідності.
%
%Ця інформація може бути отримана за рахунок безпосередньої голосової взаємодії за допомогою мобільного телефону, або у формалізованому вигляді через мобільний додаток водія з сенсорним інтерфейсом.
%
%Водії часто уникають використовувати сенсорний додаток для передачі інформації про причини невідповідності руху та плану, через те що це відволікає від безпосередніх завдань керування автомобілем.
%
%Тобто єдиним каналом отримання цієї інформації залишається безпосередня, неформалізована голосова взаємодія. Проте така голосова взаємодія є витратною по часу як для водіїв так і для диспетчера і використання мобільного телефону є проблемою бо також відволікає від водійських функцій.
% 
%Підвищення ефективності передачі повідомлень є актуальним, оскільки неформалізована голосова взаємодія відбувається часто неефективно, а автоматизована часто уникання.
%
%Все це робить тему дисертаційного дослідження інформаційних технологій формалізації голосової інформації в системах диспетчерського контролю за рухом автотранспорту \textbf{актуальною}.
%
%.
%
%.
%
%.
%
%Системи диспетччерського контрою за рухом автотранспорту призначені для підвищення ефективності своєчасного реагування на незаплановані події.
%
%Вдосконалення систем диспеттчерського контрою призваного коригувати рух автомобіля актуальне в звязку з розвитком автоматизації і ....
%
%.
%
%Системи диспетчерського контролю за рухом автотранспорту призвані ефективно коригувати відхилення від запланованих маршрутів при зіткненні з непередбачуваними обставинами. Інформація про обставини має повідомлятися диспетчеру в найкоротщі терміни і з найменшими витратами часу для ефективних рішень. Голосові повідомлення часто потрибують більше часу і ...
%
%.
%
%.
%
%.
%
%В існуючих системах диспетчерського контролю не використовується голосовий звʼязок «водій --- обладнання автомобіля --- сервер». Всі параметри, такі як GPS трек, передаються диспетчеру у формалізованому вигляді із застосуванням мобільного інтернету або супутникового звʼязку.
%
%Диспетчер бачить що характеристика руху не відповідає запланованій і може її отримати або за рахунок безпосередньої голосової взаємодії за допомогою мобільного телефону, або у формалізованому вигляді через мобільний додаток водія з сенсорним інтерфейсом. 
%
%Водії часто уникають використовувати сенсорний додаток для передачі інформації про причини невідповідності руху та плану, через те що це відволікає від безпосередніх завдань керування автомобілем.
%
%Тобто єдиним каналом отримання цієї інформації залишається безпосередня, неформалізована голосова взаємодія. Проте така голосова взаємодія є витратною по часу як для водіїв так і для диспетчера і використання мобільного телефону є проблемою бо також відволікає від водійських функцій.
%
%.
%
%Формалізація голосової взаємодії потрібна, щоб забезпечити диспетчера швидкою і якісною інформацією про причини збоїв не відволікаючи водія від виконання його основних функцій.
%
%.
%
%
%Під час руху автотранспорту завжди відбуваюся ті чи інші відхилення від плану, які в кожному випадку потребують коригування плану через комунікацію з диспетчером.
%
%.
%
%.
%
%Навіть при отриманні диспетчером всіх параметрі руху у формалізованому вигляді із застосуванням мобільного інтернету або супутникового звʼязку залишається невідомою причина відхилення від плану, що має суттево значення для прийняття диспетчером рішення, щодо подальшого руху автотранспорту.
%
%.
%
%.
%
%.
%
%
%Розвиток прикладних інформаційних технологій на ринку транспортних послуг зумовлений посиленням жорсткої економічної конкуренції та запитом на підвищення екологічності, комфорту та ефективності роботи персоналу.
%
%\ifsynopsis
%\else
%Сьогодні, світові виробники автомобілів, електроніки та телекомунікаційних технологій створюють та використовують компʼютерні інформаційні системи у спроектованих та діючих транспортних засобах. За останні десятиліття, більшість автомобілів набуло оснащення інтерактивними інформаційними системами, що включають аудіо та відео системи, супутникові навігаційні системи, гарнітури телефонії і контроль над кліматом та технічним станом автомобіля. Не дивлячись на те, що такі системи обладнанні дисплеєм, голосова взаємодія з водієм стає все більш широко використовуваною в автомобілях, що допомагає збільшити кількість контрольованих функцій і систем, кнопки яких не можуть бути встановлені на рульовому колесі та приладовій панелі, оскільки обмежено простір. Голосова технологія також дозволяє водіям не відволікатись від управління, знижуючи ймовірність виникнення небезпечних ситуацій на дорозі та підвищуючи безпеку руху. Власними назвами систем голосового управління володіють такі бренди, як Mercedes-Benz, Ford, Cadillac. На марках автомобілів Audi, BMW, Kia, Lexus установлені системи голосового управління для зручності і забезпечення комфорту водіїв. Для систем голосового управління характерна різниця, що полягає у кількості підтримуваних мов, належному рівню при розпізнаванні команд, кількості реалізації функцій управління. Найбільшою кількістю мов володіє система «Ford Sync», крім того арсенал включає і російську мову, але української – немає.
%\fi
%
%У звʼязку з дедалі активнішим використанням природного інтерфейсу і зокрема голосу для спілкування водія з технічними засобами зросло і значення систем голосового управління в самому автомобілі як носія інформації у системах диспетчерського контролю за рухом автотранспорту при здійсненні етапів дистрибуції «склад – дорога – точка доставки».
%
%Не дивлячись на інтенсивний розвиток систем диспетчерського контролю за рухом автотранспорту при взаємодії із водієм, саме голосова інформація потребує формалізації у випадку проведення автоматизації таких систем. Проте існуючі розробки в сфері формалізації голосової інформації поки не пристосовані для аналізу мовлення водіїв, з метою покращення та полегшення їх взаємодії з диспетчерською системою. Саме модель голосової взаємодії водія в системах диспетчерського контролю потребує автоматизації для підвищення ефективності процесу дистрибуції.
%
%Фінальна доставка до дверей клієнта, відома як «остання миля», є одним з найдорожчих та найскладніших у організації дистрибуції. Під час виконання доставки завжди відбуваються ті чи інші відхилення від плану, яким би оптимальним він не був, подібні відхилення в кожному випадку потребують коригування плану через комунікацію з диспетчером. Водії-експедитори та кур'єри скоріше починають виконувати доставки поза планом, якщо процеси комунікації з диспетчером та коригування планів недостатньо прості та ефективні. Для задачі дистрибуції може бути важко забезпечити постійний доступ до мережі інтернет, оскільки доставка може відбуватися до місць/регіонів, де навіть мобільний GPRS інтернет відсутній, або має надто низьку швидкість передачі даних для роботи зі звуком.
%
%Інформаційних технологій які забезпечують автоматизацію голосової взаємодії в системах дистрибуції розроблено не достатньо.
%
%Все це робить тему дисертаційного дослідження інформаційних технологій формалізації голосової інформації в системах диспетчерського контролю за рухом автотранспорту \textbf{актуальною}.

\begin{refsection}
Питаннями формалізації голосової взаємодії, побудови діалогових систем та сценаріїв голосової взаємодії займалися такі вчені, як \citea{Ishiguro_2016,Iosif_2018,Herbert_2018,Lopes_2015,Khouzaimi_2018}, але їх дослідження не були спрямовані на сферу систем диспетчерського контролю за рухом автотранспорту. 

Досліджували голосове управління у транспортних системах \citea{Kravchenko_2009,Korsun_2013,Heisterkamp_2001,Jonsson_2009}, але більша частина результатів спрямована на автоматизацію голосового управління бортовим обладнанням.

Загалом різні способи покращення диспетчерського контролю при виконання доставок автотранспортом розглядали \citea{Prasanna_2012,Stopher_2018,Prasanna_2012,Liu_2018,Govindan_2018,Stopher_2018,Papetti_2019,,Gonzalez_2013,Comendador_2012,Baumann_2012,Quak_2006}, але їхні роботи не були спрямовані на формалізацію голосової взаємодії.

Питаннями формалізації голосової інформації за рахунок переведення її у текст розглядає велика кількість учених, зокрема \citea{Pylypenko_2008,Lydovyk_2011,Vasilyeva_2012,Womack_1999,Zirneeva_2008,Gladunov_2005,Robeyko_2012,Abdel_2012,Zhang_2017,Sharma_2018,Yermolenko_2008}, на сам перед шляхом переведення переведення голосу у текст, який, за рахунок необхідності великих словників та доступу до інтернету, має певні обмеження для використання на мобільних пристроях.

Вирішення зазначених суперечностей теорії та практики формалізації голосової інформації може спиратися на дослідження голосового управління, заснованих на теорії несилової взаємодії та рефлекторні системи голосового управління, що належать такими вченим, як Тесля~Ю.~М., Пилипенко~В.~В., Чорний~О.~Ю., Єгорченков~А.~В.
\end{refsection}

У звʼязку з цим, на даний час існує необхідність вирішення актуального наукового завдання розробки моделей і методів формалізації голосової інформації в системах диспетчерського контролю за рухом автотранспорту.

%Питаннями автоматизації систем голосового управління займалися такі вчені, як: Бондарос Ю.Г., Волков А.В., Кравченко А.П., Козлов О.С., Корсун О.М., Любімов А.М, Пилипенко В.В., Робейко В.В., Тесля Ю.М., Чорний О.Ю., Чучупал В.Я., Фінаєв І.М., Яцко А.А., Britz D., Deng L., Heisterkamp P., Hinton G., Jonsson I.-M., Kim Y., LeCun Y., Saini P., Yu D., Zhang X., Zhao J.J. та багато інших. Зокрема, результати досліджень голосового управління, заснованих на теорії несилової взаємодії та рефлекторної системи голосового управління належать таким вченим як: Тесля Ю.М., Пилипенко В.В., Чорний О.Ю., Єгорченков А.В.

\textbf{Звʼязок роботи з науковими програмами і планами.}

Дисертаційна робота виконана відповідно до пріоритетного напряму розвитку інформаційних та комунікаційних технологій, що визначені в Законі України «Про пріоритетні напрями розвитку науки і техніки» на період до 2020 року та тематичного плану науково-дослідних робіт Київського національного університету імені Тараса Шевченка в рамках науково-дослідної роботи «Розробка теоретико-методологічних основ впровадження систем управління проектами для розвитку підприємств і організацій» (№ держреєстрації 0117U002694), у якій автор брав участь як виконавець, запропонувавши впровадження графу сценаріїв для систем формалізації голосового управління.

\textbf{Обʼєктом дослідження} є процеси голосової взаємодії в системах диспетчерського контролю за рухом автотранспорту.

\textbf{Предмет дослідження} – моделі і методи формалізації голосової взаємодії в системах диспетчерського контролю за рухом автотранспорту.

\textbf{Метою дослідження} підвищення ефективності розпізнавання повідомлень у голосовій взаємодії водія з диспетчером на основі розробки та використання інформаційної технології формалізації голосової інформації в системах диспетчерського контролю за рухом автотранспорту.

Для досягнення сформульованої мети поставлено ряд часткових \textbf{завдань досліджень}:

\begin{itemize}
	\item здійснити аналіз сучасних інформаційних систем обробки та формалізації голосової інформації;
	\item розробити метод формалізації голосової інформації в допоміжних системах диспетчеризації автотранспорту;
	\item розробити математичну модель голосової взаємодії водія та диспетчера в системах диспетчерського контролю за рухом автотранспорту у вигляді повного графу сценаріїв усіх етапів процесу доставки «склад – дорога – точка доставки»;
	\item адаптувати метод структурної ідентифікації згорткових нейронних мереж для класифікації голосових команд для розпізнання фонемного тексту;
	\item поєднати метод інтелектуальних рефлекторних систем з теоретичним апаратом теорії нейронних мереж;
	\item провести експериментальні дослідження на основі математичного моделювання формалізації голосової інформації, отриманої від водіїв, для оптимізації диспетчерського контролю за рухом автотранспорту.
\end{itemize}

\textbf{Методи дослідження}. Для досягнення поставленої мети в роботі використано: теорію графів --- для опису моделі голосової взаємодії, теорію інформації та теорію несилової взаємодії --- для вдосконалення методу інтелектуальних рефлекторних систем, теорію штучних нейронних мереж та методи обробки природної мови --- для вдосконалення методу згорткових нейронних мереж.

\textbf{Наукова новизна отриманих результатів} полягає в тому, що в дисертаційній роботі:

\begin{itemize}
	\item вперше розроблено метод формалізації голосової інформації в системах підтримки диспетчеризації автотранспорту, який на відміну від аналогів поєднує використання інтелектуальних рефлекторних систем та згорткових нейронних мереж, що дозволяє автоматизувати процес передачі голосової інформації;
	\item удосконалено математичну модель голосової взаємодії водія та диспетчера в системах диспетчерського контролю за рухом автотранспорту, яка на відміну від існуючих представлена у вигляді повного графу сценаріїв усіх етапів процесу доставки «склад – дорога – точка доставки», що дозволяє виділити контексти голосової взаємодії для підвищення точності подальшої формалізації;
	\item набув подальшого розвитку метод структурної ідентифікації згорткових нейронних мереж для класифікації голосових команд, в якому на відміну від існуючих ведеться розпізнання фонемного тексту, що дозволяє класифікувати голосові команди без переведення голосу в лексичний текст;
	\item отримав подальший розвиток метод інтелектуальних рефлекторних систем, який відрізняється від існуюих поєднанням з теоретичним апаратом теорії нейронних мереж, шо дає можливість оптимізувати значення інформованості та визначеності шляхом навчання методом зворотного розповсюдження помилки.
\end{itemize}

\textbf{Практичне значення} отриманих результатів полягає в тому, що з використанням наукових результатів, закладається можливість підвищення точності та швидкості розпізнавання голосових повідомлень безпосередньо на мобільному пристрої, що покращує можливості диспетчерського контролю за рухом автотранспорту. Розроблені на базі запропонованих особисто автором моделей і методів програмні засоби становлять практичний результат, який впроваджений на підприємстві ТОВ «УІТ», м. Київ.

\textbf{Особистий внесок здобувача.} Наукові положення, розробки та висновки дисертаційної роботи є результатом самостійно проведеного дослідження здобувача. Основні наукові результати, представлені в дисертації, отримані здобувачем особисто.

\textbf{Апробація результатів досліджень.} Основні положення дисертаційної роботи були апробовані на 5-х міжнародних науково-технічних конференціях, в тому числі:

\begin{itemize}
	\item XVII Міжнародна науково-технічна конференція «Системний аналіз та інформаційні технології» (м. Київ, 22--25 червня 2015 р.)
	\item XII Міжнародна конференція «Управління проектами у розвитку суспільства», тема: «Комплексне управління проектами розвитку в умовах нестабільного оточення» (м. Київ, 21-23 травня 2015 р.)
	\item ІІІ Міжнародна науково-практична конференція «Інформаційні технології та взаємодії» (м. Київ, 8--10 листопада 2016 р.)
	\item 16th EAGE International Conference on Geoinformatics - Theoretical and Applied Aspects (м. Київ, 15--17 травня 2017 р.)
	\item ІV Міжнародна науково-практична конференція «Інформаційні технології та взаємодії» (м. Київ, 8--10 листопада 2017 р.)
\end{itemize}

\printbibliography[heading=countauthor, env=countauthor, keyword=biblioauthor, section=1]%
\printbibliography[heading=countauthorvak, env=countauthorvak, keyword=biblioauthorvak, section=1]%
\printbibliography[heading=countauthorconf, env=countauthorconf, keyword=biblioauthorconf, section=1]%

\textbf{Публікації.} 
За результатами дисертаційних досліджень опубліковано
\formbytotal{citeauthor}{науков}{у працю}{і праці}{их праць}. 
Основні наукові положення викладено у 
\formbytotal{citeauthorvak}{науков}{ій статті}{их статтях}{их статтях} \cite{art2,art3,art4,art5,art8},
серед яких \cite{art2,art3,art4,art5} опубліковані у спеціалізованих фахових виданнях України, 
\cite{art8} опубліковано у закордонному науковому виданні. 
За матеріалами виступів на науково-технічних конференціях опубліковано 
\formbytotal{citeauthorconf}{тез}{а}{и}{} доповідей \cite{conf5,conf6,conf8,conf9,conf10}.
Додатково результати досліджень відображені в науковій статті \cite{art1}.

\printbibliography[heading=countauthor, env=countauthor, keyword=biblioauthor]%

\ifsynopsis

%\todo{
\textbf{Структура і обсяг роботи.} Дисертаційна робота представлена на 246 сторінках друкованого тексту, включає 51 рисунок, 20 таблиць, які розташовані на 31 повній сторінці тексту. Робота складається з вступу, чотирьох розділів, висновків і списку використаних джерел із 142 найменуваннями, який розміщений на 21 сторінці. Основний текст викладений на 124 сторінках роботи.
%}

\else

\newtotcounter{MainPages}
\setcounter{MainPages}{\pagedifference{introduction}{conclusion_end}}
\newtotcounter{RefPages}
\setcounter{RefPages}{\pagedifference{references}{references_end}}

\textbf{Структура і обсяг роботи.} Дисертаційна робота представлена на \formbytotal{TotPages}{сторін}{ці}{ках}{ках} друкованого тексту, включає \formbytotal{totalcount@figure}{рисун}{ок}{ки}{ків}, \formbytotal{totalcount@table}{таблиц}{ю}{і}{ь}, які розташовані на 31 повній сторінці тексту. Робота складається з вступу, чотирьох розділів, загальних висновків і списку використаних джерел із \formbytotal{citenum}{найменуван}{ням}{нями}{ь}, який розміщений на \formbytotal{RefPages}{сторін}{ці}{ках}{ках}. Основний текст викладений на \formbytotal{MainPages}{сторін}{ці}{ках}{ках} роботи.

\fi

