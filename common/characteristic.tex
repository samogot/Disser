\textbf{Актуальність теми дослідження.} 
Системи диспетчерського контролю за рухом автотранспорту, призначені ефективно коригувати відхилення від запланованих маршрутів при зіткненні з непередбачуваними обставинами, потребують ефективного обміну повідомленнями між водієм і диспетчером. Різні форми автоматизації диспетчерського контролю (GPS, додатки з сенсорним інтерфейсом, мобільний інтернет) на сьогодні не здатні замінити голосову взаємодію, в якій диспетчер отримує необхідну для прийняття рішень інформацію зокрема про характер і причини відхилень від плану. 

Таким чином підвищення ефективності передачі повідомлень за рахунок формалізації голосової взаємодій між водієм та диспетчером є одним із перспективних напрямів вдосконалення системи диспетчерського контролю, що робить тему дисертаційного дослідження інформаційних технологій формалізації голосової інформації в системах диспетчерського контролю за рухом автотранспорту \textbf{актуальною}.

\begin{refsection}
Питаннями формалізації голосової взаємодії, побудови діалогових систем та сценаріїв голосової взаємодії займалися такі вчені, як \citea{Ishiguro_2016,Iosif_2018,Herbert_2018,Lopes_2015,Khouzaimi_2018}, але їх дослідження не були спрямовані на сферу систем диспетчерського контролю за рухом автотранспорту. 

Досліджували голосове управління у транспортних системах \citea{Kravchenko_2009,Korsun_2013,Heisterkamp_2001,Jonsson_2009}, але більша частина результатів спрямована на автоматизацію голосового управління бортовим обладнанням.

Загалом різні способи покращення диспетчерського контролю при виконання доставок автотранспортом розглядали \citea{Prasanna_2012,Stopher_2018,Prasanna_2012,Liu_2018,Govindan_2018,Stopher_2018,Papetti_2019,,Gonzalez_2013,Comendador_2012,Baumann_2012,Quak_2006}, але їхні роботи не були спрямовані на формалізацію голосової взаємодії.

Питаннями формалізації голосової інформації за рахунок переведення її у текст розглядає велика кількість учених, зокрема \citea{Pylypenko_2008,Lydovyk_2011,Vasilyeva_2012,Womack_1999,Zirneeva_2008,Gladunov_2005,Robeyko_2012,Abdel_2012,Zhang_2017,Sharma_2018,Yermolenko_2008,He_2019}, на сам перед шляхом переведення переведення голосу у текст, який, за рахунок необхідності великих словників та доступу до інтернету, має певні обмеження для використання на мобільних пристроях.

Вирішення зазначених суперечностей теорії та практики формалізації голосової інформації може спиратися на дослідження голосового управління, заснованих на теорії несилової взаємодії та рефлекторні системи голосового управління, що належать такими вченим, як Тесля~Ю.~М., Пилипенко~В.~В., Чорний~О.~Ю., Єгорченков~А.~В.
\end{refsection}

У звʼязку з цим, на даний час існує необхідність вирішення актуального наукового завдання розробки моделей і методів формалізації голосової інформації в системах диспетчерського контролю за рухом автотранспорту.

\textbf{Звʼязок роботи з науковими програмами і планами.}

Дисертаційна робота виконана відповідно до пріоритетного напряму розвитку інформаційних та комунікаційних технологій, що визначені в Законі України «Про пріоритетні напрями розвитку науки і техніки» на період до 2020 року та тематичного плану науково-дослідних робіт Київського національного університету імені Тараса Шевченка в рамках науково-дослідної роботи «Розробка теоретико-методологічних основ впровадження систем управління проектами для розвитку підприємств і організацій» (№ держреєстрації 0117U002694), у якій автор брав участь як виконавець, запропонувавши впровадження графу сценаріїв для систем формалізації голосового управління.

\textbf{Обʼєктом дослідження} є процеси голосової взаємодії в системах диспетчерського контролю за рухом автотранспорту.

\textbf{Предмет дослідження} – моделі і методи формалізації голосової взаємодії в системах диспетчерського контролю за рухом автотранспорту.

\textbf{Метою дослідження} підвищення ефективності розпізнавання повідомлень у голосовій взаємодії водія з диспетчером на основі розробки та використання інформаційної технології формалізації голосової інформації в системах диспетчерського контролю за рухом автотранспорту.

Для досягнення сформульованої мети поставлено ряд часткових \textbf{завдань досліджень}:

\begin{itemize}
	\item здійснити аналіз сучасних інформаційних систем обробки та формалізації голосової інформації;
	\item розробити метод формалізації голосової інформації в допоміжних системах диспетчеризації автотранспорту;
	\item розробити математичну модель голосової взаємодії водія та диспетчера в системах диспетчерського контролю за рухом автотранспорту у вигляді повного графу сценаріїв усіх етапів процесу доставки «склад – дорога – точка доставки»;
	\item адаптувати метод структурної ідентифікації згорткових нейронних мереж для класифікації голосових команд для розпізнання фонемного тексту;
	\item поєднати метод інтелектуальних рефлекторних систем з теоретичним апаратом теорії нейронних мереж;
	\item провести експериментальні дослідження на основі математичного моделювання формалізації голосової інформації, отриманої від водіїв, для оптимізації диспетчерського контролю за рухом автотранспорту.
\end{itemize}

\textbf{Методи дослідження}. Для досягнення поставленої мети в роботі використано: теорію графів --- для опису моделі голосової взаємодії, теорію інформації та теорію несилової взаємодії --- для вдосконалення методу інтелектуальних рефлекторних систем, теорію штучних нейронних мереж та методи обробки природної мови --- для вдосконалення методу згорткових нейронних мереж.

\textbf{Наукова новизна отриманих результатів} полягає в тому, що в дисертаційній роботі:

\begin{itemize}
	\item вперше розроблено метод формалізації голосової інформації в системах підтримки диспетчеризації автотранспорту, який на відміну від аналогів поєднує використання інтелектуальних рефлекторних систем та згорткових нейронних мереж, що дозволяє автоматизувати процес передачі голосової інформації;
	\item удосконалено математичну модель голосової взаємодії водія та диспетчера в системах диспетчерського контролю за рухом автотранспорту, яка на відміну від існуючих представлена у вигляді повного графу сценаріїв усіх етапів процесу доставки «склад – дорога – точка доставки», що дозволяє виділити контексти голосової взаємодії для підвищення точності подальшої формалізації;
	\item набув подальшого розвитку метод структурної ідентифікації згорткових нейронних мереж для класифікації голосових команд, в якому на відміну від існуючих ведеться розпізнання фонемного тексту, що дозволяє класифікувати голосові команди без переведення голосу в лексичний текст;
	\item отримав подальший розвиток метод інтелектуальних рефлекторних систем, який відрізняється від існуюих поєднанням з теоретичним апаратом теорії нейронних мереж, шо дає можливість оптимізувати значення інформованості та визначеності шляхом навчання методом зворотного розповсюдження помилки.
\end{itemize}

\textbf{Практичне значення} отриманих результатів полягає в тому, що з використанням наукових результатів, закладається можливість підвищення точності та швидкості розпізнавання голосових повідомлень безпосередньо на мобільному пристрої, що покращує можливості диспетчерського контролю за рухом автотранспорту. Розроблені на базі запропонованих особисто автором моделей і методів програмні засоби становлять практичний результат, який впроваджений на підприємстві ТОВ «УІТ», м. Київ.

\textbf{Особистий внесок здобувача.} Наукові положення, розробки та висновки дисертаційної роботи є результатом самостійно проведеного дослідження здобувача. Основні наукові результати, представлені в дисертації, отримані здобувачем особисто.

\textbf{Апробація результатів досліджень.} Основні положення дисертаційної роботи були апробовані на 5-х міжнародних науково-технічних конференціях, в тому числі:

\begin{itemize}
	\item XVII Міжнародна науково-технічна конференція «Системний аналіз та інформаційні технології» (м. Київ, 22--25 червня 2015 р.)
	\item XII Міжнародна конференція «Управління проектами у розвитку суспільства», тема: «Комплексне управління проектами розвитку в умовах нестабільного оточення» (м. Київ, 21-23 травня 2015 р.)
	\item ІІІ Міжнародна науково-практична конференція «Інформаційні технології та взаємодії» (м. Київ, 8--10 листопада 2016 р.)
	\item 16th EAGE International Conference on Geoinformatics - Theoretical and Applied Aspects (м. Київ, 15--17 травня 2017 р.)
	\item ІV Міжнародна науково-практична конференція «Інформаційні технології та взаємодії» (м. Київ, 8--10 листопада 2017 р.)
\end{itemize}

\printbibliography[heading=countauthor, env=countauthor, keyword=biblioauthor, section=1]
\printbibliography[heading=countauthorvak, env=countauthorvak, keyword=biblioauthorvak, section=1]
\printbibliography[heading=countauthorconf, env=countauthorconf, keyword=biblioauthorconf, section=1]

\textbf{Публікації.} 
За результатами дисертаційних досліджень опубліковано
\formbytotal{citeauthor}{науков}{у працю}{і праці}{их праць}. 
Основні наукові положення викладено у 
\formbytotal{citeauthorvak}{науков}{ій статті}{их статтях}{их статтях} \cite{art2,art3,art4,art5,art8},
серед яких \cite{art2,art3,art4,art5} опубліковані у спеціалізованих фахових виданнях України, 
\cite{art8} опубліковано у закордонному науковому виданні. 
За матеріалами виступів на науково-технічних конференціях опубліковано 
\formbytotal{citeauthorconf}{тез}{а}{и}{} доповідей \cite{conf5,conf6,conf8,conf9,conf10}.
Додатково результати досліджень відображені в науковій статті \cite{art1}.

\printbibliography[heading=countauthor, env=countauthor, keyword=biblioauthor]

\ifsynopsis


\textbf{Структура і обсяг роботи.} Дисертаційна робота представлена на 246 сторінках друкованого тексту, включає 51 рисунок, 20 таблиць, які розташовані на 31 повній сторінці тексту. Робота складається з вступу, чотирьох розділів, висновків і списку використаних джерел із 142 найменуваннями, який розміщений на 21 сторінці. Основний текст викладений на 124 сторінках роботи.


\else

\newtotcounter{MainPages}
\setcounter{MainPages}{\pagedifference{introduction}{conclusion_end}}
\newtotcounter{RefPages}
\setcounter{RefPages}{\pagedifference{references}{references_end}}

\textbf{Структура і обсяг роботи.} Дисертаційна робота представлена на \formbytotal{TotPages}{сторін}{ці}{ках}{ках} друкованого тексту, включає \formbytotal{totalcount@figure}{рисун}{ок}{ки}{ків}, \formbytotal{totalcount@table}{таблиц}{ю}{і}{ь}, які розташовані на 31 повній сторінці тексту. Робота складається з вступу, чотирьох розділів, загальних висновків і списку використаних джерел із \formbytotal{citenum}{найменуван}{ням}{нями}{ь}, який розміщений на \formbytotal{RefPages}{сторін}{ці}{ках}{ках}. Основний текст викладений на \formbytotal{MainPages}{сторін}{ці}{ках}{ках} роботи.

\fi

