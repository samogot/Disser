\chapter*{Анотація}

\textbf{\thesisAuthorLastName~\thesisAuthorInitials\ \thesisTitle.} – Кваліфікаційна наукова праця на
правах рукопису.

Дисертація на здобуття наукового ступеня \thesisDegree за
спеціальністю \thesisSpecialtyNumber – «\thesisSpecialtyTitle». – \thesisOrganizationDone, \thesisCity, \thesisYear.

Дисертаційна робота присвячена вирішенню актуального наукового завдання --- розробці моделей і методів формалізації голосової інформації в системах диспетчерського контролю за рухом автотранспорту. В процесі досліджень було розвʼязано ряд наукових завдань.

\textbf{Наукова новизна отриманих результатів} полягає в тому, що в дисертаційній роботі:

\begin{itemize}
	\item вперше розроблено метод формалізації голосової інформації в допоміжних системах диспетчеризації автотранспорту, який на відміну від аналогів поєднує використання інтелектуальних рефлекторних систем та згорткових нейронних мереж, що дозволяє автоматизувати процес передачі голосової інформації;
	\item удосконалено математичну модель голосової взаємодії водія та диспетчера в системах диспетчерського контролю за рухом автотранспорту, яка на відміну від існуючих представлена у вигляді повного графу сценаріїв усіх етапів процесу доставки «склад – дорога – точка доставки», що дозволяє виділити контексти голосової взаємодії для підвищення точності подальшої формалізації;
	\item набув подальшого розвитку метод структурної ідентифікації згорткових нейронних мереж для класифікації голосових команд, в якому на відміну від існуючих ведеться розпізнання фонемного тексту, що дозволяє класифікувати голосові команди без переведення голосу в лексичний текст;
	\item отримав подальший розвиток метод інтелектуальних рефлекторних систем, який відрізняється від існуюих поєднанням з теоретичним апаратом теорії нейронних мереж, шо дає можливість оптимізувати значення інформованості та визначеності шляхом навчання методом зворотного розповсюдження помилки.
\end{itemize}

\textbf{Практичне значення} отриманих результатів полягає в тому, що з використанням наукових результатів, закладається можливість підвищення точності та швидкості розпізнавання голосових повідомлень безпосередньо на мобільному пристрої, що покращує можливості диспетчерського контролю за рухом автотранспорту. Розроблені на базі запропонованих особисто автором моделей і методів програмні засоби становлять практичний результат, який впроваджений на підприємстві ТОВ «УІТ», м. Київ.

У \textbf{вступі} наведено загальну характеристику роботи, яка підкреслює її
актуальність, відповідність науковим темам, наукову новизну та практичне
значення, визначено предмет та обʼєкт дослідження, сформульовано мету та задачі
дослідження.

У \textbf{першому розділі} «Проблема формалізації голосової інформації в системах диспетчерського контролю за рухом автотранспорту» проведено аналіз сучасних інформаційних систем формалізації голосової інформації. 
Розглянувши проблему формалізації голосової інформації в системах диспетчерського контролю за рухом автотранспорту встановлено, що інформаційні технології в управлінні дистрибуцією є достатньо розробленими для забезпечення етапів отримання продукції та її збереження, але недостатньо --- для етапу доставки продукції до кінцевих клієнтів, особливо щодо проблеми «останньої милі». Значну роль в управлінні дистрибуцією відіграють процеси голосової взаємодії, автоматизація яких здатна підвищити ефективність системи дистрибуції.
На сучасному етапі автоматизації голосового управління в організаційно-технічних системах існує проблема своєчасного коригування в необхідних випадках планових маршрутів руху автотранспорту, що інколи призводить до достатньо великих витрат часу на комунікацію, і відповідно є найбільш обґрунтованим напрямом автоматизації голосової взаємодії.
Автоматизація голосової взаємодії для своєчасного коригування планових маршрутів руху автотранспорту повинна доповнити наявні засоби автоматизації управління в системах дистрибуції, такі як відстеження руху автомобілів у режимі реального часу за допомогою GPS треку. Існуючі в дистрибуції системи автоматизації голосової взаємодії для управління зберіганням є занадто спрощеними для використання в задачах доставки.
Проведений аналіз сучасних інформаційних систем формалізації голосової інформації показав, що традиційно вони включають три етапи: попередня обробка з виділенням ознак, перетворення голосової інформації в текстову та виділення змісту з текстової інформації. Такий підхід не в змозі повністю забезпечити автоматизацію голосової взаємодії в задачах управління дистрибуцією, оскільки переведення голосової інформації в лексичний текст потребує встановлення у кабіні водія потужного обладнання або забезпечення стабільного та швидкісного доступу до Інтернету. Висунуто ідею, що розроблення моделі голосової взаємодії без блоку переведення звуку голосу в текст може принципово покращити автоматизацію голосової взаємодії в системах контролю дистрибуції.
Встановлено пріоритетність підходу до голосового управління, заснованого на теорії несилової взаємодії з рефлекторною системою голосового управління. Підхід включає безпосередній аналіз інформаційної складової, вимовленої субʼєктом, з визначенням керуючого впливу з поміж відомих реакцій в певному предметному полі.
В результаті проведеного аналізу сформована основна ідея, гіпотеза, мета та спрямування дослідження, поставлено задачі досліджень дисертаційної роботи. Визначена необхідність в розробці моделей і методів, що дадуть змогу формалізувати голосову інформацію в системах диспетчерського контролю за рухом автотранспорту, тим самим забезпечити підвищення ефективності управління системою дистрибуції.

У \textbf{другому розділі} «Науково-методологічні основи автоматизації голосової взаємодії в системі диспетчеризації» сформульовано науково-методологічні основи автоматизації голосової взаємодії в системі диспетчеризації. 
Запропоновано концепцію створення системи автоматизації голосової взаємодії в задачах управління дистрибуцією, що має дві складові: (а) інтелектуальні рефлекторні системи голосового управління, що включають блок розпізнавання звукового сигналу та блок виділення його змісту; (б) модель сценаріїв взаємодії у процесах дистрибуції на трьох етапах доставки (завантаження на складі, дорога до точки доставки, розвантаження у точці доставки).
Розроблена система автоматичного розрахунку планових маршрутів та практика її використання забезпечили накопичення параметрів непередбачуваних ситуацій на плановому маршруті доставки, що впливають на створення сценаріїв голосової взаємодії.
Модель голосової взаємодії запропоновано будувати у вигляді орієнтованого графу, в якому вершини позначають стан системи та діалогові фрази, які буде озвучувати система, а ребра – репліки (стимули), які можуть бути сприйняті системою в кожній конкретній вершині, а множина всіх ребер, що виходять з однієї вершини буде позначати перелік стимулів розпізнання для її стану. У результаті для повноцінного опису запропоновано використовувати такі сутності: «контекст» (або «стан»), «стимул» (або «подія»), «реакція» системи відповідно до стимулу.
Принципи побудови рефлекторних систем на основі теорії несилової взаємодії адаптовано для автоматизації голосової взаємодії в системах диспетчерського контролю за рухом автотранспорту.

\textbf{Третій розділ} «Методи формалізації голосової інформації в системах диспетчерського контролю за рухом автотранспорту» присвячено створенню моделі та методів формалізації голосової інформації в системах диспетчерського контролю за рухом автотранспорту.
Запропонована класифікація реакцій для субʼєктів дистрибуції «склад – дорога – точка доставки» базується на зібраних статистичних даних (зауваженнях та оригінальних коментарях) щодо процесу доставки різних вантажів автомобільним транспортом у провідних логістичних компаніях України.
Розроблено модель голосової взаємодії субʼєктів дистрибуції в системах диспетчерського контролю за рухом автотранспорту, яка представлена у вигляді повного графу сценаріїв усіх етапів дистрибуції. 
Виділено перелік унікальних контекстів голосової взаємодії, формалізація голосової інформації в яких може відбуватися незалежно, що дозволяє знизити кількість реакцій для розпізнання.
Створено метод формалізації голосової інформації в системах підтримки диспетчеризації автотранспорту з використанням інтелектуальних рефлекторних систем, що дозволяє автоматизувати голосову взаємодію субʼєктів дистрибуції з уникненням переводу звукової інформації в лексичний текст за рахунок використання двох основних модулів (автоматичного фонетичного стенографа і ядра рефлекторної системи голосового управління).
Для реалізації ядерного компонента рефлекторної системи голосового управління запропоновано дуальну систему класифікації голосових команд, яка може бути налаштована на предметну область і використовувати метод інтелектуальних рефлекторних систем або метод згорткових нейронних мереж в залежності від того, який показує кращі результати.
Метод згорткових нейронних мереж застосовано до фонемного тексту з метою класифікації голосових команд для формалізації голосової інформації в системах диспетчерського контролю за рухом автотранспорту.
Для формалізації процесів взаємодії метод інтелектуальних рефлекторних систем представлено у термінах нейронних мереж, шо дає можливість отримати оптимальні значення параметрів рефлекторних систем шляхом навчання методом зворотного розповсюдження помилки.

У \textbf{четвертому розділі} «Засоби формалізації голосової інформації в системах диспетчерського контролю за рухом автотранспорту» описане проведене експериментальне дослідження та впровадження засобів формалізації голосової інформації в системах диспетчерського контролю за рухом автотранспорту.
Розроблений засіб формалізації голосової інформації дозволяє водію не відволікатись від управління автомобілем і слідкувати за дорожніми умовами та обстановкою, що дає змогу прискорити доставку продукції в процесі дистрибуції, а також підвищити рівень безпеки.
Розглянуті особливості використання розроблених засобів формалізації голосової інформації в системах диспетчерського контролю за рухом автотранспорту показали, що водію автомобіля, який буде здійснювати доставку продукції в процесі дистрибуції і вперше зіштовхнеться із засобом формалізації голосової інформації, що діє в рамках системи диспетчерського контролю за рухом автотранспорту, необхідно попередньо перевірити і при потребі донавчити систему розпізнавати його голосові команди з відповідних контекстів.
Перевірка моделювання розпізнавання команд на основі ітеративного процесу збору даних та введення нових критеріїв оцінки, якщо попередні не дали достатньої точності оцінювання, може забезпечити процес порівняння ефективності різних методів класифікації в дуальній моделі формалізації голосової взаємодії.
Використано розширений набір метрик оцінки ефективності моделей класифікації, що включав крім оцінки точності ще робастні метрики для незбалансованої вибірки (прецизійність, повноту, F-міру) та візуальний аналіз матриць помилок.
Оцінка ефективності дуальної системи формалізації голосової інформації проведена експериментальним шляхом у три етапи: на першому етапі первинного моделювання виявлено необхідність збільшення кількості вхідних даних; на другому перевірено гіпотезу недостатності кількості вхідних даних; на третьому --- гіпотезу недостатньої якості звукового сигналу. Прийнятний для практичного використання рівень точності в моделі, побудованій методом згорткових нейронних мереж досягнуто на другому етапі моделювання, а в моделі, побудованій методом інтелектуальних рефлекторних систем --- на третьому.

Впровадження протягом року у трьох дистрибуційних компаніях підтвердило ефективність розробленої інформаційної технології формалізації голосової інформації: система підтримки диспетчеризації автотранспорту підвищує загальну ефективність процесу доставки за рахунок скорочення кількості необхідних транспортних засобів та підвищення кількості точок які можуть бути обслуговуванні одним транспортним засобом; запровадження голосового інтерфейсу може підвищити відсоток уникнення чи виправлення водіями інцидентів і відхилень від планового маршруту.

\textbf{Ключові слова}: \keywords

\chapter*{Annotation}

\textbf{Naydonov I. M. Information technology of the formalization of voice information in systems of dispatch control of vehicle traffic.} --- Manuscript.

Thesis for the degree of candidate of technical sciences in the specialty 05.13.06 -- «information technologies». --- Taras Shevchenko National University of Kyiv, Kyiv, 2018.

Thesis research is devoted to the solution of the scientific problem --- the development of models and methods for the formalization of voice information in dispatch control systems of vehicle traffic, which can improve the management of the distribution process.

%The \textbf{scientific novelty} of the obtained results is that the scientific problem of integration of models and methods of formalization of voice information with the management of the distribution process has been solved for the first time in a single system of voice information formalization in dispatch control systems of vehicle traffic. Herewith: 
%
%\begin{itemize}
%	\item the model of voice interaction of distribution entities in dispatch control systems of vehicle traffic was developed \textit{for the first time}, which is presented as a complete communication script graph of all stages of the distribution «depot - road - delivery point», which allows to narrow the scope of voice interaction to the boundaries of the subject area; 
%	\item the method for formalizing voice information in vehicle dispatching support systems using intelligent reflex systems was created \textit{for the first time}, which allows automating voice interactions; 
%	\item the method of convolutional neural networks \textit{was further developed} for the classification of voice commands in order to formalize the voice information in the systems of dispatch control of vehicle traffic, which was applied to phonemic text; 
%	\item the methods of constructing of intelligent reflex systems \textit{was further developed} based on the use of neural networks to formalize the interaction processes, which makes it possible to obtain optimal values ​​of the parameters of the reflex systems by trainig using the backpropagation method.
%\end{itemize}
\textbf{Наукова новизна отриманих результатів} полягає в тому, що в дисертаційній роботі:

\begin{itemize}
	\item вперше розроблено метод формалізації голосової інформації в допоміжних системах диспетчеризації автотранспорту, який на відміну від аналогів поєднує використання інтелектуальних рефлекторних систем та згорткових нейронних мереж, що дозволяє автоматизувати процес передачі голосової інформації;
	\item удосконалено математичну модель голосової взаємодії водія та диспетчера в системах диспетчерського контролю за рухом автотранспорту, яка на відміну від існуючих представлена у вигляді повного графу сценаріїв усіх етапів процесу доставки «склад – дорога – точка доставки», що дозволяє виділити контексти голосової взаємодії для підвищення точності подальшої формалізації;
	\item набув подальшого розвитку метод структурної ідентифікації згорткових нейронних мереж для класифікації голосових команд, в якому на відміну від існуючих ведеться розпізнання фонемного тексту, що дозволяє класифікувати голосові команди без переведення голосу в лексичний текст;
	\item отримав подальший розвиток метод інтелектуальних рефлекторних систем, який відрізняється від існуюих поєднанням з теоретичним апаратом теорії нейронних мереж, шо дає можливість оптимізувати значення інформованості та визначеності шляхом навчання методом зворотного розповсюдження помилки.
\end{itemize}

\textbf{Практичне значення} отриманих результатів полягає в тому, що з використанням наукових результатів, закладається можливість підвищення точності та швидкості розпізнавання голосових повідомлень безпосередньо на мобільному пристрої, що покращує можливості диспетчерського контролю за рухом автотранспорту. Розроблені на базі запропонованих особисто автором моделей і методів програмні засоби становлять практичний результат, який впроваджений на підприємстві ТОВ «УІТ», м. Київ.

The \textbf{introduction} provides a general description of the work, which emphasizes its relevance, correspondence with scientific topics, scientific novelty and practical significance. It also contains the determined subject and object of the research, as well as the formulated purpose and objectives.

In the \textbf{first chapter} <<The problem of formalization of voice information in dispatch control systems of vehicle traffic>>, an analysis of modern information systems for the formalization of voice information was conducted.
Having considered the problem of formalization of voice information in dispatch control systems of vehicle traffic, it has been established that information technologies in distribution management are sufficiently developed to ensure the stages of obtaining a product and its preservation, but not enough --- for the stage of delivery of products to end customers, especially with the problem of <<last mile>>. Significant role in distribution management plays the processes of voice interaction, whose automation can improve the efficiency of the distribution system.
At the present stage of automation of voice control in organizational and technical systems there is a problem of timely correction in necessary cases of planned routes of vehicles, which sometimes leads to sufficiently large expenses for communication time, and accordingly is the most grounded direction of automation of voice interaction.
Voice interaction automation for timely correction of planned traffic routes should complement existing management automation tools in distribution systems, such as tracking real-time vehicle traffic using the GPS track. Existing in the distribution of the voice interaction automation system for storage management are too simplified for use in delivery tasks.
The conducted analysis of modern information systems for the formalization of voice information has shown that they traditionally include three stages: preliminary processing with the feature extraction, the transformation of voice information into text and the extraction of content from text information. Such an approach is not able to fully ensure the automation of voice interaction in distribution management tasks, since translating voice information into lexical text requires the installation of powerful equipment in the vehicle or stable and high-speed Internet access. The idea is that the development of a voice interaction model without a block of translating voice into a text can fundamentally improve the automation of voice interaction in distribution control systems.
The priority of the approach to voice control, based on the non-force interaction theory with the reflex voice-activated control system, is established. The approach includes a direct analysis of the information component expressed by the subject, with the definition of the control effect among known reactions in a certain subject field.
As a result of the analysis, the basic idea, hypothesis, purpose and direction of the research have been formed, the tasks of research of the thesis research have been set. The necessity to develop models and methods that will be able to formalize the voice information in dispatch control systems of vehicle traffic is determined, thus, to increase the efficiency of management of the distribution system.

In the \textbf{second chapter} <<Scientific and methodological bases of automation of voice interaction in the system of vehicle dispatching>> the scientific and methodological bases of automation of voice interaction in the system of vehicle dispatching are formulated.
The concept of the creation of a system for the automation of voice interaction in distribution management tasks, which has two components: (a) intelligent reflex voice-activated control system, including a unit of recognition of a sound signal and a block of understanding its content; (b) the model of interaction scenarios in the distribution processes at the three stages of delivery (loading in depot, road to delivery point, unloading at the point of delivery).
The developed system of automatic calculation of planned routes and the practice of its use ensured accumulation of parameters of unpredictable situations in the delivery process, which influence the creation of voice interaction scritrs, which are presented in the form of a targeted graph and interaction contexts.
The model of voice interaction is proposed to be constructed in the form of a directed graph in which the vertices denote the state of the system and the dialog phrases that will be expressed by the system, and the edges are replicas (stimuli) that can be perceived by the system at each particular vertex, and the set of all edges from one vertex will mark the list of recognition stimuli for her condition. As a result, for a full description, it is proposed to use the following entities: "context" (or "state"), "stimulus" (or "event"), "reaction" of the system in accordance with the stimulus.
The principles of constructing reflex systems based on the non-force interaction theory are adapted for the formalization of voice information in systems of dispatch control of vehicle traffic.

The \textbf{third chapter} <<Methods of formalizing voice information in dispatch control systems of vehicle traffic>> is devoted to the creation of a model and methods for formalizing voice information in dispatch control systems of vehicle traffic.
The proposed classification of reactions for the distribution companies <<depot -- road -- delivery point>> is based on the collected statistical data and comments on the process of delivery of various goods by motor vehicles in the leading logistics companies of Ukraine.
The model of voice interaction of distribution entities in dispatch control systems of vehicle traffic is developed, which is presented as a complete communication tree graph of all stages of distribution.
The list of unique contexts of voice interaction is highlighted, the formalization of the voice information in which can occur independently, which reduces the number of reactions for recognition.
A method for formalizing the voice information in vehicle dispatching support systems using intelligent reflex systems is developed, which allows to automate the voice interaction of distribution entities by avoiding the translation of audio information into lexical text due to the use of two main modules (automatic phonetic stenograph and kernal of the reflex voice-activated control system).
For the implementation of the kernal component of the reflex voice-activated control system, a dual system for the classification of voice commands is proposed, which can be configured to the subject area and use the method of intelligent reflex systems or the method of convolution neural networks, depending on which shows the best results.
The method of convolution neural networks is applied to phonemic text for the purpose of classification of voice commands for the formalization of voice information in dispatch control systems of motor transport.
For the formalization of the processes of interaction, the method of intellectual reflex systems is represented in terms of neural networks, which makes it possible to obtain the optimal values of the parameters of the reflex systems by trainig using the backpropagation method.

In the \textbf{fourth chapter} <<Tools for formalization of voice information in systems of dispatch control of vehicle traffic>> describes an experimental research and introduction of tools for formalization of voice information in systems of dispatch control of vehicle traffic.
The developed tool for formalizing voice information in the form of a mobile application for the Android system allows the driver to not distract from driving and monitor road conditions that can accelerate the delivery of products during the distribution, as well as increase the level of security.
The specifics of the use of the developed tools for formalization of voice information in dispatch control systems of motovehicle traffic have been shown that the driver of the car, who will carry out the delivery of products in the process of distribution, and for the first time will face a means of formalization of voice information acting within the system of dispatch control of vehicle traffic, it is necessary to pre-check and sometimes to learn the system to recognize his voice commands from relevant contexts.
Testing the simulation of the recognition of commands based on the iterative process of data collection and the introduction of new evaluation criteria, if the previous ones did not provide sufficient accuracy of evaluation, can provide a process for comparing the effectiveness of different classification methods in the dual model of voice interaction formalization.
An expanded set of metrics for assessing the effectiveness of classification models was used, which included, besides estimating the accuracy, the robust metrics for the unbalanced datasets (precision, recall, F-score), and a visual analysis of confussion matrices.
The evaluation of the effectiveness of the dual system for the formalization of voice information was conducted experimentally in three stages: on the first stage of the primary modeling, the need to increase the number of input data was identified; on the second stage, the hypothesis of insufficient number of input data is checked; on the third --- the hypothesis of insufficient quality of the sound signal. Acceptable for practical use, the level of accuracy in the model built by the method of convolutional neural networks is achieved at the second stage of modeling, and in the model, built by the method of intelligent reflex systems --- on the third.

Actual implementation during the year in three distribution companies confirmed the effectiveness of the developed information technology for the formalization of voice information: vehicle dispatching support system increases the overall efficiency of the delivery process by reducing the number of vehicles needed and increasing the number of points that can be serviced by one vehicle; the introduction of a voice interface can increase the percentage of avoiding or correcting incidents and deviations from the planned route by drivers.

\textbf{Keywords}: intelligent reflex systems; convolution neural network; voice interaction; speech recognition; phonetic text; distribution system; delivery routes; last mile