\documentclass[conference]{IEEEtran}
%\IEEEoverridecommandlockouts
% The preceding line is only needed to identify funding in the first footnote. If that is unneeded, please comment it out.
\usepackage{cite}
\usepackage{amsmath,amssymb,amsfonts}
\usepackage{algorithmic}
\usepackage{graphicx}
\usepackage{textcomp}
\usepackage{xcolor}
\usepackage{hyperref}
\usepackage{filecontents}
\def\BibTeX{{\rm B\kern-.05em{\sc i\kern-.025em b}\kern-.08em
		T\kern-.1667em\lower.7ex\hbox{E}\kern-.125emX}}
\bibliographystyle{IEEEtran}
\begin{filecontents}{cites.bib}
%% LaTeX2e file `cites.bib'
%% generated by the `filecontents' environment
%% from source `conference' on 2019/02/14.
%%



@book{b1,
	author = {Yuriy Akkerman and Mykhaylo Naydonov and Lyubov Naydonova},
	title = {Complex "Analit": interactive calculation of contextual functional-parametric indicators of group reflexivity: a computer program},
	year = {1995},
	url = {http://iris-psy.org.ua/publ/st_0033analit.pdf},
	note = {in russian},
}

@book{b2,
	author = {V. S. Bibler},
	title = {Thinking as creativity (Introduction to the logic of mental dialogue)},
	year = {1975},
	address = {Moscow},
	publisher = {Progress},
	note = {in russian},
}

@book{b3,
	author = {L. V. Grigorovska and M. I. Naydonov and L. A. Naydonova and C. Ya. Treyger},
	title = {Indicators of  imaginative thinking and musical environment of adolescents: a program for calculating the functions and parameters of speech production},
	year = {1989},
	url = {http://iris-psy.org.ua/publ/st_0009calc.pdf},
	note = {in russian},
}

@incollection{b4,
	author = {K. Dunker},
	title = {Psychology of productive (creative) thinking},
	booktitle = {Psychology of thinking: compilation; translated from german and english},
	year = {1965},
	pages = {86--234},
	address = {Moscow},
	publisher = {Progress},
	note = {in russian},
}

@book{b5,
	author = {A. G. Dyachko and A. V. Markov and M. N. Markov and L. A. Naydonova and M. I. Naydonov and I. N. Semenov and S. Yu. Stepanov},
	title = {Calculation of indicators of creative thinking of students in group problem solving. Dep. In OfAP Minvuz USSR 22.04.1987, № М87081},
	journal = {Teaching and teacher education},
	year = {1987},
	address = {Moscow},
	url = {http://iris-psy.org.ua/publ/st_0005.pdf},
	note = {in russian},
}

@book{b7,
	author = {M. I. Naydonov and L.A.Naydonova and Yu. D. Akkerman},
	title = {Reflexive psychologist workstation },
	journal = {Teaching and teacher education},
	year = {1995},
	url = {http://iris-psy.org.ua/publ/st_0033arm.pdf},
	note = {in russian},
}

@book{b8,
	author = {Mykhaylo Naydonov},
	title = {Group reflection in creative tasks solving on condition of varying degrees of readiness for intellectual work : PhD dissertation : 19.00.01, 19.00.03},
	year = {1989},
	address = {K.},
	publisher = {NAES of Ukraine, Institute of Psychology by G. S. Kostiuk},
	numpages = {239},
	url = {http://iris-psy.org.ua/publ/000-163.pdf},
	note = {in ukrainian},
}


@book{b9,
	author = {A. V. Markov and M. I. Naydonov and L. A. Naydonova and S. Yu.  Stepanov},
	title = {The complex of calculation of functional-parametric indicators of group reflection },
	year = {1989},
	url = {http://iris-psy.org.ua/publ/st_0008calc.pdf},
	note = {in russian},
}

@book{b10,
	author = {M. I. Naydonov and L. A. Naydonova and Yu. Repetskiy},
	title = {The program for calculating indicators of reflexivity in the group reflexive process},
	year = {1991},
	url = {http://iris-psy.org.ua/publ/st_0017calc.pdf},
	note = {in russian},
}

@book{b14,
	author = {I. M. Naydonov and M. I. Naydonov},
	title = {The algorithm of content, semantic, and resource definition of answers of respondents for monitoring of social situation of personal and social development: computer program},
	year = {2009},
	url = {http://iris-psy.org.ua/publ/st_0077arm.pdf},
	note = {in ukrainian},
}


@book{b15,
	author = {Mykhaylo Naydonov},
	title = {Formation of reflexive management in organizations : dissertation of the doctor of psychological science : 19.00.05},
	year = {2010},
	address = {K.},
	publisher = {NAES of Ukraine,
		Institute of Psychology by G. S. Kostiuk},
	numpages = {434},
	url = {http://iris-psy.org.ua/publ/DDM.pdf},
	note = {in ukrainian},
}

@book{b16,
	author = {Mykhaylo І. Naydonov},
	title = {Formation of reflexive management in organizations},
	year = {2008},
	address = {K.},
	publisher = {Millennium},
	numpages = {484},
	url = {http://iris-psy.org.ua/publ/fruo.pdf},
	note = {in ukrainian},
}
@patent{b17,
	author = {I. M. Naydonov and M. I. Naydonov and L. A. Naydonova},
	title = {Questionnaire of professional self-realization on the basis of general socio-psychological indicators. NRC: Competence, Levels of Qualification (Professional Self-Realization, NQF): computer program},
	number = {64079},
	date = {2015-12-15},
	location = {ukraine},
	type = {copyright certificate},
	note = {in ukrainian},
}
@patent{b18,
	author = {Ivan Naydonov and Mykhaylo Naydonov and Lyubov A. Naydonova and Liubov Grygorovska},
	title = {Electronic Bulletin <<Monitoring of Professional Self-Execution in the Implementation of the National Qualifications Framework>> (Bulletin Generation, NQF): computer program},
	number = {64077},
	date = {2015-12-15},
	location = {ukraine},
	type = {copyright certificate},
	note = {in ukrainian},
}
@patent{b19,
	author = {I. M. Naydonov and M. I. Naydonov and L. A. Naydonova and L. V. Grigorovska},
	title = {Test of implicit associations of professional self-realization on-line (Professional self-realization, NQF, IAT): computer program},
	number = {64078},
	date = {2015-12-15},
	location = {ukraine},
	type = {copyright certificate},
	note = {in ukrainian},
}
@incollection{b20,
	author = {I. N. Semenov},
	title = {The method of studying the personality in group during a collective solution of creative tasks },
	booktitle = {Personality in a psychological experiment},
	year = {1973},
	pages = {94--115},
	address = {M.},
}

@article{b21,
	author = {Jones, E. E. and Harris, V. A.},
	title = {The attribution of attitudes},
	journal = {Journal of Experimental Social Psychology},
	year = {1967},
	volume = {1},
	number = {2},
	pages = {1--24},
	doi = {10.1016/0022-1031(67)90034-0},
	url = {http://doi.org/10.1016/0022-1031(67)90034-0}
}


@CONFERENCE{b22,
	author = {Naydonov, Ivan},
	title = {Geoinformation system of vehicle routing and parameters of voice interaction of subjects of logistics},
	booktitle = {16th EAGE International Conference on Geoinformatics - Theoretical and Applied Aspects},
	year = {2017},
	doi = {10.3997/2214-4609.201701807},
	url = {http://doi.org/10.3997/2214-4609.201701807}
	language = {english},
}
@article{b23,
	title={Sense tagging: Semantic tagging with a lexicon},
	author={Wilks, Yorick and Stevenson, Mark},
	journal={arXiv preprint cmp-lg/9705016},
	year={1997}
}
@inproceedings{b24,
	title={Feature-rich part-of-speech tagging with a cyclic dependency network},
	author={Toutanova, Kristina and Klein, Dan and Manning, Christopher D and Singer, Yoram},
	booktitle={Proceedings of the 2003 Conference of the North American Chapter of the Association for Computational Linguistics on Human Language Technology-Volume 1},
	pages={173--180},
	year={2003},
	organization={Association for Computational Linguistics}
}
@article{b25,
	title={Speech and non-speech identification and classification using KNN algorithm},
	author={Priya, T Lakshmi and Raajan, NR and Raju, N and Preethi, P and Mathini, S},
	journal={Procedia engineering},
	volume={38},
	pages={952--958},
	year={2012},
	publisher={Elsevier},
	doi = {10.1016/j.proeng.2012.06.120},
	language = {english},
}
@article{b26,
	title={Computer-assisted topic classification for mixed-methods social science research},
	author={Hillard, Dustin and Purpura, Stephen and Wilkerson, John},
	journal={Journal of Information Technology \& Politics},
	volume={4},
	number={4},
	pages={31--46},
	year={2008},
	publisher={Taylor \& Francis}
}
\end{filecontents}
\begin{document}
	
	\title{Artificial intelligence for a specialized reconstruction of thinking}
	
	
	\author{\IEEEauthorblockN{1\textsuperscript{st} Mykhaylo Naydonov}
		\IEEEauthorblockA{\textit{Institute of Reflective Investigations and Specialization} \\
			Kyiv, Ukraine \\
			iris\_psy@ukr.net}
		\and
		\IEEEauthorblockN{2\textsuperscript{nd} Ivan Naydonov}
		\IEEEauthorblockA{\textit{The Department of Technology Management} \\
			\textit{Taras Shevchenko National University of Kyiv}\\
			Kyiv, Ukraine \\
			samogot@gmail.com}
		\and
		\IEEEauthorblockN{3\textsuperscript{rd} Lyubov A. Naydonova}
		\IEEEauthorblockA{\textit{Institute of Social and Political Psychology} \\
			\textit{National Academy of Educational Sciences}\\
			Kyiv, Ukraine \\
			mediasicolo@gmail.com}
		\and
		\IEEEauthorblockN{4\textsuperscript{th} Lyubov M. Naydonova}
		\IEEEauthorblockA{\textit{Institute of Gifted Child} \\
			\textit{National Academy of Educational Sciences}\\
			Kyiv, Ukraine \\
			lyunster@gmail.com}
	}
	
	\maketitle
	
	\begin{abstract}
		On the background of the review of a number of software for the reconstruction of thinking in a reflexive approach, a strategy is proposed to achieve the goal of creating a tool for artificial intelligence specializing in the reconstruction of thinking through the methods of machine learning.
		The availability for machine learning of semantic aspects of formalization of thinking is shown. The prospect of interaction of trends in the development of software support for expert activities and machine learning is outlined.
	\end{abstract}
	
	\begin{IEEEkeywords}
		thinking,
		group reflexivity,
		creative task,
		natural language processing,
		semantic tagging,
		speech-act classification,
		thought-act tagging
	\end{IEEEkeywords}
	
	\section{Introduction}
	Research of artificial intelligence (AI) have already moved from the stage when any changes in the direction of solving the computer intellectual problems was ambitiously called artificial intelligence (as a manifestation of machine thinking), to the stage when term AI in most cases means specific tasks of machine learning. For example, in natural language processing there are the task of determining the speaker in a voice recording \cite{b25} or the task of semantically tagging \cite{b23} text in accordance with certain principles (parts of speech \cite{b24}, topics \cite{b26}, etc.)
	
	\section{The problem}
	
	The problem lies in uncertainty of personal senses of messages which make the task of machine learning in determination and processing of the meaning more difficult than the processing of the content in the message. This tasks are at the moment outside the subject of developments of inner personal senses. Therefore, the interaction of AI professional with the subject field, which has an achievement in separating from the message the semantic part (personal senses and substantive content) may indicate the prospect of future AI research.
	
	The breakthrough in the division of semantic and substantive parts of discursive thinking took place in the 70s and 80s of the 20th century by the followers of K. Dunker \cite{b4}, which later took shape first as a normative analysis, then as a psychology of reflection \cite{b20}.
	
	For this direction, as the model of thinking was used a small creative tasks (SCT), which contain implicit conditions. They act as a barrier for the usage of known algorithms. So SCT is a model of productive thinking precisely because the solver first need open the method, and then apply the algorithm for solution. Passing the stage of identifying of the implicit conditions by the solver, qualifies as a stereotype. The articulation of difficulties opens the access to the meanings of the subject for researcher through imposition of a certain norm of discursive behavior.
	
	\section{Objectives}
	Therefore, the purpose of this paper is to outline the prospects for the development of machine learning tools for the semantic and substantive parts of the discourse studied in the artificial but prognostic reality of the experiment in the field of psychology, or the formulation of requirements for the development of artificial intelligence for a specialized reconstruction of thinking of researched subject by researcher.
	
	\section{Results and Discussion}
	
	The formulated purpose reveals a substantive part as the development of means for the gradual replacement of the functional of the researcher in the status of a \textbf{public participant} in interaction with the researched subjects and in the status of an \textbf{expert} working with the products of interaction (drawings, discourse, gestures, etc.)
	
	From the goal there follows the task of operationalizing the reality of the researched subject and the researcher which will become the subject of machine learning. The task must be performed in the background of tracking the already existing progress of automatisation of activities of the experimenter-expert.
	
	Researched subject in the context of an experiment has a motivation to test his ability to be productive. Such motivation often is a derivative of the motivation for achievements. The artificial reality of thinking in a voice is usually destroyed by a participant, in the way of silent thinking, despite the presence of instructions \cite[178]{b18}.
	
	Modification of the experiment as a group \cite{b8}, where deployed communication is a necessity, while the group decision is not only a "decorating" factor, but also creates a really more precise model of thinking because of its principally dialogic nature \cite{b2}, leds to more natural fealing of the situation.
	
	In addition, the thinking of a person is always simultaneously autonomous and conscious, that is, operates with the knowledge of the tribe (leader). Both conditions are caused by different contexts. Subordination is defined by the meaning of co-operation as a derivative of resource constraints, and autonomy from the cognitive path determined by anatomy and physiology \cite{b15}. Both contexts in thought serve as a resource for each other.
	
	Consequently, the purpose of the development of  machine learning tools for semantic and content parts of discourse at this stage is also added to the third type - resource discourse.
	
	The description of the actions studied in the context of the experiment due to the reality of the resource, is possible, only through the introduction of regulated participation of the researcher (in the arena of the consciousness of the researcher), which consists of: giving instructions and tasks, adopting a coherent solution, situational regulations ("decide in a voice"; "only joint decisions are taken", "the solution is not accurate", "decide further") and one direct answer "problem has a solution" in the case of such a question.
	
	The arena of the researcher's actions is wider than that available in the observations of the researched subject and has a detailed description \cite[118--120]{b16}, \cite[149--150]{b15}.
	
	Usually such stages as the idea of research (a certain problem area), the preparation of stimulating material is not relevant to automation.
	
	At the same time 1) the organization of the studied groups; 2) support of the group decision process; 3) transformation of the product of the activity of the researched subjects in the protocol (text) of the event; 4) dismemberment of the flow of the reality of language-thought into separate conceptual constructs; 5) translation of expert data into indicators; 6) analysis and interpretation of the indicators are available to varying degrees of potential for automation.
	
	In this case, the algorithms for coordinating the procedures and results of the actions of experts (the aspect of qualifying admission to the fourth stage) in the implementation of artificial intelligence provides not only an increase in labor productivity, but also benefits from avoiding the effect of a fundamental attribution error \cite{b21}. At the same time, the requirements for the accuracy of the criteria for machine learning must be evidentiary in order to avoid and reverse the benefits. If the data used for training were attributed with errors, then the resulting model can give answers with a certain bias, which will negatively affect the expert (operator) of the system ("the machine knows better").
	
	The given estimations of the possibility of creating artificial intelligence for the automation of individual stages were realized on the background of the current advancement of the transfer of the functional of the researcher and expert to the computer. \cite{b5,b9,b3,b10,b1,b7,b14} The logic of this transfer corresponded, on the one hand, to the most laborious computational processes \cite{b5,b9,b3,b10,b1}, on the other hand, - optimization of the work of the researcher \cite{b7,b14}. The calculation programs first exhausted the nominations of all possible calculations, and then opened the prospects of solutions that without a computer could not have occurred in principle, within the capabilities of the investigator's operations. It is these decisions that include the dynamics of the thinking process \cite{b5}. They represent the picture of the solution and from the point of view of the mutual arrangement of the senses, meanings and resources, but also helped to discover trends in the understanding of thinking, which gave rise to new ideas about mechanisms such as the discharge of sensory tissue, reverse subjective reconstruction, etc. \cite[123]{b15}.
	
	The development of software tools for optimizing the work of the researcher was primarily due to the large load of the cognitive system due to the need to classify the discourse by a large number of conceptual indicators (80 finite elements grouped into a classification, each of which can simultaneously be in 40 dispositions). Without mediating the computer with a specially designed environment for the cognitive system of the researcher, such work could only take place through the repetition of the examination cycles of each act of thinking. Given that each sentence could consist of one to six acts, such a process was extremely laborious.
	
	Two iterations of the development of the type "psychologist's workstation" \cite{b7,b14} had not only their distinction in the specialization of work in the DOS environment and Windows environment. The second implementation has advantages in the form of an auto compilation for functions and the ability to work with the clipboard.
	
	As in the previous cases, the new opportunities generated the effects of additional indicators.
	
	From 1973 to 1995, to establish the relationship and differences in the data in various research series, absolute values were translated into percentages. To establish the action of the reflexive mechanism in the specified conditions. Applications to the workplace of the type "analyte" \cite{b1} have made it possible to use not a formal mathematical method, but ontologically more closely related to the act of thinking of applications analogue time.
	
	Establishing a smaller scale of thinking acts by virtue of their access to processing in a medium with a specialized interface is better to begin to process the reality of the interrupted communication of a mental act and the coexistence of mental acts that occurs in parallel vocal trajectories articulated by various participants or subjects.
	
	It is logical that the binding of indicators not to the artificial analog time, but to the actual, would give greater accuracy of scientific solutions, but such a task, again, can only be done by the computer partially or completely - depending on the complexity of the algorithms.
	
	The logical conclusion from the forecast of the possibility of creating artificial intelligence, as well as the result of the work, may be the judgment of the prospects of developing this tool of artificial intelligence. These perspectives are best described without specializing them on the reality of the researched subject and researcher. Thus, our prototype for the replacement of a recruiter and an experimenter to participate in the study may be our developments used to ensure the monitoring of public opinion \cite{b17,b18,b19}. The number of classes of speech recognition in the task of research regulation  is an order of magnitude lower than, for example, in the distribution management problem \cite{b22}. In this way, you can exclude a human experimenter from the experiment procedure. Tasks related to the expert function can be divided into two stages. The first is the assistant level of the researcher with the functional support of the decision making projects for the choice of the researcher, and the second - with a limited functionality of decision making and an offer to the researcher to skip several steps in their own decision-making.
	
	Later, this final decision function can be passed to the software.
	
	Given the number of regulatory options for the implementation of the conceptual structure, the automatic definition of some functions will give a much higher level of error than others. For example, functions of the component of intellectual reflection ("good", "natural", "elementary"), the operating part of the subject component ("6 plus 6 will be 12"), the resource scope ("tell me", "to believe", "tell more"; "do not even remember") will be recognized better, because the words and phrases that they denote are limited (all the given examples are partial).
	The substantive component and the component of personal reflection will be the most difficult to define ("Should we surrender, as a last resort?", "And what if the opinion of one against two?", "Is there another version?", "You think how to decide, all at once ?, "With what to begin, yes?", "What side to approach?", "How should we understand a condition?", "How to judge it?"), due to the multivariable of the subject layer and the tendency of people to be distracted from the subject of the task in question. That is, a large sample of input data will require to learn the system to recognize these functions. After exhausting the possibilities of creating standards for simple small creative tasks, it is possible to develop an adaptation module within the framework of specially designed tasks - tests for special subject fields or to create the obligatory test effect by experienced research subjects.
	
	The task of automating the semantic tagging of thinking functions can also be called semantic tagging of thoughtful acts. The task of speech acts classification is the most similar conceptually to this idea. The differences are in two areas. First, as the name implies, the task of semantic tagging is that the acts must first be determined in the text, and then for each of them determine the function of the act and other parameters. The task of classification in this regard is more simple because units for classification are already known.
	
	The second area is the difference between speech acts and thoughtful acts. In the case of the task of speech acts classification, the task is to determine the purpose utterance being said, so the classification takes place according to known units of a sentences or an entire dialog replicas. In the task of thoughtful acts tagging, by analyzing speech, we analyze the thinking of a person, therefore, various intonation words, exclamations, torn words and sentences, the transition of thinking from one person to another or the coexistence of speech flows etc. are very important, because they help to reconstruct the thought process of a person. That is why the task stands as a tagging, not as a classification, because these units of thoughtful acts are also needed to be determined.
	
	In terms of technical implementation, the task of parts of speech tagging or other tasks of semantic tagging will be more similar, because despite the other subject area, the technical task in them is the same - to classify each word among a certain set of classes and taking into account the context of the adjacent words, then combine words with the same classes in the tags. The difference in subject areas will be expressed only in the difference between the set of classes and the attribution of input data.
	
	\section{Conclusion}
	
	Therefore, analysis and comparison with other tasks of natural language processing as a precondition for achieving the goal was carried out. The possibility of developing the means of machine learning for the semantic as personal sense and substantive parts of the discourse is approved in the experimental (artificial, but prognostic) reality of the laboratory experiment in the field of psychology. Requirements for the development of artificial intelligence for the specialized analytical reconstruction of thinking by the researcher are formulated, also prospects for phased solving of AI problems were identified.
	
	\bibliography{cites}
	
\end{document}
