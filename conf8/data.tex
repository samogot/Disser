
\newcommand{\articleUDK}{656.073, 004.934}

\newcommand{\articleTitleUkr}{
Шляхи використання можливостей голосового управління для оптимізації процесів дистрибуції}

\newcommand{\articleTitleRus}{
Проблема голосового взаимодействия в задачах управления дистрибуцией}

\newcommand{\articleTitleEng}{
The problem of voice interaction in the distribution management tasks}

\newcommand{\annotationUkr}{
Стаття присвячення проблемі автоматизації голосової взаємодії в задачах управління дистрибуцією. Проведено аналіз вітчизняних та зарубіжних літературних джерел для пошуку шляхів використання можливостей голосового управління для оптимізації процесів дистрибуції за трьома напрямами: системи управління дистрибуцією, системи розпізнання голосу та системи голосової взаємодії. Пропонується інтегративна модель, в якій поєднується принцип написання дерева можливих сценаріїв взаємодії та рефлекторна система голосового управління.
}

\newcommand{\annotationRus}{
Статья посвящена проблеме автоматизации голосового взаимодействия в задачах управления дистрибуцией. Проведен анализ отечественных и заграничных литературных источников для поиска путей использования возможностей голосового управления для оптимизации процессов дистрибуции по трем направлениям: системы управления дистрибуцией, системы распознавания голоса и системы голосового взаимодействия. Предлагается интегративная модель, в которой объединяется принцип написания дерева возможных сценариев взаимодействия и рефлекторная система голосового управления.
}

\newcommand{\annotationEng}{
The problem of automating voice interaction in distribution management is analyzing in the article. The main goal is to find domestic and foreign experience and single out ways to use voice control capabilities to optimize distribution processes. An analysis of domestic and foreign literature gave the ways to use voice control capabilities to optimize distribution processes in three areas: distribution management systems, voice recognition systems and voice interactions. It is discovered the limits of GPS monitoring systems which are used to control of the delivery processes, but do not reflect the causes of real situation deviation from the planned route. To overcome these limits, the direction of voice interaction system for management tasks in distribution is proposed. Available traditional voice recognition systems based on neural networks and hidden Markov models do not provide the required level for distribution because of resistance to noise, require powerful hardware or stable access to the Internet. There is proposed two new principles of resolving the problem — the transition to another speech recognition unit and building a tree of possible scenarios of interaction to reduce the amount necessary to recognize commands, depending on the context of the situation. Thus, the proposed integrative model of reflex voice interaction in distribution management, which combines two articulated principles. The prospect of further research is to develop a software which be realized proposed model, and to integrate it with existing authoring system of automation routing.
}

\newcommand{\keywordsUkr}{
голосове управління, голосова взаємодія, управління дистрибуцією, розпізнання мови}

\newcommand{\keywordsRus}{
голосовое управление, голосоваое взаимодействие, управление дистрибуцией, разпознание речи}

\newcommand{\keywordsEng}{
voice control, voice interaction, distribution management, speech recognition}
